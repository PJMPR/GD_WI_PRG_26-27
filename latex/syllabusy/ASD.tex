% ============================================================
%  Sylabus: Algorytmy i struktury danych (ASD)
%  PJATK Filia w Gdańsku – studia stacjonarne
% ============================================================
\documentclass[12pt, a4paper]{article}

\usepackage[T1]{fontenc}
\usepackage[utf8]{inputenc}
\usepackage[polish]{babel}
\usepackage{lmodern}
\usepackage{microtype}
\usepackage[a4paper, top=2.5cm, bottom=2.5cm, left=2.5cm, right=2.5cm]{geometry}
\usepackage{xcolor}
\usepackage{graphicx}
\usepackage{booktabs}
\usepackage{tabularx}
\usepackage{longtable}
\usepackage{multirow}
\usepackage{array}
\usepackage{colortbl}
\usepackage{enumitem}
\usepackage{fancyhdr}
\usepackage{titlesec}
\usepackage{mdframed}
\usepackage[colorlinks=true, linkcolor=red!70!black, urlcolor=red!70!black]{hyperref}
\usepackage{eso-pic}   % watermark na każdej stronie
\usepackage{transparent} % przezroczystość watermarku

% ---------- Kolory PJATK -------------------------------------
\definecolor{pjatkRed}{RGB}{180,0,0}
\definecolor{pjatkGray}{RGB}{80,80,80}
\definecolor{pjatkLightGray}{RGB}{245,245,245}
\definecolor{tableHeader}{RGB}{220,220,220}

% ---------- Nagłówki/stopki ----------------------------------
\pagestyle{fancy}\fancyhf{}
\renewcommand{\headrulewidth}{0.4pt}
\renewcommand{\footrulewidth}{0.4pt}
\fancyhead[L]{\small\textcolor{pjatkGray}{PJATK -- Filia w Gdańsku \textbar\ Informatyka}}
\fancyhead[R]{\small\textcolor{pjatkGray}{Sylabus: ASD}}
\fancyfoot[C]{\small\thepage}

% ---------- Sekcje -------------------------------------------
\titleformat{\section}{\large\bfseries\color{pjatkRed}}{\thesection.}{0.5em}{}
  [\color{pjatkRed}\rule{\linewidth}{0.8pt}]
\setlist{noitemsep, topsep=3pt, parsep=2pt}

% ---------- Środowisko ramki informacyjnej -------------------
\newmdenv[linecolor=pjatkRed, linewidth=1.2pt, backgroundcolor=pjatkLightGray,
  innerleftmargin=10pt, innerrightmargin=10pt, innertopmargin=8pt,
  innerbottommargin=8pt, roundcorner=4pt]{infobox}

% ============================================================
\begin{document}

% ---------- Watermark na każdej stronie ----------------------
\AddToShipoutPictureBG{%
  \AtPageCenter{%
    \makebox(0,0){%
      \transparent{0.07}%
      \includegraphics[width=14cm]{../PJATK_pl_sygnet_transparent-eps-converted-to}%
    }%
  }%
}

% ---------- Nagłówek tytułowy --------------------------------
\begin{center}
  \includegraphics[height=2cm]{../PJATK_pl_poziom_1}\\[0.8cm]
  {\LARGE\bfseries\color{pjatkRed} SYLABUS PRZEDMIOTU}\\[0.8cm]
\end{center}

\begin{infobox}
\begin{tabularx}{\textwidth}{@{}lX@{}}
  \textbf{Nazwa przedmiotu:}  & {\bfseries Algorytmy i struktury danych} \\[3pt]
  \textbf{Kod przedmiotu:}    & ASD \\[3pt]
  \textbf{Kierunek / Profil:} & Informatyka / praktyczny \\[3pt]
  \textbf{Tryb studiów:}      & stacjonarny \\[3pt]
  \textbf{Rok / Semestr:}     & 2 / 3 \\[3pt]
  \textbf{Charakter:}         & obowiązkowy \\[3pt]
  \textbf{Odpowiedzialny:}    & dr hab. Marek A. Bednarczyk \\[3pt]
  \textbf{Wersja z dnia:}     & 19.02.2026 \\
\end{tabularx}
\end{infobox}

\vspace{1cm}

% ============================================================
\section{Godziny zajęć i punkty ECTS}

\begin{center}
\begin{tabular}{|>{\centering\arraybackslash}p{2.2cm}
                |>{\centering\arraybackslash}p{2.2cm}
                |>{\centering\arraybackslash}p{2.2cm}
                |>{\centering\arraybackslash}p{2.5cm}
                |>{\centering\arraybackslash}p{2.5cm}
                |>{\centering\arraybackslash}p{2.2cm}
                |>{\centering\arraybackslash}p{1.5cm}|}
\hline
\rowcolor{tableHeader}
\textbf{Wykłady} & \textbf{Ćwiczenia} & \textbf{Laboratorium} &
\textbf{Z prowadzącym} & \textbf{Praca własna} & \textbf{Łącznie} & \textbf{ECTS} \\
\hline
30 h & --- & 30 h & 60 h & 65 h & 125 h & \textbf{5} \\
\hline
\end{tabular}
\end{center}

% ============================================================
\section{Forma zaliczenia}

\begin{tabular}{ll}
  \toprule
  \textbf{Forma zajęć} & \textbf{Sposób zaliczenia} \\
  \midrule
  Wykład      & Egzamin \\
  Laboratorium & Zaliczenie z oceną \\
  \bottomrule
\end{tabular}

% ============================================================
\section{Cel dydaktyczny}

Opanowanie umiejętności analizy programów iteracyjnych i rekursywnych oraz
wykorzystania aparatu matematycznego do analizy złożoności algorytmów; wiedza
na temat złożoności problemów algorytmicznych, klas złożoności i ich praktycznego
znaczenia; analiza złożoności algorytmów sortujących i wyszukujących; drzewa,
stosy, kolejki i tablice z haszowaniem; programowanie dynamiczne i algorytmy
zachłanne.

% ============================================================
\section{Przedmioty wprowadzające}

\begin{tabularx}{\textwidth}{lX}
  \toprule
  \textbf{Przedmiot} & \textbf{Wymagane zagadnienia} \\
  \midrule
  PRG1 -- Programowanie           & Programowanie obiektowe (POJ) \\
  ALG -- Algebra liniowa          & AM -- Analiza matematyczna \\
  MAD -- Matematyka Dyskretna     & Algorytmy iteracyjne i rekursywne \\
  \bottomrule
\end{tabularx}

% ============================================================
\section{Treści programowe}

\begin{enumerate}
  \item Algorytmy i problemy algorytmiczne; złożoność czasowa i pamięciowa
  \item Analiza złożoności problemu wyszukiwania, notacje asymptotyczne
  \item Specyfikowanie problemów, własność stopu, problemy nierozstrzygalne
  \item Logika Hoare'a i metody dowodzenia własności stopu
  \item Proste algorytmy sortujące
  \item Quicksort; algorytmy sortujące w czasie liniowym
  \item Programowanie dynamiczne i zachłanne
  \item Kolejki, stosy, listy i drzewa
  \item Tablice z haszowaniem
  \item Twierdzenie o rekursji uniwersalnej
  \item Drzewa BST, drzewa czerwono-czarne
  \item Algorytmy grafowe
  \item Podstawy teorii złożoności, klasy złożoności
  \item Języki formalne -- języki regularne i kontekstowe
\end{enumerate}

% ============================================================
\section{Efekty kształcenia}

\subsection*{Wiedza}
\begin{itemize}
  \item Student zna i rozumie pojęcia: specyfikacja, weryfikacja i własność stopu algorytmu
  \item Student rozumie różnicę między algorytmami iteracyjnymi i rekursywnymi
  \item Student zna istotę podstawowych struktur danych i potrafi dobrać je do zadania
  \item Student zna notację asymptotyczną opisującą złożoność obliczeniową
  \item Zna pojęcie złożoności problemu algorytmicznego oraz klasy złożoności
\end{itemize}

\subsection*{Umiejętności}
\begin{itemize}
  \item Student potrafi dokonać analizy złożoności programów iteracyjnych
  \item Potrafi przeanalizować złożoność programów opartych o zasadę ,,dziel i zwyciężaj''
  \item Student potrafi dobrać struktury danych odpowiednie do zadania
  \item Student potrafi oszacować złożoność algorytmu
\end{itemize}

% ============================================================
\section{Kryteria oceny}

\begin{itemize}
  \item Na ćwiczeniach odbędą się dwa kolokwia oraz oceniane aktywności dodatkowe
  \item Student musi uzyskać minimum 50\% punktów
\end{itemize}

\vspace{6pt}
\noindent\textbf{Skala ocen:}

\begin{tabular}{ll}
  \toprule
  \textbf{Próg} & \textbf{Ocena} \\
  \midrule
  poniżej 50\% & niedostateczny \\
  50--59\%     & dostateczny \\
  60--69\%     & dostateczny+ \\
  70--79\%     & dobry \\
  80--89\%     & dobry+ \\
  90--100\%    & bardzo dobry \\
  \bottomrule
\end{tabular}

% ============================================================
\section{Metody dydaktyczne}

\textbf{Wykład:}
\begin{itemize}
  \item Wykład problemowy poświęcony analizie kolejnych zagadnień
  \item Rozwiązywanie zadań
\end{itemize}

\textbf{Laboratorium:}
\begin{itemize}
  \item Samodzielna implementacja algorytmów
  \item Analiza złożoności gotowych rozwiązań
\end{itemize}

% ============================================================
\section{Literatura}

\textbf{Podstawowa:}
\begin{itemize}
  \item T. Cormen, Ch. Leiserson, R. Rivest, C. Stein.
        \emph{Wprowadzenie do algorytmów},
        Wydawnictwo Naukowe PWN, 2022
\end{itemize}

\textbf{Uzupełniająca:}
\begin{itemize}
  \item A. Szepietowski. \emph{Matematyka dyskretna}, Wydawnictwo UG, 2004
  \item G. Mirkowska. \emph{Algorytmy i struktury danych}, PJWSTK, 2006
  \item G. Mirkowska. \emph{Algorytmy i struktury danych -- Zadania}, PJWSTK, 2006
  \item Z. Michalewicz. \emph{Algorytmy genetyczne + struktury danych}, PJWSTK, 2003
\end{itemize}

\end{document}

