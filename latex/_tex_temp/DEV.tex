% ===========================================================
%  Sylabus: Technologie DevOps (DEV)
% ===========================================================
\documentclass[12pt, a4paper]{article}

\usepackage[T1]{fontenc}
\usepackage[utf8]{inputenc}
\usepackage[polish]{babel}
\usepackage{lmodern}
\usepackage{microtype}
\usepackage[a4paper, top=2.5cm, bottom=2.5cm, left=2.5cm, right=2.5cm]{geometry}
\usepackage{xcolor}
\usepackage{graphicx}
\usepackage{booktabs}
\usepackage{tabularx}
\usepackage{longtable}
\usepackage{multirow}
\usepackage{array}
\usepackage{colortbl}
\usepackage{enumitem}
\usepackage{fancyhdr}
\usepackage{titlesec}
\usepackage{mdframed}
\usepackage[colorlinks=true, linkcolor=red!70!black, urlcolor=red!70!black]{hyperref}
\usepackage{eso-pic}
\usepackage{tikz}

\definecolor{pjatkRed}{RGB}{180,0,0}
\definecolor{pjatkGray}{RGB}{80,80,80}
\definecolor{pjatkLightGray}{RGB}{245,245,245}
\definecolor{tableHeader}{RGB}{220,220,220}

\pagestyle{fancy}\fancyhf{}
\renewcommand{\headrulewidth}{0.4pt}
\renewcommand{\footrulewidth}{0.4pt}
\fancyhead[L]{\small\textcolor{pjatkGray}{PJATK -- Filia w Gdańsku \textbar\ Informatyka}}
\fancyhead[R]{\small\textcolor{pjatkGray}{Sylabus: DEV}}
\fancyfoot[C]{\small\thepage}

\titleformat{\section}{\large\bfseries\color{pjatkRed}}{\thesection.}{0.5em}{}
  [\color{pjatkRed}\rule{\linewidth}{0.8pt}]
\setlist{noitemsep, topsep=3pt, parsep=2pt}

\newmdenv[linecolor=pjatkRed, linewidth=1.2pt, backgroundcolor=pjatkLightGray,
  innerleftmargin=10pt, innerrightmargin=10pt, innertopmargin=8pt,
  innerbottommargin=8pt, roundcorner=4pt]{infobox}

\begin{document}

\AddToShipoutPictureBG{%
  \begin{tikzpicture}[remember picture, overlay]
    \node[opacity=0.5] at (current page.center) {%
      \includegraphics[width=14cm]{C:/Users/adamu/WebstormProjects/pj-studies/latex/PJATK_pl_sygnet_transparent-eps-converted-to}%
    };
  \end{tikzpicture}%
}

\begin{center}
  \includegraphics[height=2cm]{C:/Users/adamu/WebstormProjects/pj-studies/latex/PJATK_pl_poziom_1}\\[0.8cm]
  {\LARGE\bfseries\color{pjatkRed} SYLABUS PRZEDMIOTU}\\[0.8cm]
\end{center}

\begin{infobox}
\begin{tabularx}{\textwidth}{@{}lX@{}}
  \textbf{Nazwa przedmiotu:}  & {\bfseries Technologie DevOps} \\[3pt]
  \textbf{Kod przedmiotu:}    & DEV \\[3pt]
  \textbf{Kierunek / Profil:} & Informatyka / praktyczny \\[3pt]
  \textbf{Tryb studiów:}      & niestacjonarny \\[3pt]
  \textbf{Rok / Semestr:}     & 4 / 7 \\[3pt]
  \textbf{Charakter:}         & obowiązkowy \\[3pt]
  \textbf{Odpowiedzialny:}    & mmiotk@pjwstk.edu.pl \\[3pt]
  \textbf{Wersja z dnia:}     & 19.02.2026 \\
\end{tabularx}
\end{infobox}

\vspace{1cm}

\section{Godziny zajęć i punkty ECTS}

\begin{center}
\begin{tabular}{|>{\centering\arraybackslash}p{2.0cm}
                |>{\centering\arraybackslash}p{2.0cm}
                |>{\centering\arraybackslash}p{2.0cm}
                |>{\centering\arraybackslash}p{2.4cm}
                |>{\centering\arraybackslash}p{2.4cm}
                |>{\centering\arraybackslash}p{2.0cm}
                |>{\centering\arraybackslash}p{1.4cm}|}
\hline
\rowcolor{tableHeader}
\textbf{Wykłady} & \textbf{Ćwiczenia} & \textbf{Laboratorium} &
\textbf{Z prowadzącym} & \textbf{Praca własna} & \textbf{Łącznie} & \textbf{ECTS} \\
\hline
30 h & --- & 30 h & 60 h & 65 h & 125 h & \textbf{5} \\
\hline
\end{tabular}
\end{center}

\section{Forma zajęć}

\begin{tabular}{ll}
  \hline
  \textbf{Forma zajęć} & \textbf{Sposób zaliczenia} \\
  \hline
  Laboratorium & Zaliczenie z oceną \\
  Wykład & Egzamin \\
  \hline
\end{tabular}

\section{Cel dydaktyczny}

Celem przedmiotu jest zapoznanie się z techniką konteneryzacji na podstawie narzędzia Docker oraz Kubernetes. Efektem przedmiotu ma być możliwość utworzenia aplikacji typu REST API, która będzie spakowana za pomocą wyżej wymienionych narzędzi.

\section{Przedmioty wprowadzające}

\begin{tabularx}{\textwidth}{lX}
  \hline
  \textbf{Przedmiot} & \textbf{Wymagane zagadnienia} \\
  \hline
  Użytkowanie komputerów & Systemy operacyjne \\
  Technologie internetu & Umiejętność posługiwania się emulatorem terminala w systemie operacyjnym \\
  Znajomość struktur plików i katalogów w systemie operacyjnym & Znajomość pojęcia procesu w systemie operacyjnym \\
  Znajomość protokołu HTTP & --- \\
  \hline
\end{tabularx}

\section{Treści programowe}

\begin{enumerate}
  \item Wprowadzenie do idei konteneryzacji
  \item Praca oraz monitorowanie kontenerów
  \item Tworzenie obrazów do użycia w kontenerze. Konteneryzacja istniejących aplikacji.
  \item Wolumeny oraz rejestry  w kontenerze
  \item Tworzenie obrazów za pomocą technik multietapowych
  \item Konfiguracja zasobów oraz sieci w kontenerach
  \item Debugowanie kontenerów
  \item Zarządzanie kontenerami na przykładzie narzędzia Docker Compose
  \item Architektura konteneryzacji aplikacji
  \item Pojęcie Continuous Integration
  \item Narzędzie Jenkins jako przykład narzędzia do używania CI/CD
  \item Zasada ciągłej interakcji (CI)
  \item Proces wdrażania aplikcji za pomocą narzędzia Ansible
  \item Proces monitorowania aplikacji w cyklu produkcyjnym
  \item Uruchomienie kontenerezowanej aplikacji w chmurze
\end{enumerate}

\section{Efekty kształcenia}

\subsection*{Wiedza}
\begin{itemize}
  \item Student zna i rozumie pojęcia w zakresie pojęcia wirtualizacji i konteneryzacji z wykorzystaniem sieci komputerowych, ich technologii oraz protokołów komunikacyjnych.
  \item Student zna i rozumie pojęcia występujące w procesie konteneryzacji na przykładzie narzędzia Docker. Zna i rozumie proces tworzenia, skonteneryzowanych bezpiecznych, warstwowych aplikacji internetowych; zna i rozumie pojęcia związane z orchestracją i monitorowaniem takiej aplikacji
\end{itemize}

\subsection*{Umiejętności}
\begin{itemize}
  \item Student potrafi potrafi ocenić przydatność wirtualizacji i konteneryzacji i związanych z nimi środowisk
  \item Student potrafi potrafi wyspecyfikować, zaprojektować, zaimplementować, przetestować oraz zdebuggować aplikcję utworzoną w procesie konteneryzacji za pomocą plików Dockerfile.
  \item Student potrafi wytworzyć warstwową aplikację webową w oparciu o wybrane wzorce architektoniczne i przy pomocy narzędzia Docker utworzyć kontener i umieścić go w technologii chmurowej.
  \item Student potrafi zaplanować i przeprowadzić automatyczny proces tworzenia aplikacji z wykorzystaniem pojęcia konteneryzacji i związanych z nimi narzędzi.
\end{itemize}

\subsection*{Kompetencje społeczne}
\begin{itemize}
  \item Student jest gotów do samodzielnego uczenia się przez całe życie
\end{itemize}

\section{Kryteria oceny}

\begin{itemize}
  \item wykład z elementami dyskusji z prezentacją multimedialną
  \item burza mózgów
  \item rozwiązywanie zadań
  \item analiza przypadków
  \item projekt praktyczny
  \item Kryteria oceny
  \item 100\% Kolokwium pisemne zawierające do 10 zadań.
  \item 40\% Ocena z laboratorium
  \item 60\% Egzamin
\end{itemize}

\section{Metody dydaktyczne}

Wykład, laboratoria, praca własna studenta.

\section{Literatura}

\textbf{Podstawowa:}
\begin{itemize}
  \item S. Kane, Docker: Up \& Running, O’Reilly, 2024.
  \item J. Nickloff, S. Kuenzli, Docker in Action, Manning Publications, 2019.
  \item G. Schenker, Learn Docker – Fundamentals of Docker 19.x, Packt Publishing, 2020.
  \item E. Stoneman, Learn Docker in a Month of Lunches, Manning Publications, 2020.
\end{itemize}

\textbf{Uzupełniająca:}
\begin{itemize}
  \item A. Davis, Bootstraping Microservices with Docker, Kubernetes and Terraform, Manning Publications, 2021.
  \item E. Fouda, A Complete Guide to Docker for Operations and Development, Apress, 2022.
  \item F. Zammetti, Modern Full Stack Development using Typescript, React, Node.js, Webpack and Docker, Apress, 2020.
\end{itemize}

\end{document}
