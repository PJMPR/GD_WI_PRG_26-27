% ===========================================================
%  Sylabus: Procesy Innowacyjne (PRIN)
% ===========================================================
\documentclass[12pt, a4paper]{article}

\usepackage[T1]{fontenc}
\usepackage[utf8]{inputenc}
\usepackage[polish]{babel}
\usepackage{lmodern}
\usepackage{microtype}
\usepackage[a4paper, top=2.5cm, bottom=2.5cm, left=2.5cm, right=2.5cm]{geometry}
\usepackage{xcolor}
\usepackage{graphicx}
\usepackage{booktabs}
\usepackage{tabularx}
\usepackage{longtable}
\usepackage{multirow}
\usepackage{array}
\usepackage{colortbl}
\usepackage{enumitem}
\usepackage{fancyhdr}
\usepackage{titlesec}
\usepackage{mdframed}
\usepackage[colorlinks=true, linkcolor=red!70!black, urlcolor=red!70!black]{hyperref}
\usepackage{eso-pic}
\usepackage{tikz}

\definecolor{pjatkRed}{RGB}{180,0,0}
\definecolor{pjatkGray}{RGB}{80,80,80}
\definecolor{pjatkLightGray}{RGB}{245,245,245}
\definecolor{tableHeader}{RGB}{220,220,220}

\pagestyle{fancy}\fancyhf{}
\renewcommand{\headrulewidth}{0.4pt}
\renewcommand{\footrulewidth}{0.4pt}
\fancyhead[L]{\small\textcolor{pjatkGray}{PJATK -- Filia w Gdańsku \textbar\ Informatyka}}
\fancyhead[R]{\small\textcolor{pjatkGray}{Sylabus: PRIN}}
\fancyfoot[C]{\small\thepage}

\titleformat{\section}{\large\bfseries\color{pjatkRed}}{\thesection.}{0.5em}{}
  [\color{pjatkRed}\rule{\linewidth}{0.8pt}]
\setlist{noitemsep, topsep=3pt, parsep=2pt}

\newmdenv[linecolor=pjatkRed, linewidth=1.2pt, backgroundcolor=pjatkLightGray,
  innerleftmargin=10pt, innerrightmargin=10pt, innertopmargin=8pt,
  innerbottommargin=8pt, roundcorner=4pt]{infobox}

\begin{document}

\AddToShipoutPictureBG{%
  \begin{tikzpicture}[remember picture, overlay]
    \node[opacity=0.5] at (current page.center) {%
      \includegraphics[width=14cm]{C:/Users/adamu/WebstormProjects/pj-studies/latex/PJATK_pl_sygnet_transparent-eps-converted-to}%
    };
  \end{tikzpicture}%
}

\begin{center}
  \includegraphics[height=2cm]{C:/Users/adamu/WebstormProjects/pj-studies/latex/PJATK_pl_poziom_1}\\[0.8cm]
  {\LARGE\bfseries\color{pjatkRed} SYLABUS PRZEDMIOTU}\\[0.8cm]
\end{center}

\begin{infobox}
\begin{tabularx}{\textwidth}{@{}lX@{}}
  \textbf{Nazwa przedmiotu:}  & {\bfseries Procesy Innowacyjne} \\[3pt]
  \textbf{Kod przedmiotu:}    & PRIN \\[3pt]
  \textbf{Kierunek / Profil:} & Informatyka / praktyczny \\[3pt]
  \textbf{Tryb studiów:}      & niestacjonarny \\[3pt]
  \textbf{Rok / Semestr:}     & 4 / 7 \\[3pt]
  \textbf{Charakter:}         & obieralny \\[3pt]
  \textbf{Odpowiedzialny:}    & dr Albert Śledzianowski, albertsledzianowski@pjwstk.edu.pl \\[3pt]
  \textbf{Wersja z dnia:}     & 19.02.2026 \\
\end{tabularx}
\end{infobox}

\vspace{1cm}

\section{Godziny zajęć i punkty ECTS}

\begin{center}
\begin{tabular}{|>{\centering\arraybackslash}p{2.0cm}
                |>{\centering\arraybackslash}p{2.0cm}
                |>{\centering\arraybackslash}p{2.0cm}
                |>{\centering\arraybackslash}p{2.4cm}
                |>{\centering\arraybackslash}p{2.4cm}
                |>{\centering\arraybackslash}p{2.0cm}
                |>{\centering\arraybackslash}p{1.4cm}|}
\hline
\rowcolor{tableHeader}
\textbf{Wykłady} & \textbf{Ćwiczenia} & \textbf{Laboratorium} &
\textbf{Z prowadzącym} & \textbf{Praca własna} & \textbf{Łącznie} & \textbf{ECTS} \\
\hline
8 h & 8 h & --- & 16 h & 34 h & 50 h & \textbf{2} \\
\hline
\end{tabular}
\end{center}

\section{Forma zajęć}

\begin{tabular}{ll}
  \hline
  \textbf{Forma zajęć} & \textbf{Sposób zaliczenia} \\
  \hline
  Ćwiczenia & Zaliczenie z oceną \\
  Wykład & Nieoceniany \\
  \hline
\end{tabular}

\section{Cel dydaktyczny}

Przedmiot “Procesy innowacyjne” ma na celu: Zapoznanie studentów z podstawami zarządzania firmą innowacyjną oraz organizacyjnymi podstawami jej funkcjonowania Zapoznanie studentów z technikami analizy i pobudzania kreatywności oraz twórczego rozwiązywania problemów inżynierskich.

\section{Przedmioty wprowadzające}

\begin{tabularx}{\textwidth}{lX}
  \hline
  \textbf{Przedmiot} & \textbf{Wymagane zagadnienia} \\
  \hline
  POZ – Podstawy Organizacji i Zarządzania znajomość podstaw organizacji i zarządzania & Znajomość podstaw organizacji i zarządzania \\
  \hline
\end{tabularx}

\section{Treści programowe}

\begin{enumerate}
  \item Wykład
  \item Ćwiczenia
  \item Kreatywność menedżerska
  \item Profil osoby kreatywnej i przedsiębiorczej
  \item Istota i motywy przedsiębiorczości gospodarczej
  \item Tworzenie i analiza ścieżki doświadczeń dla wybranego produktu / usługi (praca z szablonem
  \item customer journey map)
  \item Metody graficzne i analityczne w diagnozowaniu
  \item problemu
  \item Odkrywanie możliwości dla produktu innowacyjnego – studium przypadku
  \item Rozpoznawanie szans rynkowych i możliwości
  \item sytuacyjnych
  \item Analiza rozwoju firm innowacyjnych – macierz
  \item kluczowych czynników
  \item Planowanie rozwoju - elementy analizy strategicznej
  \item Wprowadzenie do Design Thinking
  \item Struktury organizacyjne
  \item Wprowadzenie do pracy zespołowej i projektowej
  \item Trening umiejętności pracy w zespole.
  \item Organizowanie procesów i zarządzanie zasobami
  \item Środowiska do komputerowego
  \item wspomagania pracy projektowej i
  \item zespołowej
  \item Kontrolowanie, doskonalenie i zarządzanie jakością
  \item Identyfikacja potrzeb; wprowadzenie do analizy badań zastanych i jakościowych – wywiad,
  \item obserwacja uczestnicząca, netnografia; mapa empatii, Persony
  \item Mała firma – cykl rozwojowy i czynniki sukcesu
  \item Diagnoza problemu:
  \item Metoda 5 x dlaczego
  \item Metoda Pareto-Lorenza i Diagram
  \item Ishikawy
  \item Rysowanie odręczne
  \item Zarządzanie personelem i motywowanie
  \item Metoda Kano
  \item Wybrane techniki oceny wariantów i wspomagania
  \item decyzji
  \item Indywidualne generowanie pomysłów do problemu (mindmapping, analiza
  \item morfologiczna)
  \item Generowanie pomysłów: burza mózgów,
  \item MindMapping, deBono
  \item Zespołowe generowanie pomysłów:
  \item burza mózgów i metoda 6 kapeluszy
  \item Wybrane techniki oceny wariantów i wspomagania
  \item decyzji
  \item Metody oceny wariantów; testowanie wygenerowanych rozwiązań
  \item Organizacja kreatywnej pracy zespołowej
  \item Ocena potencjału motywacyjnego
  \item personelu; analiza i dopracowywanie własnego CV, SWOT osobisty
  \item Zarządzanie wiedzą i innowacjami w organizacji
  \item Zasady efektywnej komunikacji;
  \item Prezentacja rozwiązań projektowych
\end{enumerate}

\section{Efekty kształcenia}

\subsection*{Wiedza}
\begin{itemize}
  \item Student zna i rozumie pojęcia z zakresu planowania przedsięwzięcia informatycznego, wstępnej oceny ekonomicznej, aspektów społecznych oraz analizy wykonalności
  \item Student zna i rozumie podstawowe problemy etyczne, społeczne i zawodowe informatyki, rozumie odpowiedzialność związaną z działalnością w obszarze informatyki; zna i rozumie podstawowe pojęcia z zakresu ochrony własności intelektualnej oraz prawa patentowego i autorskiego; zna i rozumie pozatechniczne aspekty informatyki, powiązanie przedsięwzięć informatycznych z ich otoczeniem i zagrożenia stąd płynące
  \item Student zna i rozumie podstawowe pojęcia dotyczące prowadzenia działalności gospodarczej, szczególnie przedsięwzięć informatycznych i rozumie rolę jej innowacyjności. Zna i rozumie ogólne zasady tworzenia i rozwoju form indywidualnej przedsiębiorczości, szczególnie w zakresie zastosowań rozwiązań informatycznych.
\end{itemize}

\subsection*{Umiejętności}
\begin{itemize}
  \item Student potrafi pracować w zespole; potrafi oszacować czas i koszty potrzebne na realizację zleconego zadania; potrafi zaplanować, opracować i zrealizować harmonogram prac zapewniający dotrzymanie terminów
\end{itemize}

\subsection*{Kompetencje społeczne}
\begin{itemize}
  \item Student jest gotów do zastosowań informatyki na rzecz rozwoju nauki i społeczeństwa informacyjnego
  \item Student jest gotów do współdziałań i współpracy w zespole, przyjmując różne role, m.in. zamawiającego, klienta, analityka, projektanta, wykonawcy
  \item Student jest gotów do określenia priorytetów służących realizacji zadania
\end{itemize}

\section{Kryteria oceny}

\begin{itemize}
  \item analiza tekstów z dyskusją
  \item rozwiązywanie zadań
  \item burza mózgów
  \item praca grupowa nad projektem z wykorzystaniem nowoczesnych technik informatycznych
  \item warsztaty
  \item Kryteria oceny
  \item ćwiczenia związane z tematyką wykładu
  \item opracowania obliczeniowe i koncepcyjne
  \item ocena sporządzonego sprawozdania
  \item brak
\end{itemize}

\section{Metody dydaktyczne}

Wykład, laboratoria, praca własna studenta.

\section{Literatura}

\textbf{Podstawowa:}
\begin{itemize}
  \item Kosieradzka, A., Szopiński, T., Stanisławiak, E., Głażewska, I., Kąkol, U., Smagowicz, J. (2021). Metody i techniki pobudzania kreatywności w organizacji i zarządzaniu. Warszawa: PWN. ISBN 978-83-01-21845-0.
  \item Kosieradzka, A. (2019). Zarządzanie produktywnością w przedsiębiorstwie. Kraków: edu-Libri. ISBN 978-83-63804-04-6.
\end{itemize}

\textbf{Uzupełniająca:}
\begin{itemize}
  \item Szmidt, K. J. (2021). Teoria i praktyka twórczości: Od badań do edukacji kreatywnej. Łódź: Wydawnictwo Uniwersytetu Łódzkiego. ISBN: 978-83-8142-929-0
  \item Penc-Pietrzak, I. (2019). Kreatywność w zarządzaniu. Nowe wyzwania dla liderów organizacji. Warszawa: Difin. ISBN: 978-83-8085-882-2
  \item Trzebiński, J., \& Siemieńska, R. (2020). Twórcze rozwiązywanie problemów: Podstawy teoretyczne i zastosowania praktyczne. Warszawa: PWN. ISBN: 978-83-01-21489-7
  \item Nęcka, E., Orzechowski, J., \& Słabosz, A. (2019). Kreatywność: Procesy poznawcze, osobowość, środowisko. Warszawa: PWN. ISBN: 978-83-01-20526-0
\end{itemize}

\end{document}
