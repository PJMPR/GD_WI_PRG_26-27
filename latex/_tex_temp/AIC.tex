% ===========================================================
%  Sylabus: Analiza Incydentów Cyberbezpieczeństwa (AIC)
% ===========================================================
\documentclass[12pt, a4paper]{article}

\usepackage[T1]{fontenc}
\usepackage[utf8]{inputenc}
\usepackage[polish]{babel}
\usepackage{lmodern}
\usepackage{microtype}
\usepackage[a4paper, top=2.5cm, bottom=2.5cm, left=2.5cm, right=2.5cm]{geometry}
\usepackage{xcolor}
\usepackage{graphicx}
\usepackage{booktabs}
\usepackage{tabularx}
\usepackage{longtable}
\usepackage{multirow}
\usepackage{array}
\usepackage{colortbl}
\usepackage{enumitem}
\usepackage{fancyhdr}
\usepackage{titlesec}
\usepackage{mdframed}
\usepackage[colorlinks=true, linkcolor=red!70!black, urlcolor=red!70!black]{hyperref}
\usepackage{eso-pic}
\usepackage{tikz}

\definecolor{pjatkRed}{RGB}{180,0,0}
\definecolor{pjatkGray}{RGB}{80,80,80}
\definecolor{pjatkLightGray}{RGB}{245,245,245}
\definecolor{tableHeader}{RGB}{220,220,220}

\pagestyle{fancy}\fancyhf{}
\renewcommand{\headrulewidth}{0.4pt}
\renewcommand{\footrulewidth}{0.4pt}
\fancyhead[L]{\small\textcolor{pjatkGray}{PJATK -- Filia w Gdańsku \textbar\ Informatyka}}
\fancyhead[R]{\small\textcolor{pjatkGray}{Sylabus: AIC}}
\fancyfoot[C]{\small\thepage}

\titleformat{\section}{\large\bfseries\color{pjatkRed}}{\thesection.}{0.5em}{}
  [\color{pjatkRed}\rule{\linewidth}{0.8pt}]
\setlist{noitemsep, topsep=3pt, parsep=2pt}

\newmdenv[linecolor=pjatkRed, linewidth=1.2pt, backgroundcolor=pjatkLightGray,
  innerleftmargin=10pt, innerrightmargin=10pt, innertopmargin=8pt,
  innerbottommargin=8pt, roundcorner=4pt]{infobox}

\begin{document}

\AddToShipoutPictureBG{%
  \begin{tikzpicture}[remember picture, overlay]
    \node[opacity=0.5] at (current page.center) {%
      \includegraphics[width=14cm]{C:/Users/adamu/WebstormProjects/pj-studies/latex/PJATK_pl_sygnet_transparent-eps-converted-to}%
    };
  \end{tikzpicture}%
}

\begin{center}
  \includegraphics[height=2cm]{C:/Users/adamu/WebstormProjects/pj-studies/latex/PJATK_pl_poziom_1}\\[0.8cm]
  {\LARGE\bfseries\color{pjatkRed} SYLABUS PRZEDMIOTU}\\[0.8cm]
\end{center}

\begin{infobox}
\begin{tabularx}{\textwidth}{@{}lX@{}}
  \textbf{Nazwa przedmiotu:}  & {\bfseries Analiza Incydentów Cyberbezpieczeństwa} \\[3pt]
  \textbf{Kod przedmiotu:}    & AIC \\[3pt]
  \textbf{Kierunek / Profil:} & Informatyka / praktyczny \\[3pt]
  \textbf{Tryb studiów:}      & niestacjonarny \\[3pt]
  \textbf{Rok / Semestr:}     & 3 / 6 \\[3pt]
  \textbf{Charakter:}         & obowiązkowy \\[3pt]
  \textbf{Odpowiedzialny:}    & mgr inż. Paweł Lelental \\[3pt]
  \textbf{Wersja z dnia:}     & 19.02.2026 \\
\end{tabularx}
\end{infobox}

\vspace{1cm}

\section{Godziny zajęć i punkty ECTS}

\begin{center}
\begin{tabular}{|>{\centering\arraybackslash}p{2.0cm}
                |>{\centering\arraybackslash}p{2.0cm}
                |>{\centering\arraybackslash}p{2.0cm}
                |>{\centering\arraybackslash}p{2.4cm}
                |>{\centering\arraybackslash}p{2.4cm}
                |>{\centering\arraybackslash}p{2.0cm}
                |>{\centering\arraybackslash}p{1.4cm}|}
\hline
\rowcolor{tableHeader}
\textbf{Wykłady} & \textbf{Ćwiczenia} & \textbf{Laboratorium} &
\textbf{Z prowadzącym} & \textbf{Praca własna} & \textbf{Łącznie} & \textbf{ECTS} \\
\hline
30 h & --- & 30 h & 60 h & 40 h & 100 h & \textbf{5} \\
\hline
\end{tabular}
\end{center}

\section{Forma zajęć}

\begin{tabular}{ll}
  \hline
  \textbf{Forma zajęć} & \textbf{Sposób zaliczenia} \\
  \hline
  Laboratorium & Zaliczenie z oceną \\
  Wykład & Nieoceniany \\
  \hline
\end{tabular}

\section{Cel dydaktyczny}

Analiza incydentów cyberbezpieczeństwa to proces badania i oceny zdarzeń, które mogą stanowić naruszenie cyberbezpieczeństwa. Celem jest zrozumienie, jak doszło do incydentu, jakie są jego skutki, jak można go zażegnać i jak zapobiegać podobnym incydentom w przyszłości. W całym tym procesie kluczowe jest posiadanie odpowiednich umiejętności analizy systemów i sieci, a także do podejmowania szybkich i skutecznych działań w celu zminimalizowania wpływu incydentów na cyberbezpieczeństwo.

\section{Treści programowe}

\begin{enumerate}
  \item Wykład:
  \item Anatomia zagrożenia – Ransomware (wg. ENISA)
  \item Anatomia zagrożenia – Malware (wg. ENISA)
  \item Anatomia zagrożenia – Cryptojacking (wg. ENISA)
  \item Anatomia zagrożenia - E-mail related threats (wg. ENISA)
  \item Anatomia zagrożenia - Threats against data (wg. ENISA)
  \item Anatomia zagrożenia - Threats against availability and integrity (wg. ENISA)
  \item Anatomia zagrożenia - Disinformation – misinformation (wg. ENISA)
  \item Anatomia zagrożenia - Non-malicious threats (wg. ENISA)
  \item Laboratoria:
  \item Zapoznanie się z organizacją oraz utworzenie Zespołów SOC
  \item Atak 1: Atak typu Ransomware, Poziom 1– Reakcja i prewencja
  \item Atak 2: Atak typu Malware. Poziom 1 – Reakcja i prewencja
  \item Atak 3: Atak typu Cryptojacking, Poziom 1 – Reakcja i prewencja
  \item Atak 4: Atak typu E-mail related threats, Poziom 1 – Reakcja i prewencja
  \item Atak 5: Atak typu Threats against data, Poziom 1 – Reakcja i prewencja
  \item Atak 6: Atak typu Threats against availability and integrity, Poziom 1 – Reakcja i prewencja
  \item Atak 7: Atak typu Disinformation – misinformation, Poziom 1 – Reakcja i prewencja
  \item Atak 8: Atak typu Non-malicious threats, Poziom 1 – Reakcja i prewencja
  \item Atak 9: Atak typu Zaawansowany wektor ataku 1, Poziom 2  – Reakcja i prewencja
  \item Atak 10: Atak typu Zaawansowany wektor ataku 2, Poziom 2  – Reakcja i prewencja
  \item Atak 11: Atak typu Zaawansowany wektor ataku 3, Poziom 2  – Reakcja i prewencja
  \item Atak 12 : Atak typu Zaawansowany wektor ataku 4, Poziom 2  – Reakcja i prewencja
\end{enumerate}

\section{Efekty kształcenia}

\subsection*{Wiedza}
\begin{itemize}
  \item Student zna i rozumie anatomię kluczowych zagrożeń Cyber
  \item Student zna i rozumie podstawowe środki ochrony i reakcji na kluczowe zagrożenia Cyber
\end{itemize}

\subsection*{Umiejętności}
\begin{itemize}
  \item Student potrafi przygotować zestaw środków ochrony informacji odpowiednich dla różnych typów zagrożeń
  \item Student potrafi wskazać mocne i słabe strony systemów ochrony informacji
\end{itemize}

\subsection*{Kompetencje społeczne}
\begin{itemize}
  \item Student jest gotów do pracy zespołowej w dziale SOC
\end{itemize}

\section{Kryteria oceny}

\begin{itemize}
  \item burza mózgów
  \item warsztaty
  \item Kryteria oceny
  \item Laboratorium/Projekt
  \item rezultaty gry strategicznej, kolokwium końcowe
  \item Kryteria oceny:niedostateczny: < 50\% możliwych do zdobycia punktów dostateczny  :     50\%-60\% możliwych do zdobycia punktów dostateczny+ :    61\%- 70\% możliwych do zdobycia punktówdobry :                71\%-80\% możliwych do zdobycia punktówdobry +:              81\%-90\% możliwych do zdobycia punktów bardzo dobry  :   >90\%  możliwych do zdobycia punktów
  \item Zaliczenie – kolokwium końcowe z Laboratoriów
\end{itemize}

\section{Metody dydaktyczne}

Wykład, laboratoria, praca własna studenta.

\section{Literatura}

\textbf{Podstawowa:}
\begin{itemize}
  \item William Stallings, Lawrie Brown - Bezpieczeństwo systemów informatycznych. Zasady i praktyka. (ang. Computer Security: Principles and Practice), Wydanie IV. Tom 1 i 2, Helion 2019
\end{itemize}

\textbf{Uzupełniająca:}
\begin{itemize}
  \item Brak danych.
\end{itemize}

\end{document}
