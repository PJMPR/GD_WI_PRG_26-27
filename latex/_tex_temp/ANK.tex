% ===========================================================
%  Sylabus: Animacje komputerowe (ANK)
% ===========================================================
\documentclass[12pt, a4paper]{article}

\usepackage[T1]{fontenc}
\usepackage[utf8]{inputenc}
\usepackage[polish]{babel}
\usepackage{lmodern}
\usepackage{microtype}
\usepackage[a4paper, top=2.5cm, bottom=2.5cm, left=2.5cm, right=2.5cm]{geometry}
\usepackage{xcolor}
\usepackage{graphicx}
\usepackage{booktabs}
\usepackage{tabularx}
\usepackage{longtable}
\usepackage{multirow}
\usepackage{array}
\usepackage{colortbl}
\usepackage{enumitem}
\usepackage{fancyhdr}
\usepackage{titlesec}
\usepackage{mdframed}
\usepackage[colorlinks=true, linkcolor=red!70!black, urlcolor=red!70!black]{hyperref}
\usepackage{eso-pic}
\usepackage{tikz}

\definecolor{pjatkRed}{RGB}{180,0,0}
\definecolor{pjatkGray}{RGB}{80,80,80}
\definecolor{pjatkLightGray}{RGB}{245,245,245}
\definecolor{tableHeader}{RGB}{220,220,220}

\pagestyle{fancy}\fancyhf{}
\renewcommand{\headrulewidth}{0.4pt}
\renewcommand{\footrulewidth}{0.4pt}
\fancyhead[L]{\small\textcolor{pjatkGray}{PJATK -- Filia w Gdańsku \textbar\ Informatyka}}
\fancyhead[R]{\small\textcolor{pjatkGray}{Sylabus: ANK}}
\fancyfoot[C]{\small\thepage}

\titleformat{\section}{\large\bfseries\color{pjatkRed}}{\thesection.}{0.5em}{}
  [\color{pjatkRed}\rule{\linewidth}{0.8pt}]
\setlist{noitemsep, topsep=3pt, parsep=2pt}

\newmdenv[linecolor=pjatkRed, linewidth=1.2pt, backgroundcolor=pjatkLightGray,
  innerleftmargin=10pt, innerrightmargin=10pt, innertopmargin=8pt,
  innerbottommargin=8pt, roundcorner=4pt]{infobox}

\begin{document}

\AddToShipoutPictureBG{%
  \begin{tikzpicture}[remember picture, overlay]
    \node[opacity=0.5] at (current page.center) {%
      \includegraphics[width=14cm]{C:/Users/adamu/WebstormProjects/pj-studies/latex/PJATK_pl_sygnet_transparent-eps-converted-to}%
    };
  \end{tikzpicture}%
}

\begin{center}
  \includegraphics[height=2cm]{C:/Users/adamu/WebstormProjects/pj-studies/latex/PJATK_pl_poziom_1}\\[0.8cm]
  {\LARGE\bfseries\color{pjatkRed} SYLABUS PRZEDMIOTU}\\[0.8cm]
\end{center}

\begin{infobox}
\begin{tabularx}{\textwidth}{@{}lX@{}}
  \textbf{Nazwa przedmiotu:}  & {\bfseries Animacje komputerowe} \\[3pt]
  \textbf{Kod przedmiotu:}    & ANK \\[3pt]
  \textbf{Kierunek / Profil:} & Informatyka / praktyczny \\[3pt]
  \textbf{Tryb studiów:}      & niestacjonarny \\[3pt]
  \textbf{Rok / Semestr:}     & 3 / 6 \\[3pt]
  \textbf{Charakter:}         & obowiązkowy \\[3pt]
  \textbf{Odpowiedzialny:}    & Dr Piotr Arłukowicz \\[3pt]
  \textbf{Wersja z dnia:}     & 19.02.2026 \\
\end{tabularx}
\end{infobox}

\vspace{1cm}

\section{Godziny zajęć i punkty ECTS}

\begin{center}
\begin{tabular}{|>{\centering\arraybackslash}p{2.0cm}
                |>{\centering\arraybackslash}p{2.0cm}
                |>{\centering\arraybackslash}p{2.0cm}
                |>{\centering\arraybackslash}p{2.4cm}
                |>{\centering\arraybackslash}p{2.4cm}
                |>{\centering\arraybackslash}p{2.0cm}
                |>{\centering\arraybackslash}p{1.4cm}|}
\hline
\rowcolor{tableHeader}
\textbf{Wykłady} & \textbf{Ćwiczenia} & \textbf{Laboratorium} &
\textbf{Z prowadzącym} & \textbf{Praca własna} & \textbf{Łącznie} & \textbf{ECTS} \\
\hline
30 h & 30 h & --- & 60 h & 65 h & 125 h & \textbf{5} \\
\hline
\end{tabular}
\end{center}

\section{Forma zajęć}

\begin{tabular}{ll}
  \hline
  \textbf{Forma zajęć} & \textbf{Sposób zaliczenia} \\
  \hline
  Laboratorium & Zaliczenie z oceną \\
  Wykład & Projekt końcowy na prawach egzaminu \\
  \hline
\end{tabular}

\section{Cel dydaktyczny}

Celem przedmiotu jest przygotowanie studentów do bardziej wszechstronnej pracy w branży Gamedev w charakterze twórców dynamicznych treści cyfrowych związanych z animacją w różnych jej odmianach. Nacisk kładziony jest na wykorzystanie mechanizmów parametrycznych i hierarchicznych, tworzenie rigów 3D oraz związane z nimi tematy takie jak 12 podstaw animacji Disneya, przechwytywanie ruchu, symulowanie ruchu, upraszczanie i edycja krzywych ruchu, tworzenie mechanik i mechanizmów z elementami obliczeń inżynierskich i skryptowania oraz zarządzanie danymi dynamicznymi, takimi jak mockupy, pozy, klucze kształtów i sterowniki/więzy ograniczające.

\section{Przedmioty wprowadzające}

\begin{tabularx}{\textwidth}{lX}
  \hline
  \textbf{Przedmiot} & \textbf{Wymagane zagadnienia} \\
  \hline
  Niezbędna jest ogólna znajomość obsługi komputera. & M3D lub G3D \\
  Bardzo zalecane jest wcześniejsze skończenie kursu Grafiki komputerowej lub Modelowania 3D dla gier. & --- \\
  \hline
\end{tabularx}

\section{Treści programowe}

\begin{enumerate}
  \item 1. Wprowadzenie do podstaw animacji metodą klatek kluczowych, animacja w wielu wymiarach.
  \item 2. Animacja typu straight-ahead piłki spadającej ze schodów ze zderzeniami doskonale sprężystymi.
  \item 3. Modyfikatory animacji, dopesheet, edytor krzywych ruchu (graph editor).
  \item 4. Modelowanie animacji: prędkość, dynamika, tworzenie wrażenia energii ruchu i charakteru obiektu.
  \item 5. Dwanaście zasad animacji wg Disneya – podstawowe principia.
  \item 6. Animacje zależne: układ słoneczny, zastosowanie więzów (constraints).
  \item 7. Animacja uniwersalna za pomocą driverów – tworzenie skomplikowanego układu kół zębatych.
  \item 8. Wstęp do armatur – podstawy tworzenia rigów.
  \item 9. Rig piłki i realizacja zasady squash/stretch.
  \item 10. Rig falującego ogona, pojęcie secondary animation, przekazanie energii, punkt działania.
  \item 11. Inverse kinematics, rig ręki, kombinacja FK+IK.
  \item 12. Rig stopy (opcjonalnie) lub biped rig, wstęp do walkcycle.
  \item 13. Walkcycle i stosowanie referencji animacji
  \item 14. Edytor NLA, akcje, biblioteka póz, animacje z motion-capture.
  \item 15. Tworzenie zaawansowanej wieloskładnikowej animacji pełnego aktora na przykładzie profesjonalnych rigów przemysłowych (np. blenrig5 lub innych).
\end{enumerate}

\section{Efekty kształcenia}

\subsection*{Wiedza}
\begin{itemize}
  \item Student zna i rozumie pojęcia takie jak animacja, klatki kluczowe, więzy, easing, drivery, antycypacja, squash/stretch, rodzaje symulacji, kluczowanie, parametryczne zmiany kształtu, mocupy, motion-capture, rig, armature, dopesheet, edytor akcji, NLA, i inne
\end{itemize}

\subsection*{Umiejętności}
\begin{itemize}
  \item Student potrafi wymodelować ruchy obiektów po zadanej, oczekiwanej trajektorii w przestrzeni 3D
  \item Student potrafi utworzyć animację w oparciu o rig z kości lub innych obiektów powiązanych hierarchicznie
  \item Student potrafi tworzyć animacje zależne oraz parametryczne, włącznie z obliczaniem parametrów ruchu (np. prędkości kół zębatych w przekładni planetarnej, itp.)
  \item Student potrafi zastosować mockupy z zewnętrznych baz motion-capture i stworzyć z nich animację złożoną z różnych nakładających się na siebie akcji
\end{itemize}

\subsection*{Kompetencje społeczne}
\begin{itemize}
  \item Student jest gotów do współpracy i dzielenia się swoją wiedzą
  \item Student jest gotów do dalszej nauki i pogłębiania swojej wiedzy i umiejętności
\end{itemize}

\section{Kryteria oceny}

\begin{itemize}
  \item prezentacja na żywo
  \item wykład z elementami dyskusji
  \item studium przypadku
  \item najlepsze praktyki
  \item praca samodzielna ucznia nad zadaniami i mini-projektami
  \item projekt semestralny
  \item Projekt końcowy na prawach egzaminu
  \item Kryteria oceny
  \item Ćwiczenia/Projekt
  \item Student musi wykonać zgodnie z regułami sztuki kluczowe ćwiczenia prezentowane w trakcie semestru. Sprawdzana jest zgodność z zasadami ale dopuszczane są inwencja twórcza oraz kreatywność (dziedzina jest twórcza). Tworzone jest w ciągu semestru portfolio rozwojowe pokazujące eksperymenty własne i ścieżkę osiągnięć, które są następnie na końcu zajęć oceniane.
  \item Student zobowiązany jest przedstawić większy projekt własny w którym wykorzystuje poznane techniki modelowania. Projekty są negocjowane indywidualnie już podczas semestru i można je przygotowywać w trakcie jego trwania.
\end{itemize}

\section{Metody dydaktyczne}

Wykład, laboratoria, praca własna studenta.

\section{Literatura}

\textbf{Podstawowa:}
\begin{itemize}
  \item 2. Realizing 3D Animation in Blender: Master the fundamentals of 3D animation in Blender, from keyframing to character movement, ISBN-13: 978-1801077217
  \item 3. A Complete Guide to Character Rigging for Games Using Blender, ISBN-13: 978-1032203003
\end{itemize}

\textbf{Uzupełniająca:}
\begin{itemize}
  \item 1. Mind-Melding Unity and Blender for 3D Game Development: Unleash the power of Unity and Blender to create amazing games, ISBN-13: 978-1801071550
  \item 2. Beginner’s Guide to Creating Characters in Blender, ISBN-13: 978-1912843138
  \item 3. Learn Blender Simulations the Right Way: Create attractive and realistic animations with Mantaflow, rigid and soft bodies, and Dynamic Paint, ISBN-13: 978-1803234151
\end{itemize}

\end{document}
