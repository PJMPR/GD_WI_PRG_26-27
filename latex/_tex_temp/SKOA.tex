% ===========================================================
%  Sylabus: Sieci komputerowe (SKO1)
% ===========================================================
\documentclass[12pt, a4paper]{article}

\usepackage[T1]{fontenc}
\usepackage[utf8]{inputenc}
\usepackage[polish]{babel}
\usepackage{lmodern}
\usepackage{microtype}
\usepackage[a4paper, top=2.5cm, bottom=2.5cm, left=2.5cm, right=2.5cm]{geometry}
\usepackage{xcolor}
\usepackage{graphicx}
\usepackage{booktabs}
\usepackage{tabularx}
\usepackage{longtable}
\usepackage{multirow}
\usepackage{array}
\usepackage{colortbl}
\usepackage{enumitem}
\usepackage{fancyhdr}
\usepackage{titlesec}
\usepackage{mdframed}
\usepackage[colorlinks=true, linkcolor=red!70!black, urlcolor=red!70!black]{hyperref}
\usepackage{eso-pic}
\usepackage{tikz}

\definecolor{pjatkRed}{RGB}{180,0,0}
\definecolor{pjatkGray}{RGB}{80,80,80}
\definecolor{pjatkLightGray}{RGB}{245,245,245}
\definecolor{tableHeader}{RGB}{220,220,220}

\pagestyle{fancy}\fancyhf{}
\renewcommand{\headrulewidth}{0.4pt}
\renewcommand{\footrulewidth}{0.4pt}
\fancyhead[L]{\small\textcolor{pjatkGray}{PJATK -- Filia w Gdańsku \textbar\ Informatyka}}
\fancyhead[R]{\small\textcolor{pjatkGray}{Sylabus: SKO1}}
\fancyfoot[C]{\small\thepage}

\titleformat{\section}{\large\bfseries\color{pjatkRed}}{\thesection.}{0.5em}{}
  [\color{pjatkRed}\rule{\linewidth}{0.8pt}]
\setlist{noitemsep, topsep=3pt, parsep=2pt}

\newmdenv[linecolor=pjatkRed, linewidth=1.2pt, backgroundcolor=pjatkLightGray,
  innerleftmargin=10pt, innerrightmargin=10pt, innertopmargin=8pt,
  innerbottommargin=8pt, roundcorner=4pt]{infobox}

\begin{document}

\AddToShipoutPictureBG{%
  \begin{tikzpicture}[remember picture, overlay]
    \node[opacity=0.5] at (current page.center) {%
      \includegraphics[width=14cm]{C:/Users/adamu/WebstormProjects/pj-studies/latex/PJATK_pl_sygnet_transparent-eps-converted-to}%
    };
  \end{tikzpicture}%
}

\begin{center}
  \includegraphics[height=2cm]{C:/Users/adamu/WebstormProjects/pj-studies/latex/PJATK_pl_poziom_1}\\[0.8cm]
  {\LARGE\bfseries\color{pjatkRed} SYLABUS PRZEDMIOTU}\\[0.8cm]
\end{center}

\begin{infobox}
\begin{tabularx}{\textwidth}{@{}lX@{}}
  \textbf{Nazwa przedmiotu:}  & {\bfseries Sieci komputerowe} \\[3pt]
  \textbf{Kod przedmiotu:}    & SKO1 \\[3pt]
  \textbf{Kierunek / Profil:} & Informatyka / praktyczny \\[3pt]
  \textbf{Tryb studiów:}      & niestacjonarny \\[3pt]
  \textbf{Rok / Semestr:}     & 2 / 4 \\[3pt]
  \textbf{Charakter:}         & obowiązkowy \\[3pt]
  \textbf{Odpowiedzialny:}    & Mgr inż. Agata Puchalska \\[3pt]
  \textbf{Wersja z dnia:}     & 19.02.2026 \\
\end{tabularx}
\end{infobox}

\vspace{1cm}

\section{Godziny zajęć i punkty ECTS}

\begin{center}
\begin{tabular}{|>{\centering\arraybackslash}p{2.0cm}
                |>{\centering\arraybackslash}p{2.0cm}
                |>{\centering\arraybackslash}p{2.0cm}
                |>{\centering\arraybackslash}p{2.4cm}
                |>{\centering\arraybackslash}p{2.4cm}
                |>{\centering\arraybackslash}p{2.0cm}
                |>{\centering\arraybackslash}p{1.4cm}|}
\hline
\rowcolor{tableHeader}
\textbf{Wykłady} & \textbf{Ćwiczenia} & \textbf{Laboratorium} &
\textbf{Z prowadzącym} & \textbf{Praca własna} & \textbf{Łącznie} & \textbf{ECTS} \\
\hline
16 h & --- & 24 h & 40 h & 85 h & 125 h & \textbf{5} \\
\hline
\end{tabular}
\end{center}

\section{Forma zajęć}

\begin{tabular}{ll}
  \hline
  \textbf{Forma zajęć} & \textbf{Sposób zaliczenia} \\
  \hline
  Wykład & Egzamin \\
  \hline
\end{tabular}

\section{Cel dydaktyczny}

Przedmiot poświęcony jest prezentacji zasad funkcjonowania  współczesnych sieci teleinformatycznych w oparciu o model warstwowy OSI. Omawiane są funkcje poszczególnych  warstw OSI i powiązań między nimi, używane technologie, standardy, protokoły. Studenci nabywają umiejętności niezbędne do opracowania  projektu sieci LAN  łącznie z konfiguracją IPv4 i IPv6, hermetyzacją  routerów i przełączników ,  routingiem i podstawowym zabezpieczeniem urządzeń przed atakami z  sieci Inside  i Outside .

\section{Przedmioty wprowadzające}

\begin{tabularx}{\textwidth}{lX}
  \hline
  \textbf{Przedmiot} & \textbf{Wymagane zagadnienia} \\
  \hline
  Użytkowanie Komputerów i Podstawy Systemów Operacyjnych & Wstęp do informatyki i architektury komputerów \\
  Wymagane zagadnienia/umiejętności niezbędne w realizacji danego przedmiotu & Znajomość regulaminu i zasad BHP pracowni informatycznej i sieciowej \\
  Rozumienie pojęć z zakresu kluczowych zagadnień dotyczących systemów operacyjnych – zasady ich działania, konstrukcji, organizacji współbieżności & Rozumienie  pojęć z zakresu  sieci komputerowych,  technologii, protokołów komunikacyjnych i zagadnień bezpieczeństwa, telekomunikacji \\
  Podstawowa wiedza o organizacji systemu komputerowego i architekturze warstwowej & --- \\
  \hline
\end{tabularx}

\section{Treści programowe}

\begin{enumerate}
  \item Komunikacja w sieci komputerowej , elementy sieci, podziały sieci , podstawowe zjawiska w sieciach komputerowych, modele sieci komputerowych . Cechy współczesnych sieci komputerowych , najnowsze trendy w sieciach komp.
  \item Media sieciowe, rodzaje mediów,  parametry, zastosowanie
  \item Urządzenia sieciowe – ewolucja , charakterystyka, zastosowanie . Metody dostępu do urządzeń sieciowych, hermetyzacja, tryby pracy urządzeń
  \item Technologia Ethernet, ramka Ethernet,  topologie logiczne i fizyczne w sieci LAN, WAN, algorytm CSMA/CD, CSMA/CA , rodzaje transmisji w sieci ( unicast, multicast, broadcast ) ,  protokół MAC, ARP
  \item Anatomia adresu IP v4 , rodzaje  IPv4, rodzaje adresów IPv4   planowanie adresów  sieci wewnątrz organizacji metodą VLSM oraz ze stałą maską.
  \item Anatomia adresu IP v6 , rodzaje  IPv6,  rodzaje adresów IPv6, dynamiczne i statyczne przydzielanie adresów IPv6. Porównanie IPv4 via IPv6, protokół ICMP dla IPv4 via ICMP dla IPv6
  \item Routing statyczny a dynamiczny, tablica routingu hosta, routera, brama domyślna, trasa ostatniej szansy
  \item Obsługa niezawodnej komunikacji TCP, trójstopniowe uzgadnianie, kontrola przepływu, dynamiczne wymiary okien. Protokół TCP a UDP.
  \item Usługi i protokoły warstwy aplikacji ( DNS, HTTP, HTTPS, Telnet, SSH , FTP, DHCP, SMTP, POP3, IMAP, S ) – charakterystyka, wykorzystanie , konfiguracja
  \item Metody zarządzania siecią LAN
\end{enumerate}

\section{Efekty kształcenia}

\subsection*{Wiedza}
\begin{itemize}
  \item Student zna i rozumie pojęcia  z zakresu elektrotechniki, elektroniki i miernictwa; powiązania informatyki z tymi obszarami
  \item Student zna i rozumie zagadnienia z zakresu techniki cyfrowej i systemów cyfrowych, architektury i organizacji systemów komputerowych
  \item Student zna i rozumie zagadnienia z zakresu sieci komputerowych, ich technologii, protokołów komunikacyjnych i zagadnień bezpieczeństwa, telekomunikacji oraz potrzebę przenoszenia dobrych praktyk na grunt informatyki
\end{itemize}

\subsection*{Umiejętności}
\begin{itemize}
  \item Student potrafi   zaprojektować, zainstalować i administrować siecią LAN z interfejsami WAN, która umożliwia także realizację kluczowych usług sieciowych z zachowaniem zasad bezpieczeństwa informacji
\end{itemize}

\subsection*{Kompetencje społeczne}
\begin{itemize}
  \item Student jest gotów do uczenia się przez całe życie; potrafi inspirować i organizować proces uczenia się innych osób
  \item Student jest gotów do określenia priorytetów służących realizacji zadania
  \item Student jest gotów do myślenia i działania w sposób innowacyjny i przedsiębiorczy
  \item Student jest gotów do komunikacji w skuteczny sposób z inwestorami z różnych środowisk, pozyskując od nich wiedzę tworzącą wartość dodaną przedsięwzięć informatycznych
\end{itemize}

\section{Kryteria oceny}

\begin{itemize}
  \item Ćwiczenia / Laboratorium:
  \item rozwiązywanie zadań
  \item ćwiczenia projektowe realizowane praktycznie na urządzeniach Cisco na wyposażeniu laboratorium
  \item projektowanie sieci w wykorzystaniem   symulatora sieci Cisco Packet Tracer
  \item Ćwiczenia/Laboratorium
  \item Kryteria oceny
  \item Ćwiczenia/Laboratorium/Projekt/Lektorat
  \item Ocenę stanowi 40\% kolokwium z zakresu przyznawania  adresów IP v4 i v 6  , 40\%  ćwiczenia praktyczne na urządzeniach Cisco  lub na symulatorze sieci Cisco Packet Tracer  wykonywane w trakcie zajęć, 20\% aktywność
  \item Ocenę stanowi 45\% wyniku uzyskanego na egzaminie,45\% ocen otrzymywanych systematycznie w trakcie trwania semestru z tytułu realizacji modułów tematycznych(egzaminycząstkowe), 10\% aktywność
\end{itemize}

\section{Metody dydaktyczne}

Wykład, laboratoria, praca własna studenta.

\section{Literatura}

\textbf{Podstawowa:}
\begin{itemize}
  \item Materiały szkoleniowe „CCNAv7: Introduction to Networks”umieszczone na platformie netacad.com
\end{itemize}

\textbf{Uzupełniająca:}
\begin{itemize}
  \item Szkolenia  online „Securac"
  \item „Wprowadzenie do bezpieczeństwa IT „ Michał Sajdak
\end{itemize}

\end{document}
