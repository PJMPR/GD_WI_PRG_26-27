% ===========================================================
%  Sylabus: Wprowadzenie do Flutter (MFL)
% ===========================================================
\documentclass[12pt, a4paper]{article}

\usepackage[T1]{fontenc}
\usepackage[utf8]{inputenc}
\usepackage[polish]{babel}
\usepackage{lmodern}
\usepackage{microtype}
\usepackage[a4paper, top=2.5cm, bottom=2.5cm, left=2.5cm, right=2.5cm]{geometry}
\usepackage{xcolor}
\usepackage{graphicx}
\usepackage{booktabs}
\usepackage{tabularx}
\usepackage{longtable}
\usepackage{multirow}
\usepackage{array}
\usepackage{colortbl}
\usepackage{enumitem}
\usepackage{fancyhdr}
\usepackage{titlesec}
\usepackage{mdframed}
\usepackage[colorlinks=true, linkcolor=red!70!black, urlcolor=red!70!black]{hyperref}
\usepackage{eso-pic}
\usepackage{tikz}

\definecolor{pjatkRed}{RGB}{180,0,0}
\definecolor{pjatkGray}{RGB}{80,80,80}
\definecolor{pjatkLightGray}{RGB}{245,245,245}
\definecolor{tableHeader}{RGB}{220,220,220}

\pagestyle{fancy}\fancyhf{}
\renewcommand{\headrulewidth}{0.4pt}
\renewcommand{\footrulewidth}{0.4pt}
\fancyhead[L]{\small\textcolor{pjatkGray}{PJATK -- Filia w Gdańsku \textbar\ Informatyka}}
\fancyhead[R]{\small\textcolor{pjatkGray}{Sylabus: MFL}}
\fancyfoot[C]{\small\thepage}

\titleformat{\section}{\large\bfseries\color{pjatkRed}}{\thesection.}{0.5em}{}
  [\color{pjatkRed}\rule{\linewidth}{0.8pt}]
\setlist{noitemsep, topsep=3pt, parsep=2pt}

\newmdenv[linecolor=pjatkRed, linewidth=1.2pt, backgroundcolor=pjatkLightGray,
  innerleftmargin=10pt, innerrightmargin=10pt, innertopmargin=8pt,
  innerbottommargin=8pt, roundcorner=4pt]{infobox}

\begin{document}

\AddToShipoutPictureBG{%
  \begin{tikzpicture}[remember picture, overlay]
    \node[opacity=0.5] at (current page.center) {%
      \includegraphics[width=14cm]{C:/Users/adamu/WebstormProjects/pj-studies/latex/PJATK_pl_sygnet_transparent-eps-converted-to}%
    };
  \end{tikzpicture}%
}

\begin{center}
  \includegraphics[height=2cm]{C:/Users/adamu/WebstormProjects/pj-studies/latex/PJATK_pl_poziom_1}\\[0.8cm]
  {\LARGE\bfseries\color{pjatkRed} SYLABUS PRZEDMIOTU}\\[0.8cm]
\end{center}

\begin{infobox}
\begin{tabularx}{\textwidth}{@{}lX@{}}
  \textbf{Nazwa przedmiotu:}  & {\bfseries Wprowadzenie do Flutter} \\[3pt]
  \textbf{Kod przedmiotu:}    & MFL \\[3pt]
  \textbf{Kierunek / Profil:} & Informatyka / praktyczny \\[3pt]
  \textbf{Tryb studiów:}      & stacjonarny \\[3pt]
  \textbf{Rok / Semestr:}     & 2 / 3 \\[3pt]
  \textbf{Charakter:}         & obieralny \\[3pt]
  \textbf{Odpowiedzialny:}    & do ustalenia \\[3pt]
  \textbf{Wersja z dnia:}     & 20.02.2026 \\
\end{tabularx}
\end{infobox}

\vspace{1cm}

\section{Godziny zajęć i punkty ECTS}

\begin{center}
\begin{tabular}{|>{\centering\arraybackslash}p{2.0cm}
                |>{\centering\arraybackslash}p{2.0cm}
                |>{\centering\arraybackslash}p{2.0cm}
                |>{\centering\arraybackslash}p{2.4cm}
                |>{\centering\arraybackslash}p{2.4cm}
                |>{\centering\arraybackslash}p{2.0cm}
                |>{\centering\arraybackslash}p{1.4cm}|}
\hline
\rowcolor{tableHeader}
\textbf{Wykłady} & \textbf{Ćwiczenia} & \textbf{Laboratorium} &
\textbf{Z prowadzącym} & \textbf{Praca własna} & \textbf{Łącznie} & \textbf{ECTS} \\
\hline
15 h & --- & 15 h & 30 h & 20 h & 50 h & \textbf{2} \\
\hline
\end{tabular}
\end{center}

\section{Forma zajęć}

\begin{tabular}{ll}
  \hline
  \textbf{Forma zajęć} & \textbf{Sposób zaliczenia} \\
  \hline
  Projekt & Zaliczenie z oceną \\
  \hline
\end{tabular}

\section{Cel dydaktyczny}

Celem przedmiotu jest zapoznanie studentów z podstawami tworzenia wieloplatformowych aplikacji mobilnych z wykorzystaniem frameworka Flutter i języka Dart. Studenci poznają architekturę aplikacji opartych na widżetach, zasady budowania interfejsu użytkownika, zarządzanie stanem oraz integrację aplikacji z zewnętrznymi usługami. Zajęcia mają charakter praktyczny i przygotowują do samodzielnego tworzenia prostych aplikacji działających na systemach Android i iOS.

\section{Treści programowe}

\begin{enumerate}
  \item Wprowadzenie do Flutter i architektury aplikacji
  \item Środowisko pracy i konfiguracja projektu Flutter
  \item Podstawy języka Dart
  \item Widżety i budowanie interfejsu użytkownika
  \item Layout i stylowanie aplikacji
  \item Zarządzanie stanem aplikacji
  \item Nawigacja i routing
  \item Komunikacja z API i obsługa danych zdalnych
  \item Debugowanie i testowanie aplikacji Flutter
  \item Projekt zaliczeniowy – implementacja i prezentacja
\end{enumerate}

\section{Efekty kształcenia}

\subsection*{Wiedza}
\begin{itemize}
  \item Student zna podstawowe pojęcia związane z tworzeniem aplikacji mobilnych w Flutter, w tym widżety, drzewo widżetów, cykl życia aplikacji oraz podstawy języka Dart.
\end{itemize}

\subsection*{Umiejętności}
\begin{itemize}
  \item Student potrafi zaprojektować i zaimplementować prostą aplikację mobilną w Flutter, budować interfejs użytkownika, zarządzać stanem aplikacji, obsługiwać nawigację oraz komunikować się z zewnętrznym API.
\end{itemize}

\subsection*{Kompetencje społeczne}
\begin{itemize}
  \item Student potrafi samodzielnie planować realizację projektu aplikacji mobilnej, korzystać z dokumentacji technicznej oraz dbać o jakość i czytelność kodu.
\end{itemize}

\section{Kryteria oceny}

\begin{itemize}
  \item Projekt aplikacji mobilnej realizowany indywidualnie lub w parach – 100\%
  \item Ocena projektu obejmuje: poprawność działania aplikacji, strukturę widżetów, zarządzanie stanem, nawigację, integrację z API oraz jakość kodu
  \item Warunkiem zaliczenia jest oddanie kompletnego projektu oraz jego pozytywna ocena
\end{itemize}

\section{Metody dydaktyczne}

Wykład, laboratoria, praca własna studenta.

\section{Literatura}

\textbf{Podstawowa:}
\begin{itemize}
  \item Dokumentacja Flutter: https://docs.flutter.dev
  \item Dokumentacja Dart: https://dart.dev/guides
  \item Flutter Cookbook – materiały online
\end{itemize}

\textbf{Uzupełniająca:}
\begin{itemize}
  \item Marco L. Napoli, Beginning Flutter
  \item Materiały społeczności Flutter
\end{itemize}

\end{document}
