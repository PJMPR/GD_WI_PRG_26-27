% ===========================================================
%  Sylabus: Programowanie w języku Python (JPT)
% ===========================================================
\documentclass[12pt, a4paper]{article}

\usepackage[T1]{fontenc}
\usepackage[utf8]{inputenc}
\usepackage[polish]{babel}
\usepackage{lmodern}
\usepackage{microtype}
\usepackage[a4paper, top=2.5cm, bottom=2.5cm, left=2.5cm, right=2.5cm]{geometry}
\usepackage{xcolor}
\usepackage{graphicx}
\usepackage{booktabs}
\usepackage{tabularx}
\usepackage{longtable}
\usepackage{multirow}
\usepackage{array}
\usepackage{colortbl}
\usepackage{enumitem}
\usepackage{fancyhdr}
\usepackage{titlesec}
\usepackage{mdframed}
\usepackage[colorlinks=true, linkcolor=red!70!black, urlcolor=red!70!black]{hyperref}
\usepackage{eso-pic}
\usepackage{tikz}

\definecolor{pjatkRed}{RGB}{180,0,0}
\definecolor{pjatkGray}{RGB}{80,80,80}
\definecolor{pjatkLightGray}{RGB}{245,245,245}
\definecolor{tableHeader}{RGB}{220,220,220}

\pagestyle{fancy}\fancyhf{}
\renewcommand{\headrulewidth}{0.4pt}
\renewcommand{\footrulewidth}{0.4pt}
\fancyhead[L]{\small\textcolor{pjatkGray}{PJATK -- Filia w Gdańsku \textbar\ Informatyka}}
\fancyhead[R]{\small\textcolor{pjatkGray}{Sylabus: JPT}}
\fancyfoot[C]{\small\thepage}

\titleformat{\section}{\large\bfseries\color{pjatkRed}}{\thesection.}{0.5em}{}
  [\color{pjatkRed}\rule{\linewidth}{0.8pt}]
\setlist{noitemsep, topsep=3pt, parsep=2pt}

\newmdenv[linecolor=pjatkRed, linewidth=1.2pt, backgroundcolor=pjatkLightGray,
  innerleftmargin=10pt, innerrightmargin=10pt, innertopmargin=8pt,
  innerbottommargin=8pt, roundcorner=4pt]{infobox}

\begin{document}

\AddToShipoutPictureBG{%
  \begin{tikzpicture}[remember picture, overlay]
    \node[opacity=0.5] at (current page.center) {%
      \includegraphics[width=14cm]{C:/Users/adamu/WebstormProjects/pj-studies/latex/PJATK_pl_sygnet_transparent-eps-converted-to}%
    };
  \end{tikzpicture}%
}

\begin{center}
  \includegraphics[height=2cm]{C:/Users/adamu/WebstormProjects/pj-studies/latex/PJATK_pl_poziom_1}\\[0.8cm]
  {\LARGE\bfseries\color{pjatkRed} SYLABUS PRZEDMIOTU}\\[0.8cm]
\end{center}

\begin{infobox}
\begin{tabularx}{\textwidth}{@{}lX@{}}
  \textbf{Nazwa przedmiotu:}  & {\bfseries Programowanie w języku Python} \\[3pt]
  \textbf{Kod przedmiotu:}    & JPT \\[3pt]
  \textbf{Kierunek / Profil:} & Informatyka / praktyczny \\[3pt]
  \textbf{Tryb studiów:}      & stacjonarny \\[3pt]
  \textbf{Rok / Semestr:}     & 1 / 2 \\[3pt]
  \textbf{Charakter:}         & obieralny \\[3pt]
  \textbf{Odpowiedzialny:}    & do ustalenia \\[3pt]
  \textbf{Wersja z dnia:}     & 20.02.2026 \\
\end{tabularx}
\end{infobox}

\vspace{1cm}

\section{Godziny zajęć i punkty ECTS}

\begin{center}
\begin{tabular}{|>{\centering\arraybackslash}p{2.0cm}
                |>{\centering\arraybackslash}p{2.0cm}
                |>{\centering\arraybackslash}p{2.0cm}
                |>{\centering\arraybackslash}p{2.4cm}
                |>{\centering\arraybackslash}p{2.4cm}
                |>{\centering\arraybackslash}p{2.0cm}
                |>{\centering\arraybackslash}p{1.4cm}|}
\hline
\rowcolor{tableHeader}
\textbf{Wykłady} & \textbf{Ćwiczenia} & \textbf{Laboratorium} &
\textbf{Z prowadzącym} & \textbf{Praca własna} & \textbf{Łącznie} & \textbf{ECTS} \\
\hline
15 h & --- & 15 h & 30 h & 20 h & 50 h & \textbf{2} \\
\hline
\end{tabular}
\end{center}

\section{Forma zajęć}

\begin{tabular}{ll}
  \hline
  \textbf{Forma zajęć} & \textbf{Sposób zaliczenia} \\
  \hline
  Projekt & Zaliczenie z oceną \\
  \hline
\end{tabular}

\section{Cel dydaktyczny}

Celem przedmiotu jest zapoznanie studentów z podstawami programowania w języku Python oraz wykształcenie umiejętności samodzielnego tworzenia prostych aplikacji i skryptów. Zajęcia kładą nacisk na czytelność kodu, dobre praktyki programistyczne oraz praktyczne zastosowania języka Python w analizie danych, automatyzacji zadań i tworzeniu prostych aplikacji. Przedmiot przygotowuje studentów do dalszej nauki technologii wykorzystujących Python oraz do realizacji niewielkich projektów programistycznych.

\section{Treści programowe}

\begin{enumerate}
  \item Wprowadzenie do języka Python: zastosowania, środowisko pracy, interpreter, pierwsze programy.
  \item Typy danych: int, float, bool, string, operatory i konwersje typów.
  \item Instrukcje sterujące: if, for, while, zakresy, instrukcje break i continue.
  \item Struktury danych: listy, krotki, słowniki, zbiory – operacje i zastosowania.
  \item Funkcje: definicja, parametry, wartości zwracane, zakres zmiennych.
  \item Moduły i pakiety, korzystanie z bibliotek standardowych.
  \item Obsługa plików: odczyt i zapis danych, formaty tekstowe.
  \item Wyjątki i obsługa błędów.
  \item Wprowadzenie do programowania obiektowego w Pythonie.
  \item  Dobre praktyki: PEP8, czytelność kodu, komentarze i dokumentacja.
  \item  Testowanie i debugowanie prostych programów.
\end{enumerate}

\section{Efekty kształcenia}

\subsection*{Wiedza}
\begin{itemize}
  \item Student zna podstawowe konstrukcje języka Python, typy danych, instrukcje sterujące oraz standardowe biblioteki. Rozumie zasady czytelnego zapisu kodu i znaczenie dobrych praktyk programistycznych.
\end{itemize}

\subsection*{Umiejętności}
\begin{itemize}
  \item Student potrafi tworzyć i testować proste programy w języku Python, korzystać z dokumentacji, dzielić program na funkcje i moduły oraz zaimplementować niewielki projekt programistyczny rozwiązujący praktyczny problem.
\end{itemize}

\subsection*{Kompetencje społeczne}
\begin{itemize}
  \item Student potrafi samodzielnie planować pracę nad projektem programistycznym, terminowo realizować zadania oraz prezentować efekty swojej pracy.
\end{itemize}

\section{Kryteria oceny}

\begin{itemize}
  \item Projekt programistyczny realizowany indywidualnie lub w parach – 100\%
  \item Ocena projektu obejmuje: poprawność działania, jakość kodu, strukturę programu, dokumentację oraz prezentację rozwiązania
  \item Warunkiem zaliczenia jest oddanie kompletnego projektu oraz jego pozytywna ocena
\end{itemize}

\section{Metody dydaktyczne}

Wykład, laboratoria, praca własna studenta.

\section{Literatura}

\textbf{Podstawowa:}
\begin{itemize}
  \item Mark Lutz, Learning Python, O'Reilly Media
  \item Eric Matthes, Python Crash Course, No Starch Press
  \item Dokumentacja Python: https://docs.python.org
\end{itemize}

\textbf{Uzupełniająca:}
\begin{itemize}
  \item Al Sweigart, Automate the Boring Stuff with Python
  \item Real Python – materiały online
\end{itemize}

\end{document}
