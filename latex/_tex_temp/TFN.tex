% ===========================================================
%  Sylabus: Technologie frontendowe (TFN)
% ===========================================================
\documentclass[12pt, a4paper]{article}

\usepackage[T1]{fontenc}
\usepackage[utf8]{inputenc}
\usepackage[polish]{babel}
\usepackage{lmodern}
\usepackage{microtype}
\usepackage[a4paper, top=2.5cm, bottom=2.5cm, left=2.5cm, right=2.5cm]{geometry}
\usepackage{xcolor}
\usepackage{graphicx}
\usepackage{booktabs}
\usepackage{tabularx}
\usepackage{longtable}
\usepackage{multirow}
\usepackage{array}
\usepackage{colortbl}
\usepackage{enumitem}
\usepackage{fancyhdr}
\usepackage{titlesec}
\usepackage{mdframed}
\usepackage[colorlinks=true, linkcolor=red!70!black, urlcolor=red!70!black]{hyperref}
\usepackage{eso-pic}
\usepackage{tikz}

\definecolor{pjatkRed}{RGB}{180,0,0}
\definecolor{pjatkGray}{RGB}{80,80,80}
\definecolor{pjatkLightGray}{RGB}{245,245,245}
\definecolor{tableHeader}{RGB}{220,220,220}

\pagestyle{fancy}\fancyhf{}
\renewcommand{\headrulewidth}{0.4pt}
\renewcommand{\footrulewidth}{0.4pt}
\fancyhead[L]{\small\textcolor{pjatkGray}{PJATK -- Filia w Gdańsku \textbar\ Informatyka}}
\fancyhead[R]{\small\textcolor{pjatkGray}{Sylabus: TFN}}
\fancyfoot[C]{\small\thepage}

\titleformat{\section}{\large\bfseries\color{pjatkRed}}{\thesection.}{0.5em}{}
  [\color{pjatkRed}\rule{\linewidth}{0.8pt}]
\setlist{noitemsep, topsep=3pt, parsep=2pt}

\newmdenv[linecolor=pjatkRed, linewidth=1.2pt, backgroundcolor=pjatkLightGray,
  innerleftmargin=10pt, innerrightmargin=10pt, innertopmargin=8pt,
  innerbottommargin=8pt, roundcorner=4pt]{infobox}

\begin{document}

\AddToShipoutPictureBG{%
  \begin{tikzpicture}[remember picture, overlay]
    \node[opacity=0.5] at (current page.center) {%
      \includegraphics[width=14cm]{C:/Users/adamu/WebstormProjects/pj-studies/latex/PJATK_pl_sygnet_transparent-eps-converted-to}%
    };
  \end{tikzpicture}%
}

\begin{center}
  \includegraphics[height=2cm]{C:/Users/adamu/WebstormProjects/pj-studies/latex/PJATK_pl_poziom_1}\\[0.8cm]
  {\LARGE\bfseries\color{pjatkRed} SYLABUS PRZEDMIOTU}\\[0.8cm]
\end{center}

\begin{infobox}
\begin{tabularx}{\textwidth}{@{}lX@{}}
  \textbf{Nazwa przedmiotu:}  & {\bfseries Technologie frontendowe} \\[3pt]
  \textbf{Kod przedmiotu:}    & TFN \\[3pt]
  \textbf{Kierunek / Profil:} & Informatyka / praktyczny \\[3pt]
  \textbf{Tryb studiów:}      & stacjonarny \\[3pt]
  \textbf{Rok / Semestr:}     & 3 / 5 \\[3pt]
  \textbf{Charakter:}         & obowiązkowy \\[3pt]
  \textbf{Odpowiedzialny:}    & mmiotk@pjwstk.edu.pl \\[3pt]
  \textbf{Wersja z dnia:}     & 19.02.2026 \\
\end{tabularx}
\end{infobox}

\vspace{1cm}

\section{Godziny zajęć i punkty ECTS}

\begin{center}
\begin{tabular}{|>{\centering\arraybackslash}p{2.0cm}
                |>{\centering\arraybackslash}p{2.0cm}
                |>{\centering\arraybackslash}p{2.0cm}
                |>{\centering\arraybackslash}p{2.4cm}
                |>{\centering\arraybackslash}p{2.4cm}
                |>{\centering\arraybackslash}p{2.0cm}
                |>{\centering\arraybackslash}p{1.4cm}|}
\hline
\rowcolor{tableHeader}
\textbf{Wykłady} & \textbf{Ćwiczenia} & \textbf{Laboratorium} &
\textbf{Z prowadzącym} & \textbf{Praca własna} & \textbf{Łącznie} & \textbf{ECTS} \\
\hline
30 h & 30 h & --- & 60 h & 65 h & 125 h & \textbf{5} \\
\hline
\end{tabular}
\end{center}

\section{Forma zajęć}

\begin{tabular}{ll}
  \hline
  \textbf{Forma zajęć} & \textbf{Sposób zaliczenia} \\
  \hline
  Laboratorium & Zaliczenie z oceną \\
  Wykład & Egzamin \\
  \hline
\end{tabular}

\section{Cel dydaktyczny}

Celem przedmiotu jest zapoznanie się z nowoczesnymi technologiami, technikami oraz narzędziami do wytwarzania graficznego interfejsu użytkownikach w aplikacjach webowych.

\section{Przedmioty wprowadzające}

\begin{tabularx}{\textwidth}{lX}
  \hline
  \textbf{Przedmiot} & \textbf{Wymagane zagadnienia} \\
  \hline
  Warsztaty programistyczne & Technologie internetu \\
  Dobra znajomość HTML, CSS oraz JavaScript & --- \\
  \hline
\end{tabularx}

\section{Treści programowe}

\begin{enumerate}
  \item Historia frontendu. Język znaczników HTML i CSS
  \item Użycie preprocesorów CSS jako alternatywa pisania stylów w aplikacjach webowych
  \item Animacje oraz inne technologie wspierające proces pisania interfejsów użytkownika za pomocą CSS
  \item Pojęcie responsywności oraz dostępności w aplikacjach webowych
  \item Architektura aplikacji SPA: Środowisko Node.js
  \item Wprowadzenie do frameworka budowania aplikacji webowych w środowisku Node.js
  \item Format JSX. Integracja CSS w aplikacji.
  \item Pojęcie stanu aplikacji w aplikacjach webowych
  \item Formularze oraz ich walidacja w aplikacjach webowych
  \item Inne rodzaje hooków i dobre praktyki w tworzeniu aplikacji webowych
  \item Integracja danych w aplikacjach webowych
  \item Trasowanie aplikacji w aplikacjach webowych
  \item Testowanie aplikacji frontendowych
  \item Technologia suspense w aplikacjach webowych
  \item Metody optymalizacji w aplikacjach webowych
  \item Informacje dodatkowe
  \item Wymagane oprogramowanie w laboratoriach: Serwer Node.js.
  \item Uzasadnienie dla prowadzenia przedmiotu/współpraca z rynkiem pracy
  \item A. W jakiego typu firmach będą potrzebne umiejętności nabyte w trakcie zajęć:
  \item Wszystkie firmy zajmujące się budowanie serwisów internetowych.
  \item B. W jakich zawodach wiedza i umiejętności są istotne:
  \item Wiedza i umiejętności z tworzenia interaktywnych interfejsów użytkownika jest bardzo przydatna we
  \item wszystkich zawodach informatycznych.
\end{enumerate}

\section{Efekty kształcenia}

\subsection*{Wiedza}
\begin{itemize}
  \item Student zna i rozumie pojęcia związane z pojęciem Frontend Development. Student zna i rozumie pojęcie preprocesora CSS oraz tworzenia aplikacji webowej z wykorzystaniem narzędzia Node.js oraz przykładowego narzędzia w tym środowisku.
  \item Student zna i rozumie zaawansowane  pojęcia związane z pojęciem Frontend Development. Student zna i rozumie zaawansowane pojęcie preprocesora CSS oraz tworzenia aplikacji webowej z wykorzystaniem narzędzia Node.js oraz przykładowego narzędzia w tym środowisku.
  \item Student zna i rozumie elementy programowania funkcyjnego na przykładzie języka JavaScript. Student zna i rozumie techniki oraz metody programistyczne wykorzystywane w języku JavaScript.
  \item Student zna i rozumie pojęcia z zakresu tworzenia interaktywnych aplikacji internetowych, umożliwiających komunikację typu człowiek-komputer.
\end{itemize}

\subsection*{Umiejętności}
\begin{itemize}
  \item Student potrafi ocenić przydatność różnych podejść programistycznych na podstawie języka JavaScript i związanych z nimi środowisk na przykładzie narzędzia Node.js
  \item Student potrafi wyspecyfikować, zaprojektować, zaimplementować, przetestować oraz zdebuggować aplikację webową; potrafi korzystać z bibliotek przy użyciu narzędzia npm, środowisk programistycznych, integrujących i uruchomieniowych.
  \item Student potrafi wytworzyć warstwową aplikację webową w oparciu o wybrane wzorce architektoniczne , przy pomocy narzędzia Node.js oraz wybranego narzędzia w tym środowisku.
  \item Student potrafi wytworzyć zaawansowaną warstwową aplikację webową w oparciu o wybrane wzorce architektoniczne , przy pomocy narzędzia Node.js oraz wybranego narzędzia w tym środowisku.
\end{itemize}

\subsection*{Kompetencje społeczne}
\begin{itemize}
  \item Student jest gotów do jest gotów do samodzielnego uczenia się przez całe życie
\end{itemize}

\section{Kryteria oceny}

\begin{itemize}
  \item wykład z elementami dyskusji z prezentacją multimedialną
  \item burza mózgów
  \item rozwiązywanie zadań
  \item analiza przypadków
  \item projekt praktyczny
  \item Kryteria oceny
  \item 50\% Kolokwium
  \item 50\% Projekt praktyczny
  \item 40\% Ocena z laboratorium
  \item 60\% Egzamin
\end{itemize}

\section{Metody dydaktyczne}

Wykład, laboratoria, praca własna studenta.

\section{Literatura}

\textbf{Podstawowa:}
\begin{itemize}
  \item E. Porcello and A. Banks, Learning react: Modern patterns for developing react apps, O’Reilly Media, Incorporated, 2020.
  \item A. Prabhu, Beggining CSS Preprocessors. With Sass, Compass and Less, Apress, 2015.
  \item C. Rippon, Learn react with typescript 3: Beginner’s guide to modern react web development with typescript 3, Packt Publishing, 2018.
  \item M.T. Thomas, React in action, Manning Publications, 2018.
  \item F. Zammetti, Modern full-stack development: Using typescript, react, node.js, webpack, and docker, Apress, 2020.
\end{itemize}

\textbf{Uzupełniająca:}
\begin{itemize}
  \item A. Boduch and R. Derks, React and react native: A complete hands-on guide to modern web and mobile development with react.js, Packt Publishing, 2020.
  \item E. Brown, Web Development with Node and Express, O'Reilly Media, Incorporated, 2020.
  \item A. Freeman, Pro react 16, Apress, 2019.
  \item D. Griffits and D. Griffits, React Cookbook. Recipes for Mastering the React Framework, O'Reily, 2021.
  \item G. Lim, Beginning react with hooks, Independently Published, 2020.
  \item C.S. Roldán, React cookbook: Create dynamic web apps with react using redux, webpack, node.js, and graphql, Packt Publishing, 2018.
  \item R. Wieruch, The road to react: Your journey to master plain yet pragmatic react.js, Independently Published, 2017.
\end{itemize}

\end{document}
