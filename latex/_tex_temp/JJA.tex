% ===========================================================
%  Sylabus: Programowanie w języku Java (JJA)
% ===========================================================
\documentclass[12pt, a4paper]{article}

\usepackage[T1]{fontenc}
\usepackage[utf8]{inputenc}
\usepackage[polish]{babel}
\usepackage{lmodern}
\usepackage{microtype}
\usepackage[a4paper, top=2.5cm, bottom=2.5cm, left=2.5cm, right=2.5cm]{geometry}
\usepackage{xcolor}
\usepackage{graphicx}
\usepackage{booktabs}
\usepackage{tabularx}
\usepackage{longtable}
\usepackage{multirow}
\usepackage{array}
\usepackage{colortbl}
\usepackage{enumitem}
\usepackage{fancyhdr}
\usepackage{titlesec}
\usepackage{mdframed}
\usepackage[colorlinks=true, linkcolor=red!70!black, urlcolor=red!70!black]{hyperref}
\usepackage{eso-pic}
\usepackage{tikz}

\definecolor{pjatkRed}{RGB}{180,0,0}
\definecolor{pjatkGray}{RGB}{80,80,80}
\definecolor{pjatkLightGray}{RGB}{245,245,245}
\definecolor{tableHeader}{RGB}{220,220,220}

\pagestyle{fancy}\fancyhf{}
\renewcommand{\headrulewidth}{0.4pt}
\renewcommand{\footrulewidth}{0.4pt}
\fancyhead[L]{\small\textcolor{pjatkGray}{PJATK -- Filia w Gdańsku \textbar\ Informatyka}}
\fancyhead[R]{\small\textcolor{pjatkGray}{Sylabus: JJA}}
\fancyfoot[C]{\small\thepage}

\titleformat{\section}{\large\bfseries\color{pjatkRed}}{\thesection.}{0.5em}{}
  [\color{pjatkRed}\rule{\linewidth}{0.8pt}]
\setlist{noitemsep, topsep=3pt, parsep=2pt}

\newmdenv[linecolor=pjatkRed, linewidth=1.2pt, backgroundcolor=pjatkLightGray,
  innerleftmargin=10pt, innerrightmargin=10pt, innertopmargin=8pt,
  innerbottommargin=8pt, roundcorner=4pt]{infobox}

\begin{document}

\AddToShipoutPictureBG{%
  \begin{tikzpicture}[remember picture, overlay]
    \node[opacity=0.5] at (current page.center) {%
      \includegraphics[width=14cm]{C:/Users/adamu/WebstormProjects/pj-studies/latex/PJATK_pl_sygnet_transparent-eps-converted-to}%
    };
  \end{tikzpicture}%
}

\begin{center}
  \includegraphics[height=2cm]{C:/Users/adamu/WebstormProjects/pj-studies/latex/PJATK_pl_poziom_1}\\[0.8cm]
  {\LARGE\bfseries\color{pjatkRed} SYLABUS PRZEDMIOTU}\\[0.8cm]
\end{center}

\begin{infobox}
\begin{tabularx}{\textwidth}{@{}lX@{}}
  \textbf{Nazwa przedmiotu:}  & {\bfseries Programowanie w języku Java} \\[3pt]
  \textbf{Kod przedmiotu:}    & JJA \\[3pt]
  \textbf{Kierunek / Profil:} & Informatyka / praktyczny \\[3pt]
  \textbf{Tryb studiów:}      & stacjonarny \\[3pt]
  \textbf{Rok / Semestr:}     & 1 / 2 \\[3pt]
  \textbf{Charakter:}         & obieralny \\[3pt]
  \textbf{Odpowiedzialny:}    & do ustalenia \\[3pt]
  \textbf{Wersja z dnia:}     & 20.02.2026 \\
\end{tabularx}
\end{infobox}

\vspace{1cm}

\section{Godziny zajęć i punkty ECTS}

\begin{center}
\begin{tabular}{|>{\centering\arraybackslash}p{2.0cm}
                |>{\centering\arraybackslash}p{2.0cm}
                |>{\centering\arraybackslash}p{2.0cm}
                |>{\centering\arraybackslash}p{2.4cm}
                |>{\centering\arraybackslash}p{2.4cm}
                |>{\centering\arraybackslash}p{2.0cm}
                |>{\centering\arraybackslash}p{1.4cm}|}
\hline
\rowcolor{tableHeader}
\textbf{Wykłady} & \textbf{Ćwiczenia} & \textbf{Laboratorium} &
\textbf{Z prowadzącym} & \textbf{Praca własna} & \textbf{Łącznie} & \textbf{ECTS} \\
\hline
15 h & --- & 15 h & 30 h & 20 h & 50 h & \textbf{2} \\
\hline
\end{tabular}
\end{center}

\section{Forma zajęć}

\begin{tabular}{ll}
  \hline
  \textbf{Forma zajęć} & \textbf{Sposób zaliczenia} \\
  \hline
  Projekt & Zaliczenie z oceną \\
  \hline
\end{tabular}

\section{Cel dydaktyczny}

Celem przedmiotu jest zapoznanie studentów z podstawami programowania w języku Java oraz wykształcenie umiejętności tworzenia prostych aplikacji z wykorzystaniem paradygmatu obiektowego. Zajęcia kładą nacisk na poprawne modelowanie klas, czytelność kodu, obsługę wyjątków oraz wykorzystanie standardowych bibliotek języka Java. Przedmiot przygotowuje studentów do dalszej nauki technologii opartych na JVM oraz do realizacji niewielkich projektów programistycznych.

\section{Treści programowe}

\begin{enumerate}
  \item Wprowadzenie do języka Java, platformy JVM oraz środowiska programistycznego
  \item Typy danych, zmienne, operatory oraz instrukcje sterujące
  \item Metody, przekazywanie parametrów oraz organizacja kodu
  \item Tablice oraz kolekcje wprowadzające (List, Set, Map)
  \item Podstawy programowania obiektowego: klasy, obiekty, enkapsulacja
  \item Konstruktory, dziedziczenie i polimorfizm
  \item Obsługa wyjątków i mechanizmy kontroli błędów
  \item Operacje wejścia i wyjścia oraz praca z plikami
  \item Dobre praktyki programistyczne i czytelność kodu
  \item Projekt programistyczny – implementacja i prezentacja
\end{enumerate}

\section{Efekty kształcenia}

\subsection*{Wiedza}
\begin{itemize}
  \item Student zna podstawowe konstrukcje języka Java, typy danych, instrukcje sterujące oraz zasady programowania obiektowego. Rozumie działanie maszyny wirtualnej JVM oraz rolę bibliotek standardowych.
\end{itemize}

\subsection*{Umiejętności}
\begin{itemize}
  \item Student potrafi tworzyć, kompilować i uruchamiać proste aplikacje w języku Java, projektować klasy i ich relacje, obsługiwać wyjątki oraz zaimplementować niewielki projekt programistyczny rozwiązujący praktyczny problem.
\end{itemize}

\subsection*{Kompetencje społeczne}
\begin{itemize}
  \item Student potrafi samodzielnie planować i realizować projekt programistyczny, dbać o jakość kodu oraz terminowo prezentować efekty swojej pracy.
\end{itemize}

\section{Kryteria oceny}

\begin{itemize}
  \item Projekt programistyczny realizowany indywidualnie lub w parach – 100\%
  \item Ocena projektu obejmuje: poprawność działania aplikacji, jakość i czytelność kodu, strukturę projektu, obsługę wyjątków oraz dokumentację
  \item Warunkiem zaliczenia jest oddanie kompletnego projektu oraz jego pozytywna ocena
\end{itemize}

\section{Metody dydaktyczne}

Wykład, laboratoria, praca własna studenta.

\section{Literatura}

\textbf{Podstawowa:}
\begin{itemize}
  \item Cay S. Horstmann, Core Java, Volume I
  \item Herbert Schildt, Java: The Complete Reference
  \item Dokumentacja Java: https://docs.oracle.com/javase
\end{itemize}

\textbf{Uzupełniająca:}
\begin{itemize}
  \item Joshua Bloch, Effective Java
  \item Oracle Java Tutorials
\end{itemize}

\end{document}
