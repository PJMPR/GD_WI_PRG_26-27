% ===========================================================
%  Sylabus: Trendy w rozwoju API (TAPI)
% ===========================================================
\documentclass[12pt, a4paper]{article}

\usepackage[T1]{fontenc}
\usepackage[utf8]{inputenc}
\usepackage[polish]{babel}
\usepackage{lmodern}
\usepackage{microtype}
\usepackage[a4paper, top=2.5cm, bottom=2.5cm, left=2.5cm, right=2.5cm]{geometry}
\usepackage{xcolor}
\usepackage{graphicx}
\usepackage{booktabs}
\usepackage{tabularx}
\usepackage{longtable}
\usepackage{multirow}
\usepackage{array}
\usepackage{colortbl}
\usepackage{enumitem}
\usepackage{fancyhdr}
\usepackage{titlesec}
\usepackage{mdframed}
\usepackage[colorlinks=true, linkcolor=red!70!black, urlcolor=red!70!black]{hyperref}
\usepackage{eso-pic}
\usepackage{tikz}

\definecolor{pjatkRed}{RGB}{180,0,0}
\definecolor{pjatkGray}{RGB}{80,80,80}
\definecolor{pjatkLightGray}{RGB}{245,245,245}
\definecolor{tableHeader}{RGB}{220,220,220}

\pagestyle{fancy}\fancyhf{}
\renewcommand{\headrulewidth}{0.4pt}
\renewcommand{\footrulewidth}{0.4pt}
\fancyhead[L]{\small\textcolor{pjatkGray}{PJATK -- Filia w Gdańsku \textbar\ Informatyka}}
\fancyhead[R]{\small\textcolor{pjatkGray}{Sylabus: TAPI}}
\fancyfoot[C]{\small\thepage}

\titleformat{\section}{\large\bfseries\color{pjatkRed}}{\thesection.}{0.5em}{}
  [\color{pjatkRed}\rule{\linewidth}{0.8pt}]
\setlist{noitemsep, topsep=3pt, parsep=2pt}

\newmdenv[linecolor=pjatkRed, linewidth=1.2pt, backgroundcolor=pjatkLightGray,
  innerleftmargin=10pt, innerrightmargin=10pt, innertopmargin=8pt,
  innerbottommargin=8pt, roundcorner=4pt]{infobox}

\begin{document}

\AddToShipoutPictureBG{%
  \begin{tikzpicture}[remember picture, overlay]
    \node[opacity=0.5] at (current page.center) {%
      \includegraphics[width=14cm]{C:/Users/adamu/WebstormProjects/pj-studies/latex/PJATK_pl_sygnet_transparent-eps-converted-to}%
    };
  \end{tikzpicture}%
}

\begin{center}
  \includegraphics[height=2cm]{C:/Users/adamu/WebstormProjects/pj-studies/latex/PJATK_pl_poziom_1}\\[0.8cm]
  {\LARGE\bfseries\color{pjatkRed} SYLABUS PRZEDMIOTU}\\[0.8cm]
\end{center}

\begin{infobox}
\begin{tabularx}{\textwidth}{@{}lX@{}}
  \textbf{Nazwa przedmiotu:}  & {\bfseries Trendy w rozwoju API} \\[3pt]
  \textbf{Kod przedmiotu:}    & TAPI \\[3pt]
  \textbf{Kierunek / Profil:} & Informatyka / praktyczny \\[3pt]
  \textbf{Tryb studiów:}      & stacjonarny \\[3pt]
  \textbf{Rok / Semestr:}     & 4 / 7 \\[3pt]
  \textbf{Charakter:}         & obowiązkowy \\[3pt]
  \textbf{Odpowiedzialny:}    & mgr Mateusz Miotk \\[3pt]
  \textbf{Wersja z dnia:}     & 19.02.2026 \\
\end{tabularx}
\end{infobox}

\vspace{1cm}

\section{Godziny zajęć i punkty ECTS}

\begin{center}
\begin{tabular}{|>{\centering\arraybackslash}p{2.0cm}
                |>{\centering\arraybackslash}p{2.0cm}
                |>{\centering\arraybackslash}p{2.0cm}
                |>{\centering\arraybackslash}p{2.4cm}
                |>{\centering\arraybackslash}p{2.4cm}
                |>{\centering\arraybackslash}p{2.0cm}
                |>{\centering\arraybackslash}p{1.4cm}|}
\hline
\rowcolor{tableHeader}
\textbf{Wykłady} & \textbf{Ćwiczenia} & \textbf{Laboratorium} &
\textbf{Z prowadzącym} & \textbf{Praca własna} & \textbf{Łącznie} & \textbf{ECTS} \\
\hline
30 h & 30 h & --- & 60 h & 65 h & 125 h & \textbf{5} \\
\hline
\end{tabular}
\end{center}

\section{Forma zajęć}

\begin{tabular}{ll}
  \hline
  \textbf{Forma zajęć} & \textbf{Sposób zaliczenia} \\
  \hline
  Laboratorium & Zaliczenie z oceną \\
  Wykład & Egzamin \\
  \hline
\end{tabular}

\section{Cel dydaktyczny}

Przedmiot "Trendy w rozwoju API" ma na celu zaznajomienie studentów z aktualnymi kierunkami rozwoju i innowacjami w dziedzinie tworzenia API dla aplikacji internetowych. Studenci zdobędą wiedzę teoretyczną i praktyczną na temat najnowszych standardów, protokołów i narzędzi używanych do projektowania, implementacji i zarządzania interfejsami API. Uczestnicy będą mieli okazję zgłębić tematykę takich kwestii jak: bezpieczeństwo API, skalowalność, wydajność oraz dobre praktyki programistyczne. Poznają również różnorodność stylów API, w tym REST, GraphQL, gRPC i inne, analizując ich zalety i wady oraz zastosowania w różnych kontekstach projektowych. Kolejnym istotnym elementem przedmiotu jest praca grupowa, która ma na celu rozwijanie umiejętności współpracy, komunikacji i innowacyjnego myślenia. Studenci będą mieli za zadanie zanalizować i opracować materiał dotyczący wybranej technologii z zakresu tworzenia API, co pozwoli na pogłębienie wiedzy i umiejętności praktycznych w konkretnym obszarze.

\section{Przedmioty wprowadzające}

\begin{tabularx}{\textwidth}{lX}
  \hline
  \textbf{Przedmiot} & \textbf{Wymagane zagadnienia} \\
  \hline
  Technologie Frontendowe & Technologie Backendowe \\
  Technologie internetu & Umiejętność posługiwania się emulatorem terminala w systemie operacyjnym \\
  Znajomość struktur plików i katalogów w systemie operacyjnym & Znajomość pojęcia procesu w systemie operacyjnym \\
  Znajomość protokołu HTTP & Znajomość pojęć z zakresu tworzenia aplikacji internetowej z podziałem na frontend oraz backend. \\
  \hline
\end{tabularx}

\section{Treści programowe}

\begin{enumerate}
  \item Wprowadzenie do API (Application Programming Interface). Przegląd różnych typów API (REST, SOAP, GraphQL, gRPC, itd.). Znaczenie i role API w rozwoju aplikacji internetowych
  \item Konwencje, normy i najlepsze praktyki w tworzeniu REST API
  \item GraphQL - wprowadzenie i porównanie z REST. Tworzenie prostych API z wykorzystaniem GraphQL
  \item Wprowadzenie do gRPC. Porównanie gRPC z REST i GraphQL, zalety i wady
  \item Rola i znaczenie dokumentacji API. Narzędzia do tworzenia dokumentacji API - OpenAPI, Swagger, Postman
  \item Język TypeScript
  \item Inne znane technologie wspierające proces budowania aplikacji internetowych.
\end{enumerate}

\section{Efekty kształcenia}

\subsection*{Wiedza}
\begin{itemize}
  \item Student rozumie podstawowe koncepcje i zasady rządzące współczesnymi API, w tym ich role i zastosowania w różnych typach aplikacji internetowych.
  \item Student rozumie metody zapewniania jakości i testowania API, włączając w to testy jednostkowe, integracyjne oraz testy obciążenia.
  \item Student rozumie kwestie związane z zabezpieczaniem API, w tym uwierzytelnianie, autoryzację oraz ochronę przed powszechnymi zagrożeniami i atakami.
  \item Student rozumie etyczne i prawne aspekty tworzenia i zarządzania API, w tym prywatność, ochronę danych i zgodność z regulacjami prawnymi.
\end{itemize}

\subsection*{Umiejętności}
\begin{itemize}
  \item Student umie projektować interfejsy API, z wykorzystaniem odpowiednich metod, protokołów i standardów, aby zapewnić ich wydajność, bezpieczeństwo i skalowalność.
  \item Student umie analizować i oceniać różne style i architektury API (takie jak REST, GraphQL, gRPC), rozumiejąc ich zalety, wady i najlepsze zastosowania.
  \item Student umie efektywnie komunikować się w zespole programistycznym, współtworzyć dokumentację API i udostępniać informacje o API zespołowi i interesariuszom.
  \item Student umie wykorzystywać narzędzia i technologie wspierające proces tworzenia, wdrażania i monitorowania API, zwiększając efektywność i jakość pracy.
  \item Student umie projektować zaawansowane interfejsy API, z wykorzystaniem odpowiednich metod, protokołów i standardów, aby zapewnić ich wydajność, bezpieczeństwo i skalowalność.
\end{itemize}

\subsection*{Kompetencje społeczne}
\begin{itemize}
  \item Student jest gotów do jest gotów do samodzielnego uczenia się przez całe życie
  \item Student potrafi pracować w zespole nad wspólnym projektem związanym z tworzeniem API, wykazując się umiejętnościami analitycznymi, kreatywnymi i komunikacyjnymi.
\end{itemize}

\section{Kryteria oceny}

\begin{itemize}
  \item wykład z elementami dyskusji z prezentacją multimedialną
  \item burza mózgów
  \item rozwiązywanie zadań
  \item analiza przypadków
  \item projekt praktyczny
  \item Kryteria oceny
  \item 50\% Projekt programistyczny
  \item 50\% Prezentacja projektu oraz dokumentacji
  \item 40\% Ocena z laboratorium
  \item 60\% Egzamin z zagadnień poruszanych na wykładzie
\end{itemize}

\section{Metody dydaktyczne}

Wykład, laboratoria, praca własna studenta.

\section{Literatura}

\textbf{Podstawowa:}
\begin{itemize}
  \item M. Amundsen, RESTful Web API Patterns and Practices Cookbook, O’Reilly Media, 2023.
  \item S. Buna, GraphQL in Action, Manning Publications, 2021.
  \item A. Freeman, Essential TypeScript, Apress, 2019.
  \item J. S. Ponelat, Designing APIs with Swagger and OpenAPI, Manning Publications, 2021.
\end{itemize}

\textbf{Uzupełniająca:}
\begin{itemize}
  \item F, Dogilo, REST API Development with Node.js, Apress, 2018.
  \item JJ. Geewax, API Design Patterns, Manning Publications, 2021.
\end{itemize}

\end{document}
