% ===========================================================
%  Sylabus: JĘZYK ANGIELSKI ()
% ===========================================================
\documentclass[12pt, a4paper]{article}

\usepackage[T1]{fontenc}
\usepackage[utf8]{inputenc}
\usepackage[polish]{babel}
\usepackage{lmodern}
\usepackage{microtype}
\usepackage[a4paper, top=2.5cm, bottom=2.5cm, left=2.5cm, right=2.5cm]{geometry}
\usepackage{xcolor}
\usepackage{graphicx}
\usepackage{booktabs}
\usepackage{tabularx}
\usepackage{longtable}
\usepackage{multirow}
\usepackage{array}
\usepackage{colortbl}
\usepackage{enumitem}
\usepackage{fancyhdr}
\usepackage{titlesec}
\usepackage{mdframed}
\usepackage[colorlinks=true, linkcolor=red!70!black, urlcolor=red!70!black]{hyperref}
\usepackage{eso-pic}
\usepackage{tikz}

\definecolor{pjatkRed}{RGB}{180,0,0}
\definecolor{pjatkGray}{RGB}{80,80,80}
\definecolor{pjatkLightGray}{RGB}{245,245,245}
\definecolor{tableHeader}{RGB}{220,220,220}

\pagestyle{fancy}\fancyhf{}
\renewcommand{\headrulewidth}{0.4pt}
\renewcommand{\footrulewidth}{0.4pt}
\fancyhead[L]{\small\textcolor{pjatkGray}{PJATK -- Filia w Gdańsku \textbar\ Informatyka}}
\fancyhead[R]{\small\textcolor{pjatkGray}{Sylabus: }}
\fancyfoot[C]{\small\thepage}

\titleformat{\section}{\large\bfseries\color{pjatkRed}}{\thesection.}{0.5em}{}
  [\color{pjatkRed}\rule{\linewidth}{0.8pt}]
\setlist{noitemsep, topsep=3pt, parsep=2pt}

\newmdenv[linecolor=pjatkRed, linewidth=1.2pt, backgroundcolor=pjatkLightGray,
  innerleftmargin=10pt, innerrightmargin=10pt, innertopmargin=8pt,
  innerbottommargin=8pt, roundcorner=4pt]{infobox}

\begin{document}

\AddToShipoutPictureBG{%
  \begin{tikzpicture}[remember picture, overlay]
    \node[opacity=0.5] at (current page.center) {%
      \includegraphics[width=14cm]{C:/Users/adamu/WebstormProjects/pj-studies/latex/PJATK_pl_sygnet_transparent-eps-converted-to}%
    };
  \end{tikzpicture}%
}

\begin{center}
  \includegraphics[height=2cm]{C:/Users/adamu/WebstormProjects/pj-studies/latex/PJATK_pl_poziom_1}\\[0.8cm]
  {\LARGE\bfseries\color{pjatkRed} SYLABUS PRZEDMIOTU}\\[0.8cm]
\end{center}

\begin{infobox}
\begin{tabularx}{\textwidth}{@{}lX@{}}
  \textbf{Nazwa przedmiotu:}  & {\bfseries JĘZYK ANGIELSKI} \\[3pt]
  \textbf{Kod przedmiotu:}    &  \\[3pt]
  \textbf{Kierunek / Profil:} & Informatyka / praktyczny \\[3pt]
  \textbf{Tryb studiów:}      & stacjonarny \\[3pt]
  \textbf{Rok / Semestr:}     &  /  \\[3pt]
  \textbf{Charakter:}         & obowiązkowy \\[3pt]
  \textbf{Odpowiedzialny:}    & Studium Języków Obcych \\[3pt]
  \textbf{Wersja z dnia:}     & 19.02.2026 \\
\end{tabularx}
\end{infobox}

\vspace{1cm}

\section{Godziny zajęć i punkty ECTS}

\begin{center}
\begin{tabular}{|>{\centering\arraybackslash}p{2.0cm}
                |>{\centering\arraybackslash}p{2.0cm}
                |>{\centering\arraybackslash}p{2.0cm}
                |>{\centering\arraybackslash}p{2.4cm}
                |>{\centering\arraybackslash}p{2.4cm}
                |>{\centering\arraybackslash}p{2.0cm}
                |>{\centering\arraybackslash}p{1.4cm}|}
\hline
\rowcolor{tableHeader}
\textbf{Wykłady} & \textbf{Ćwiczenia} & \textbf{Laboratorium} &
\textbf{Z prowadzącym} & \textbf{Praca własna} & \textbf{Łącznie} & \textbf{ECTS} \\
\hline
--- & 60 h & --- & 60 h & --- & 60 h & \textbf{3} \\
\hline
\end{tabular}
\end{center}

\section{Forma zajęć}

\begin{tabular}{ll}
  \hline
  \textbf{Forma zajęć} & \textbf{Sposób zaliczenia} \\
  \hline
  Lektorat & Zaliczenie z oceną \\
  \hline
\end{tabular}

\section{Cel dydaktyczny}

Abstrakt po polsku: Celem kursu jest nauczanie studentów ogólnego języka angielskiego uzupełnionego o wprowadzenie elementów języka specjalistycznego, tj. biznesu, grafiki, sztuki nowych mediów oraz o treści typowo akademickie na poziomie docelowym B2. Studenci z wejściowym poziomem minimum B2 osiągają poziom C1 lub C2. Kurs obejmuje nauczanie specjalistycznego słownictwa, sprawności komunikacyjnych w biznesie, zwłaszcza umiejętności przygotowania i przedstawienia profesjonalnej prezentacji, jak również rozwijanie podstawowych sprawności komunikacyjnych, a szczególnie szybkiego czytania tekstów specjalistycznych zwłaszcza z zakresu grafiki (artykuły prasowe, książki, strony internetowe poświęcone tematyce) i słuchania tekstów o wzrastającym poziomie trudności. Podczas kursu studenci zostają zapoznani z terminologią dotyczącą grafiki, przygotowują się do porozumiewania się na tematy fachowe podczas pracy. Abstrakt po angielsku: The aim of the course is to pass the knowledge enhancing the basic English  lexis of the specific aspects of business and the studied area (graphic design), teach communication skills and raise the awareness of cross-cultural circumstances in business communication, as well as to revise and expand general language skills to the B2-C2 level (depending on the initial language level of students). Communicative skills in business are transferable to the native language. Secondary education graduates are usually deficient in this matter. These skills and abilities are indispensable in the modern working environment, irrespective of the sector and area of specialization.

\section{Przedmioty wprowadzające}

\begin{tabularx}{\textwidth}{lX}
  \hline
  \textbf{Przedmiot} & \textbf{Wymagane zagadnienia} \\
  \hline
  Znajomość języka angielskiego na poziomie min.A2+/B1. & --- \\
  \hline
\end{tabularx}

\section{Treści programowe}

\begin{enumerate}
  \item 1. Zagadnienia z matematyki np.
  \item algebra,
  \item liczby naturalne,
  \item liczby rzeczywiste,
  \item liczby całkowite,
  \item liczby wymierne,
  \item działania wykonywane na liczbach,
  \item notacja pozycyjna,
  \item figury geometryczne,
  \item macierze,
  \item typy danych,
  \item tryb całkowitoliczbowy,
  \item tryb zmiennoprzecinkowy,
  \item tekstowe typy danych etc.
  \item 2.Zagadnienia biznesowe (w zależności od poziomu zaawansowania grupy) np.
  \item debatowanie,
  \item wywieranie wpływu,
  \item podejmowanie decyzji,
  \item komunikacja w biznesie,
  \item marki,
  \item zatrudnienie,
  \item etyka zawodowa,
  \item team building,
  \item style zarządzania,  rynki międzynarodowe etc.
  \item 3. Teksty specjalistyczne o tematyce związanej z informatyką, technologią.  Pogłębianie umiejętności czytania ze zrozumieniem tekstów bazujących na oryginalnych materiałach źródłowych.
  \item 4. Zagadnienia z zakresu informatyki (w zależności od poziomu zaawansowania grupy np).
  \item zastosowanie komputerów w życiu codziennym,
  \item systemy operacyjne,
  \item mikroprocesory,
  \item bazy danych,
  \item Internet,
  \item wirusy komputerowe,
  \item gry komputerowe, sieci komputerowe,
  \item nowe technologie,
  \item języki programowania,
  \item oprogramowanie komputerowe etc.)
  \item 5. Pogłębianie znajomości podstawowych oraz specjalistycznych wyrażeń i zwrotów z zakresu języka technicznego, akademickiego i świata pracy.
  \item 6.Ćwiczenie złożonych struktur leksykalnych, omówienie właściwości fizycznych materii, kształtów, wprowadzenie terminologii matematycznej, interpretacja rysunków, diagramów, opis procesu. Wprowadzenie słownictwa specjalistycznego z dziedziny informatyki.
  \item 7.Realizowanie gramatyki w zakresie wymaganym dla danego poziomu znajomości języka. Nauczanie struktur niezbędnych do komunikacji werbalnej i pisemnej w środowisku akademickim i świata pracy.
  \item 8.Pisanie różnorodnych tekstów, niezbędnych w pracy i na uczelni, np.: raportu, życiorysu zawodowego, wiadomości email, streszczenia, notatki, abstraktu, instrukcji, objaśnienia procesu.
  \item 9. Słuchanie w oparciu o materiały przedstawiające sytuacje związane ze
  \item środowiskiem pracy, akademickim i życiem codziennym, np.: rozmowy telefoniczne, wywiady, sytuacje związane z obsługą klienta, wykłady oraz prezentacje.
  \item 10. Komunikacja w świecie pracy i społeczności akademickiej, np: prezentacje, rozmowa kwalifikacyjna, rozmowy formalne i nieformalne, negocjacje, przedstawianie argumentów, rozwiązywanie problemów, case studies, prowadzenie spotkań formalnych, itp. Ćwiczenie wymowy i prawidłowego akcentowania wyrazów.
  \item 11. Analiza i rozumienie języka specjalistycznego oraz tłumaczenie krótkich tekstów technicznych. K\_U01
  \item 12. Nauka prezentowania w języku angielskim, w tym używania adekwatnych konstrukcji językowych i odpowiedniego słownictwa, analiza technik prezentowania, zapoznanie się z prawidłową strukturą prezentacji i odpowiednim włączaniem materiałów wizualnych, przygotowanie kilkuminutowej prezentacji o tematyce związanej z kierunkiem studiów, wybraną specjalizacją i zainteresowaniami.(
  \item 13. Ćwiczenie umiejętności językowych (mówienie, słuchania, pisania i czytania oraz wzbogacanie słownictwa i gramatyki pod kątem przygotowania studentów do egzaminów międzynarodowych Cambridge ESOL (BEC, FCE, CAE, CPE, IELTS) oraz TOEIC Reading\&Listening.
  \item 14. Wypowiedzi ustne studentów (prezentacje multimedialne, artykuły prasowe,  prezentacje projektów etc.)
  \item 15. Colloquium / Podsumowanie semestru.
\end{enumerate}

\section{Efekty kształcenia}

\subsection*{Wiedza}
\begin{itemize}
  \item Brak danych.
\end{itemize}

\subsection*{Umiejętności}
\begin{itemize}
  \item Student potrafi pozyskiwać informacje z różnych źródeł bez naruszania praw autorskich.
  \item Student potrafi zrozumieć, przeanalizować i tłumaczyć teksty techniczne w języku angielskim.
  \item Student potrafi posługiwać się językiem formalnym.
  \item Student potrafi poprawnie komunikować się w języku angielskim w środowisku akademickim i zawodowym używając fachowego słownictwa korzystając z narzędzi telekomunikacyjnych i prezentacji multimedialnych.
  \item Student potrafi komunikować się w języku angielskim w życiu codziennym.
  \item Student potrafi pisać abstrakt, streszczenie, pisać wykres, diagram, list motywacyjny i CV wykorzystując słownictwo i struktury gramatyczne w odpowiednim kontekście.
  \item Student potrafi konstruować wypowiedź ustną i pisemną poprawną pod względem logicznym i merytorycznym wdrażając słownictwo specjalistyczne i biznesowe.
\end{itemize}

\subsection*{Kompetencje społeczne}
\begin{itemize}
  \item Student jest gotów do permanentnego doskonalenia kompetencji językowych w celach zawodowych.
  \item Student jest gotów do ciągłego samokształcenia własnych kompetencji językowych oraz dostrzega potrzebę podejmowania inicjatyw w dziedzinach biznesowych.
\end{itemize}

\section{Kryteria oceny}

\begin{itemize}
  \item analiza tekstów z dyskusją
  \item rozwiązywanie zadań
  \item burza mózgów
  \item mind map
  \item praca grupowa nad projektem z wykorzystaniem nowoczesnych technik informatycznych
  \item warsztaty
  \item Kryteria oceny
  \item obecność na zajęciach- max. 4  nieusprawiedliwione nieobecności w czasie semestru (zwolnienia lekarskie przedstawiane na bieżąco)
  \item prace pisemne- min. jedna praca na semestr-w zależności od poziomu i tempa pracy grupy
  \item formal/ informal letter/ e-mail
  \item covering letter
  \item CV
  \item report
  \item memo
  \item discursive essay
  \item postcard
  \item note
  \item portfolio
  \item kolokwia/ test/ kartkówki na podstawie opracowanego materiału - min. 2 na semestr
  \item wypowiedź ustna- (o rodzaju decyduje lektor)
  \item news report
  \item newspaper article presentation
  \item multimedia presentation
  \item group projects
  \item przygotowywanie prac domowych
  \item aktywne uczestnictwo w zajęciach
  \item bieżąca obserwacja pracy studenta
  \item Kryteria zaliczenia:
  \item uzyskanie przynajmniej 51\% możliwych punktów za prace pisemne/ testy
  \item pozytywna (min. 3.0) ocena łączna za pozostałe formy
\end{itemize}

\section{Metody dydaktyczne}

Wykład, laboratoria, praca własna studenta.

\section{Literatura}

\textbf{Podstawowa:}
\begin{itemize}
  \item 1.Margaret O’Keeffee, Iwonna Dubicka, Bob Dignem, Mike Hogan, Business Partner A2+-C1; Wydawnictwo PEARSON, 2018
  \item Cotton D, Falvey D, Kent S, Rogers J, Market Leader 3rd edition (pre-intermediate/ intermediate/ upper-intermediate / advanced), PEARSON 2015.
  \item 2. Maksymowicz Roman, Język angielski dla elektroników i informatyków. Wydawnictwo Oświatowe FOSZE.
  \item 3. Błaszczyk Beata, English 4 IT. Praktyczny kurs języka angielskiego dla specjalistów i nie tylko. Helion, 2017.
  \item 4. Maciejewska J, Kucharska-Raczunas A, Information Technology for students of technical studies. Wydawnictwo Politechniki Gdańskiej.
  \item 5. Esteras, S.R. i Fabre, E. M. Professional English in Use, ICT. Cambridge, 2007.
  \item 6. Esteras, S. R. Infotech,  English for computer users. Cambridge, 2008.
  \item 7. Maciejewska, J, Kucharska-Raczunas A, English for Information Technology, Wydawnictwo Politechniki Gdańskiej, 2012.
  \item 8. Mascull Bill, Business Vocabulary in Use, Cambridge University Press, 2004.
  \item 9. Glendinning Eric H, Oxford English for Information Technology 2nd ed, Oxford Univeristy Press, 2006.
\end{itemize}

\textbf{Uzupełniająca:}
\begin{itemize}
  \item 1. R. Murphy, English Grammar in Use, Cambridge University Press, Cambridge 2011.
  \item 2. G. Gójska, Technical English Grammar, Wydawnictwo Politechniki Gdańskiej, Gdańsk 2000.
  \item 3. I. Mokwa - Tarnowska, Technical Writing in English, Wydawnictwo Politechniki Gdańskiej, Gdańsk 2006.
  \item 4. Badger, I. ed. English for Work, Everyday Technical English. Pearson Longman: Harlow, 2003.
  \item 5. Bonamy, D. Technical English. Pearson Longman: Harlow, 2011.
  \item 6. Brieger, N. i Pohl, A. Technical English Vocabulary and Grammar, Summertown Publishing: Oxford, 2007.
  \item 7.Fitzgerald P, McCullagh M, Tabor C, English for ICT Studies in Higher Education Studies, Garnet Education, 2011.
  \item 8. Comfort J, Effective Presentations, Oxford University Press - kurs video na poziomie średnim plus.
  \item 9. Skrypty, artykuły popularnonaukowe i naukowe
\end{itemize}

\end{document}
