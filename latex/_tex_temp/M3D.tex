% ===========================================================
%  Sylabus: Modelowanie 3D dla gier (M3D)
% ===========================================================
\documentclass[12pt, a4paper]{article}

\usepackage[T1]{fontenc}
\usepackage[utf8]{inputenc}
\usepackage[polish]{babel}
\usepackage{lmodern}
\usepackage{microtype}
\usepackage[a4paper, top=2.5cm, bottom=2.5cm, left=2.5cm, right=2.5cm]{geometry}
\usepackage{xcolor}
\usepackage{graphicx}
\usepackage{booktabs}
\usepackage{tabularx}
\usepackage{longtable}
\usepackage{multirow}
\usepackage{array}
\usepackage{colortbl}
\usepackage{enumitem}
\usepackage{fancyhdr}
\usepackage{titlesec}
\usepackage{mdframed}
\usepackage[colorlinks=true, linkcolor=red!70!black, urlcolor=red!70!black]{hyperref}
\usepackage{eso-pic}
\usepackage{tikz}

\definecolor{pjatkRed}{RGB}{180,0,0}
\definecolor{pjatkGray}{RGB}{80,80,80}
\definecolor{pjatkLightGray}{RGB}{245,245,245}
\definecolor{tableHeader}{RGB}{220,220,220}

\pagestyle{fancy}\fancyhf{}
\renewcommand{\headrulewidth}{0.4pt}
\renewcommand{\footrulewidth}{0.4pt}
\fancyhead[L]{\small\textcolor{pjatkGray}{PJATK -- Filia w Gdańsku \textbar\ Informatyka}}
\fancyhead[R]{\small\textcolor{pjatkGray}{Sylabus: M3D}}
\fancyfoot[C]{\small\thepage}

\titleformat{\section}{\large\bfseries\color{pjatkRed}}{\thesection.}{0.5em}{}
  [\color{pjatkRed}\rule{\linewidth}{0.8pt}]
\setlist{noitemsep, topsep=3pt, parsep=2pt}

\newmdenv[linecolor=pjatkRed, linewidth=1.2pt, backgroundcolor=pjatkLightGray,
  innerleftmargin=10pt, innerrightmargin=10pt, innertopmargin=8pt,
  innerbottommargin=8pt, roundcorner=4pt]{infobox}

\begin{document}

\AddToShipoutPictureBG{%
  \begin{tikzpicture}[remember picture, overlay]
    \node[opacity=0.5] at (current page.center) {%
      \includegraphics[width=14cm]{C:/Users/adamu/WebstormProjects/pj-studies/latex/PJATK_pl_sygnet_transparent-eps-converted-to}%
    };
  \end{tikzpicture}%
}

\begin{center}
  \includegraphics[height=2cm]{C:/Users/adamu/WebstormProjects/pj-studies/latex/PJATK_pl_poziom_1}\\[0.8cm]
  {\LARGE\bfseries\color{pjatkRed} SYLABUS PRZEDMIOTU}\\[0.8cm]
\end{center}

\begin{infobox}
\begin{tabularx}{\textwidth}{@{}lX@{}}
  \textbf{Nazwa przedmiotu:}  & {\bfseries Modelowanie 3D dla gier} \\[3pt]
  \textbf{Kod przedmiotu:}    & M3D \\[3pt]
  \textbf{Kierunek / Profil:} & Informatyka / praktyczny \\[3pt]
  \textbf{Tryb studiów:}      & niestacjonarny \\[3pt]
  \textbf{Rok / Semestr:}     & 3 / 5 \\[3pt]
  \textbf{Charakter:}         & obowiązkowy \\[3pt]
  \textbf{Odpowiedzialny:}    & Dr Piotr Arłukowicz \\[3pt]
  \textbf{Wersja z dnia:}     & 19.02.2026 \\
\end{tabularx}
\end{infobox}

\vspace{1cm}

\section{Godziny zajęć i punkty ECTS}

\begin{center}
\begin{tabular}{|>{\centering\arraybackslash}p{2.0cm}
                |>{\centering\arraybackslash}p{2.0cm}
                |>{\centering\arraybackslash}p{2.0cm}
                |>{\centering\arraybackslash}p{2.4cm}
                |>{\centering\arraybackslash}p{2.4cm}
                |>{\centering\arraybackslash}p{2.0cm}
                |>{\centering\arraybackslash}p{1.4cm}|}
\hline
\rowcolor{tableHeader}
\textbf{Wykłady} & \textbf{Ćwiczenia} & \textbf{Laboratorium} &
\textbf{Z prowadzącym} & \textbf{Praca własna} & \textbf{Łącznie} & \textbf{ECTS} \\
\hline
30 h & --- & 30 h & 60 h & 40 h & 100 h & \textbf{4} \\
\hline
\end{tabular}
\end{center}

\section{Forma zajęć}

\begin{tabular}{ll}
  \hline
  \textbf{Forma zajęć} & \textbf{Sposób zaliczenia} \\
  \hline
  Laboratorium & Zaliczenie z oceną \\
  Wykład & Nieoceniany \\
  \hline
\end{tabular}

\section{Cel dydaktyczny}

Celem przedmiotu jest przygotowanie studentów do pracy w branży Gamedev w charakterze twórców treści cyfrowych. Nacisk kładziony jest na modelowanie 3D oraz związane z nim tematy takie jak teksturowanie, tworzenie materiałów, wypalanie, techniki optymalizacyjne i strategie związane z tworzeniem gier.

\section{Przedmioty wprowadzające}

\begin{tabularx}{\textwidth}{lX}
  \hline
  \textbf{Przedmiot} & \textbf{Wymagane zagadnienia} \\
  \hline
  Niezbędna jest ogólna znajomość obsługi komputera. & --- \\
  \hline
\end{tabularx}

\section{Treści programowe}

\begin{enumerate}
  \item 1. Wprowadzenie do podstaw obsługi Blendera.
  \item 2. Modelowanie w trybie obiektowym.
  \item 3. Modyfikatory.
  \item 4. Modelowanie w trybie edycji.
  \item 5. Modelowanie high-poly w trybie rzeźbienia.
  \item 6. Oświetlenie.
  \item 7. Materiały.
  \item 8. Tekstury proceduralne.
  \item 9. Mapowanie tekstur i przestrzenie współrzędnych.
  \item 10. Tekstury bitmapowe.
  \item 11. Wypalanie tekstur.
  \item 12. Optymalizacja sceny 3D i rendering.
  \item 13. Komponowanie (compositing).
\end{enumerate}

\section{Efekty kształcenia}

\subsection*{Wiedza}
\begin{itemize}
  \item Student zna i rozumie pojęcia takie jak modelowanie, low-poly, mid-poly, high-poly, teksturowanie, cieniowanie, mapowanie, przestrzenie współrzędnych, komponowanie, wypalanie, optymalizacja
\end{itemize}

\subsection*{Umiejętności}
\begin{itemize}
  \item Student potrafi wymodelować obiekty w technice low-poly oraz high-poly
  \item Student potrafi oteksturować obiekt metodami tekstur proceduralnych
  \item Student potrafi rozwinąć siatkę obiektu we współrzędnych UV
  \item Student potrafi wypalić tekstury proceduralne do tekstur bitmapowych
\end{itemize}

\subsection*{Kompetencje społeczne}
\begin{itemize}
  \item Student jest gotów do współpracy i dzielenia się swoją wiedzą
  \item Student jest gotów do dalszej nauki i pogłębiania swojej wiedzy i umiejętności
\end{itemize}

\section{Kryteria oceny}

\begin{itemize}
  \item prezentacja na żywo
  \item wykład z elementami dyskusji
  \item studium przypadku
  \item najlepsze praktyki
  \item Ćwiczenia / Laboratorium/Lektorat:
  \item praca samodzielna ucznia nad zadaniami i mini-projektami
  \item projekt semestralny
  \item Kryteria oceny
  \item Student musi wykonać zgodnie z regułami sztuki kluczowe ćwiczenia prezentowane w trakcie semestru. Sprawdzana jest zgodność z zasadami ale dopuszczane są inwencja twórcza oraz kreatywność (dziedzina jest twórcza). Tworzone jest w ciągu semestru portfolio rozwojowe pokazujące eksperymenty własne i ścieżkę osiągnięć, które są następnie na końcu zajęć oceniane.
  \item Student zobowiązany jest przedstawić większy projekt własny w którym wykorzystuje poznane techniki modelowania. Projekty są negocjowane indywidualnie już podczas semestru i można je przygotowywać w trakcie jego trwania.
\end{itemize}

\section{Metody dydaktyczne}

Wykład, laboratoria, praca własna studenta.

\section{Literatura}

\textbf{Podstawowa:}
\begin{itemize}
  \item 2. Blender 3D For Beginners: The Complete Guide: The Complete Beginner’s Guide to Getting Started with Navigating, Modeling, Animating, Texturing, Lighting, Compositing and Rendering within Blender, ISBN-13: 978-1523238811
\end{itemize}

\textbf{Uzupełniająca:}
\begin{itemize}
  \item 1. Mind-Melding Unity and Blender for 3D Game Development: Unleash the power of Unity and Blender to create amazing games, ISBN-13: 978-1801071550
\end{itemize}

\end{document}
