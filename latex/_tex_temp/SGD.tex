% ===========================================================
%  Sylabus: Symulacje i gry decyzyjne (SGD)
% ===========================================================
\documentclass[12pt, a4paper]{article}

\usepackage[T1]{fontenc}
\usepackage[utf8]{inputenc}
\usepackage[polish]{babel}
\usepackage{lmodern}
\usepackage{microtype}
\usepackage[a4paper, top=2.5cm, bottom=2.5cm, left=2.5cm, right=2.5cm]{geometry}
\usepackage{xcolor}
\usepackage{graphicx}
\usepackage{booktabs}
\usepackage{tabularx}
\usepackage{longtable}
\usepackage{multirow}
\usepackage{array}
\usepackage{colortbl}
\usepackage{enumitem}
\usepackage{fancyhdr}
\usepackage{titlesec}
\usepackage{mdframed}
\usepackage[colorlinks=true, linkcolor=red!70!black, urlcolor=red!70!black]{hyperref}
\usepackage{eso-pic}
\usepackage{tikz}

\definecolor{pjatkRed}{RGB}{180,0,0}
\definecolor{pjatkGray}{RGB}{80,80,80}
\definecolor{pjatkLightGray}{RGB}{245,245,245}
\definecolor{tableHeader}{RGB}{220,220,220}

\pagestyle{fancy}\fancyhf{}
\renewcommand{\headrulewidth}{0.4pt}
\renewcommand{\footrulewidth}{0.4pt}
\fancyhead[L]{\small\textcolor{pjatkGray}{PJATK -- Filia w Gdańsku \textbar\ Informatyka}}
\fancyhead[R]{\small\textcolor{pjatkGray}{Sylabus: SGD}}
\fancyfoot[C]{\small\thepage}

\titleformat{\section}{\large\bfseries\color{pjatkRed}}{\thesection.}{0.5em}{}
  [\color{pjatkRed}\rule{\linewidth}{0.8pt}]
\setlist{noitemsep, topsep=3pt, parsep=2pt}

\newmdenv[linecolor=pjatkRed, linewidth=1.2pt, backgroundcolor=pjatkLightGray,
  innerleftmargin=10pt, innerrightmargin=10pt, innertopmargin=8pt,
  innerbottommargin=8pt, roundcorner=4pt]{infobox}

\begin{document}

\AddToShipoutPictureBG{%
  \begin{tikzpicture}[remember picture, overlay]
    \node[opacity=0.5] at (current page.center) {%
      \includegraphics[width=14cm]{C:/Users/adamu/WebstormProjects/pj-studies/latex/PJATK_pl_sygnet_transparent-eps-converted-to}%
    };
  \end{tikzpicture}%
}

\begin{center}
  \includegraphics[height=2cm]{C:/Users/adamu/WebstormProjects/pj-studies/latex/PJATK_pl_poziom_1}\\[0.8cm]
  {\LARGE\bfseries\color{pjatkRed} SYLABUS PRZEDMIOTU}\\[0.8cm]
\end{center}

\begin{infobox}
\begin{tabularx}{\textwidth}{@{}lX@{}}
  \textbf{Nazwa przedmiotu:}  & {\bfseries Symulacje i gry decyzyjne} \\[3pt]
  \textbf{Kod przedmiotu:}    & SGD \\[3pt]
  \textbf{Kierunek / Profil:} & Informatyka / praktyczny \\[3pt]
  \textbf{Tryb studiów:}      & niestacjonarny \\[3pt]
  \textbf{Rok / Semestr:}     & 4 / 8 \\[3pt]
  \textbf{Charakter:}         & obieralny \\[3pt]
  \textbf{Odpowiedzialny:}    & dr Elżbieta Puźniakowska-Gałuch, ela@pjwstk.edu.pl,  dr Tadeusz Puźniakowski,   pantadeusz@pjwstk.edu.pl \\[3pt]
  \textbf{Wersja z dnia:}     & 19.02.2026 \\
\end{tabularx}
\end{infobox}

\vspace{1cm}

\section{Godziny zajęć i punkty ECTS}

\begin{center}
\begin{tabular}{|>{\centering\arraybackslash}p{2.0cm}
                |>{\centering\arraybackslash}p{2.0cm}
                |>{\centering\arraybackslash}p{2.0cm}
                |>{\centering\arraybackslash}p{2.4cm}
                |>{\centering\arraybackslash}p{2.4cm}
                |>{\centering\arraybackslash}p{2.0cm}
                |>{\centering\arraybackslash}p{1.4cm}|}
\hline
\rowcolor{tableHeader}
\textbf{Wykłady} & \textbf{Ćwiczenia} & \textbf{Laboratorium} &
\textbf{Z prowadzącym} & \textbf{Praca własna} & \textbf{Łącznie} & \textbf{ECTS} \\
\hline
8 h & --- & 8 h & 16 h & 34 h & 50 h & \textbf{2} \\
\hline
\end{tabular}
\end{center}

\section{Forma zajęć}

\begin{tabular}{ll}
  \hline
  \textbf{Forma zajęć} & \textbf{Sposób zaliczenia} \\
  \hline
  Ćwiczenia & Zaliczenie z oceną \\
  Wykład & Nieoceniany \\
  \hline
\end{tabular}

\section{Cel dydaktyczny}

Przedmiot "Symulacje i gry decyzyjne" ma na celu: zapoznanie studentów z wybranymi metodami wspomagania decyzji nabycie umiejętności budowy modeli decyzyjnych z pomocą oprogramowania nabycie umiejętności podejmowania decyzji w dynamicznych środowiskach decyzyjnych. Symulacje i gry decyzyjne to przedmiot, gdzie studenci w części praktycznej mogą nauczyć się jak tworzyć gry z naciskiem na projektowanie logiki i strategii.

\section{Przedmioty wprowadzające}

\begin{tabularx}{\textwidth}{lX}
  \hline
  \textbf{Przedmiot} & \textbf{Wymagane zagadnienia} \\
  \hline
  Wiedza z zakresu szkoły średniej. Algebra liniowa. & --- \\
  Wymagane jest ukończenie kursu Animacja 3D lub posiadanie certykatu BFCT (Blender Foundation Certied Trainer). & --- \\
  \hline
\end{tabularx}

\section{Treści programowe}

\begin{enumerate}
  \item Modele decyzyjne statyczne. Przykłady reprezentacji problemów decyzyjnych. Postać normalna,  drzewa decyzyjne.
  \item Tablice decyzyjne, drzewa decyzyjne. Punkty równowagi w sensie Pareto. Gry o sumie zerowej.  Twierdzenie o minimaksie. Strategia optymalna. Gry 2 na 2. Gry 2 na n i n na 2. Rozwiązywanie gier metodą graficzną.
  \item Strategie mieszane i czyste. Punkty równowagi Nasha w strategiach czystych i mieszanych. Zastosowanie modelu w rzeczywistości. Strategie zdominowane i dominujące.   Rozwiązywanie gier 3 na 3.
  \item Programowanie liniowe. Dualność. Algorytm sympleks. Gry z ograniczeniami.  Zastosowanie programowania liniowego w rzeczywistych zagadnieniach.
  \item Gry w postaci ekstensywnej. Gry grane wielokrotnie. Nauka w grach powtarzanych wielokrotnie.
  \item Gry nieskończone. Punkty równowagi w grach powtarzanych nieskończenie wiele razy. Gry powtarzane ze zredukowaną wypłatą. Twierdzenie dla gier ze zredukowaną wypłatą.
  \item Twierdzenie o indukcji wstecznej. Problemy decyzyjne.  Gry Bayesowskie.
  \item Podstawowe elementy silnika gry komputerowej – pętla gry, zarządzanie czasem symulacji.
  \item Poznaje podstawy biblioteki SDL2, która jest stosowana w branży gier komputerowych
  \item Poznaje podstawy symulacji fizycznej – modelowanie zmian świata gry wraz z prostą fizyką.
\end{enumerate}

\section{Efekty kształcenia}

\subsection*{Wiedza}
\begin{itemize}
  \item Student zna i rozumie podstawowe zasady symulacji komputerowych. Zna elementy silnika.
  \item Student zna i rozumie, jak zastosować pojęcia teorii gier w różnych problemach decyzyjnych oraz grach przedstawianych w różnych formach.  Zna podstawowe zasady symulacji komputerowych.
\end{itemize}

\subsection*{Umiejętności}
\begin{itemize}
  \item Student potrafi formułować i  rozwiązywać gry w postaci normalnej i ekstensywnej. Potrafi rozwiązywać problemy decyzyjne.  Potrafi przeprowadzać proste wnioskowanie statystyczne.
  \item Student potrafi zaprogramować prosty silnik gry komputerowej. Rozumie sposoby realizacji fizyki w grach komputerowych i jest w stanie stworzyć własny prosty silnik fizyczny. Rozumie metody tworzenia prostych efektów cząsteczkowych.
  \item Student potrafi formułować i  rozwiązywać gry w postaci normalnej i ekstensywnej. Potrafi rozwiązywać problemy decyzyjne. Potrafi przeprowadzić symulację takiej gry lub problemu decyzyjnego.
\end{itemize}

\section{Kryteria oceny}

\begin{itemize}
  \item wykład z elementami dyskusji z prezentacją multimedialną
  \item burza mózgów
  \item rozwiązywanie zadań
  \item analiza przypadków
  \item Kryteria oceny
  \item Kilka kartkówek. Na zaliczenie części teoretycznej ćwiczeń student jest zobowiązany uzyskać wynik powyżej 50\% możliwych punktów.  Istnieje możliwość poprawy kartkówek na dziewiątych zajęciach.
  \item Projekt programistyczny.
  \item Ustalenie oceny zaliczeniowej na podstawie ocen cząstkowych otrzymanych w  trakcie trwania semestru
\end{itemize}

\section{Metody dydaktyczne}

Wykład, laboratoria, praca własna studenta.

\section{Literatura}

\textbf{Podstawowa:}
\begin{itemize}
  \item - Deulofeu J., Dylemat więźniów i zwycięskie strategie. Teoria gier., RBA, 2022
  \item - Watson J. Stategia. Wprowadzenie do teorii gier. WNT Warszawa 2004.
  \item - Gajda J.B., Prognozowanie i symulacja a decyzje gospodarcze, Wyd. C.H.Beck, Warszawa 2001
  \item - Rymarczyk M. (red.): Gry, symulacje, sieci. Wyd. WBS Poznań, 1997.
  \item - Poundstone, W. (1992), Prisoner's Dilemma: John von Neumann, Game Theory and the Puzzle of the Bomb
  \item - Maynard-Smith, J. (1982), Evolution and the theory of games
  \item - Axelrod R. Advancing the Art of Simulation in the Social Sciences
  \item -Hucara Ł., „Projektowanie i programowanie gier video”, Wydawnictwo PJWSTK, 2011
  \item -Edams E., „Projektowanie gier”, Helion 2011
  \item - Dokumentacja biblioteki SDL2 dostępna online na stronie projektu – jest najbardziej aktualna
\end{itemize}

\textbf{Uzupełniająca:}
\begin{itemize}
  \item - Orlin B., Math Games with Bad Drawings: 75 ¼ Simple, Challenging, Go-Anywhere Games – And Why They Matter,  Black Dog \& Leventhal, 2022
  \item - DeLoura M. ,  Game programming gems 1 - 6– seria książek
  \item - Nystrom R, 2014, Game programming patterns
\end{itemize}

\end{document}
