% ===========================================================
%  Sylabus: Przetwarzanie Języka Naturalnego (PJN)
% ===========================================================
\documentclass[12pt, a4paper]{article}

\usepackage[T1]{fontenc}
\usepackage[utf8]{inputenc}
\usepackage[polish]{babel}
\usepackage{lmodern}
\usepackage{microtype}
\usepackage[a4paper, top=2.5cm, bottom=2.5cm, left=2.5cm, right=2.5cm]{geometry}
\usepackage{xcolor}
\usepackage{graphicx}
\usepackage{booktabs}
\usepackage{tabularx}
\usepackage{longtable}
\usepackage{multirow}
\usepackage{array}
\usepackage{colortbl}
\usepackage{enumitem}
\usepackage{fancyhdr}
\usepackage{titlesec}
\usepackage{mdframed}
\usepackage[colorlinks=true, linkcolor=red!70!black, urlcolor=red!70!black]{hyperref}
\usepackage{eso-pic}
\usepackage{tikz}

\definecolor{pjatkRed}{RGB}{180,0,0}
\definecolor{pjatkGray}{RGB}{80,80,80}
\definecolor{pjatkLightGray}{RGB}{245,245,245}
\definecolor{tableHeader}{RGB}{220,220,220}

\pagestyle{fancy}\fancyhf{}
\renewcommand{\headrulewidth}{0.4pt}
\renewcommand{\footrulewidth}{0.4pt}
\fancyhead[L]{\small\textcolor{pjatkGray}{PJATK -- Filia w Gdańsku \textbar\ Informatyka}}
\fancyhead[R]{\small\textcolor{pjatkGray}{Sylabus: PJN}}
\fancyfoot[C]{\small\thepage}

\titleformat{\section}{\large\bfseries\color{pjatkRed}}{\thesection.}{0.5em}{}
  [\color{pjatkRed}\rule{\linewidth}{0.8pt}]
\setlist{noitemsep, topsep=3pt, parsep=2pt}

\newmdenv[linecolor=pjatkRed, linewidth=1.2pt, backgroundcolor=pjatkLightGray,
  innerleftmargin=10pt, innerrightmargin=10pt, innertopmargin=8pt,
  innerbottommargin=8pt, roundcorner=4pt]{infobox}

\begin{document}

\AddToShipoutPictureBG{%
  \begin{tikzpicture}[remember picture, overlay]
    \node[opacity=0.5] at (current page.center) {%
      \includegraphics[width=14cm]{C:/Users/adamu/WebstormProjects/pj-studies/latex/PJATK_pl_sygnet_transparent-eps-converted-to}%
    };
  \end{tikzpicture}%
}

\begin{center}
  \includegraphics[height=2cm]{C:/Users/adamu/WebstormProjects/pj-studies/latex/PJATK_pl_poziom_1}\\[0.8cm]
  {\LARGE\bfseries\color{pjatkRed} SYLABUS PRZEDMIOTU}\\[0.8cm]
\end{center}

\begin{infobox}
\begin{tabularx}{\textwidth}{@{}lX@{}}
  \textbf{Nazwa przedmiotu:}  & {\bfseries Przetwarzanie Języka Naturalnego} \\[3pt]
  \textbf{Kod przedmiotu:}    & PJN \\[3pt]
  \textbf{Kierunek / Profil:} & Informatyka / praktyczny \\[3pt]
  \textbf{Tryb studiów:}      & niestacjonarny \\[3pt]
  \textbf{Rok / Semestr:}     & 4 / 8 \\[3pt]
  \textbf{Charakter:}         & obowiązkowy \\[3pt]
  \textbf{Odpowiedzialny:}    & Dr Tadeusz Puźniakowski \\[3pt]
  \textbf{Wersja z dnia:}     & 19.02.2026 \\
\end{tabularx}
\end{infobox}

\vspace{1cm}

\section{Godziny zajęć i punkty ECTS}

\begin{center}
\begin{tabular}{|>{\centering\arraybackslash}p{2.0cm}
                |>{\centering\arraybackslash}p{2.0cm}
                |>{\centering\arraybackslash}p{2.0cm}
                |>{\centering\arraybackslash}p{2.4cm}
                |>{\centering\arraybackslash}p{2.4cm}
                |>{\centering\arraybackslash}p{2.0cm}
                |>{\centering\arraybackslash}p{1.4cm}|}
\hline
\rowcolor{tableHeader}
\textbf{Wykłady} & \textbf{Ćwiczenia} & \textbf{Laboratorium} &
\textbf{Z prowadzącym} & \textbf{Praca własna} & \textbf{Łącznie} & \textbf{ECTS} \\
\hline
30 h & --- & 30 h & 60 h & 65 h & 125 h & \textbf{5} \\
\hline
\end{tabular}
\end{center}

\section{Forma zajęć}

\begin{tabular}{ll}
  \hline
  \textbf{Forma zajęć} & \textbf{Sposób zaliczenia} \\
  \hline
  Wykład & Egzamin \\
  \hline
\end{tabular}

\section{Cel dydaktyczny}

Zapoznanie z podstawowymi problemami wizji komputerowej, metodami ich rozwiązywania oraz oceny przygotowywanych rozwiązań.

\section{Przedmioty wprowadzające}

\begin{tabularx}{\textwidth}{lX}
  \hline
  \textbf{Przedmiot} & \textbf{Wymagane zagadnienia} \\
  \hline
  Narzędzia Sztucznej Inteligencji & Future of Deep Learning \\
  Algebra liniowa z geometrią & Praca ze środowiskami: Colab, HuggingFace, metryki, zadania uczenia maszynowego \\
  Podstawy działania sieci neuronowej, transfer learning, self-supervised learning, foundation models, podstawy pracy w PyTorchu & Podstawowe operacje na macierzach \\
  \hline
\end{tabularx}

\section{Treści programowe}

\begin{enumerate}
  \item Wprowadzenie do NLP
  \item •Historia NLP: od pierwszych prób do współczesnych zastosowań
  \item •Kluczowe zadania w NLP: analiza sentymentu, tłumaczenie maszynowe, rozpoznawanie nazwanych encji (NER)
  \item Tokenizacja i Wektory Słów
  \item •Metody tokenizacji: podejścia oparte na regułach i uczeniu maszynowym
  \item •Wektoryzacja słów: Word2Vec, GloVe, FastText
  \item Rekurencyjne Sieci Neuronowe (RNN)
  \item •Architektura RNN i jej zastosowania w NLP
  \item •Problemy z RNN: zanikanie gradientu, eksplozja gradientu
  \item •Wprowadzenie do LSTM i GRU
  \item Transformery
  \item •Wprowadzenie do architektury transformera
  \item •Zastosowania transformera w NLP
  \item Wariacje Transformerów
  \item •BERT, GPT, T5: różnice i zastosowania
  \item •Jak dostosować architekturę transformera do różnych zadań
  \item Modele Języka
  \item •Budowa modeli językowych
  \item •Zastosowania i wyzwania związane z modelami językowymi
  \item Enkodery
  \item •Rola enkoderów w modelach transformacyjnych
  \item •Przykłady zastosowań enkoderów w różnych zadaniach NLP
  \item Fine-Tuning
  \item •Proces dostrajania modeli pretrenowanych
  \item •Techniki fine-tuningu dla różnych zadań NLP
  \item Kompresja Modeli
  \item •Wprowadzenie do kompresji modeli: skwantyzowanie, prunning
  \item •Korzyści i wyzwania związane z kompresją
  \item Stworzenie Skwantyzowanej Wersji Modelu
  \item •Praktyczne podejście do kompresji modeli
  \item •Implementacja skwantyzacji na przykładzie wybranego modelu
  \item Pair Programming - Rozpoznawanie Nazwanych Encji (NER)
  \item •Wprowadzenie do NER
  \item •Przykład implementacji NER za pomocą Lightning Transformers
  \item •Warsztaty praktyczne
  \item Pair Programming - Chatbot
  \item •Projektowanie chatbota: wymagania i architektura
  \item •Implementacja chatbota na przykładzie Chartum
  \item •Warsztaty praktyczne
  \item Przykłady Aplikacji NLP
  \item •Przegląd rzeczywistych zastosowań NLP w różnych branżach
  \item •Studium przypadku: analiza skuteczności modeli NLP w zastosowaniach komercyjnych
  \item Podsumowanie i Dyskusja
  \item •Przegląd kluczowych zagadnień omówionych w kursie
  \item •Dyskusja na temat przyszłości NLP i jego wpływu na różne dziedziny
\end{enumerate}

\section{Efekty kształcenia}

\subsection*{Wiedza}
\begin{itemize}
  \item Student potrafi: zrozumieć podstawowe zagadnienia probabilistyki i statystyki, co jest niezbędne do analizy danych tekstowych i tworzenia modeli językowych w NLP. Wiedza ta pozwala na skuteczne modelowanie i interpretację wyników, co jest kluczowe w praktycznych zastosowaniach NLP.
  \item Student potrafi: zrozumieć podstawowe zagadnienia probabilistyki i statystyki, co jest niezbędne do analizy danych tekstowych i tworzenia modeli językowych w NLP. Wiedza ta pozwala na skuteczne modelowanie i interpretację wyników, co jest kluczowe w praktycznych zastosowaniach NLP.
  \item Student potrafi: zrozumieć zaawansowane pojęcia związane z wybraną specjalizacją w NLP, co pozwala na rozwijanie innowacyjnych rozwiązań oraz adaptację aktualnych technologii w praktyce. Wiedza ta umożliwia studentom angażowanie się w badania i rozwój w dziedzinie NLP.
  \item Student potrafi: zastosować zaawansowane pojęcia z zakresu programowania, co jest niezbędne przy tworzeniu, testowaniu i uruchamianiu aplikacji wykorzystujących NLP. Umiejętność programowania w różnych językach oraz znajomość technik i narzędzi programistycznych jest kluczowa w realizacji projektów związanych z NLP.
\end{itemize}

\subsection*{Umiejętności}
\begin{itemize}
  \item Student potrafi: pozyskiwać i analizować specjalistyczne informacje z literatury oraz źródeł internetowych w obszarze NLP. Umiejętność krytycznej analizy i syntezy informacji jest niezbędna do rozwoju projektów oraz badań naukowych w dziedzinie przetwarzania języka naturalnego.
  \item Student potrafi: analizować i wyjaśniać zjawiska związane z NLP, a także tworzyć modele rzeczywiste. Umiejętność weryfikacji modeli i ich zastosowania do predykcji stanów jest kluczowa w rozwoju aplikacji i systemów opartych na NLP.
\end{itemize}

\subsection*{Kompetencje społeczne}
\begin{itemize}
  \item Student potrafi: samodzielnie uczyć się przez całe życie, co jest niezbędne w szybko rozwijającej się dziedzinie NLP. Techniki i narzędzia w tej dziedzinie stale ewoluują, więc umiejętność ciągłego kształcenia się jest kluczowa dla rozwoju zawodowego.
  \item Student potrafi: współdziałać w zespole, co jest ważne w projektach związanych z NLP, gdzie często wymagane są różnorodne umiejętności i kompetencje. Praca zespołowa sprzyja innowacyjności i efektywności w realizacji złożonych zadań.
\end{itemize}

\section{Kryteria oceny}

\begin{itemize}
  \item Case study - prezentacja oprogramowania (ostatnie 4 zajęcia).
  \item Ćwiczenia / Laboratorium/Lektorat:
  \item Rozwiązywanie zadań w Google Colab (pierwsze 10 zajęć).
  \item Projekt zespołowy (ostatnie 5 zajęć).
  \item Laboratorium oraz Projekt
  \item Kryteria oceny
  \item Ocena zadań laboratoryjnych
  \item Prezentacja projektu zespołowego w trakcie sesji posterowej, jakość kodu i dokumentacji. 50\% laboratorium, 50\% projekt.
  \item Skala ocen
  \item 51\%-60\%3
  \item 61\%-70\%3,5
  \item 71\%-80\%4
  \item 81\%-90\%4,5
  \item 91\%-100\%5
\end{itemize}

\section{Metody dydaktyczne}

Wykład, laboratoria, praca własna studenta.

\section{Literatura}

\textbf{Podstawowa:}
\begin{itemize}
  \item Géron, Aurélien. Hands-on machine learning with Scikit-Learn, Keras, and TensorFlow. " O'Reilly Media, Inc.", 2022.  Rozdział 14, 17
\end{itemize}

\textbf{Uzupełniająca:}
\begin{itemize}
  \item Foley, James D., et al. "Computer graphics: Principles and practice, in c." Color Research and Application 22.1 (1997): 65-65. Rozdział 5.
\end{itemize}

\end{document}
