% ===========================================================
%  Sylabus: Programowanie w języku C++ (CPP)
% ===========================================================
\documentclass[12pt, a4paper]{article}

\usepackage[T1]{fontenc}
\usepackage[utf8]{inputenc}
\usepackage[polish]{babel}
\usepackage{lmodern}
\usepackage{microtype}
\usepackage[a4paper, top=2.5cm, bottom=2.5cm, left=2.5cm, right=2.5cm]{geometry}
\usepackage{xcolor}
\usepackage{graphicx}
\usepackage{booktabs}
\usepackage{tabularx}
\usepackage{longtable}
\usepackage{multirow}
\usepackage{array}
\usepackage{colortbl}
\usepackage{enumitem}
\usepackage{fancyhdr}
\usepackage{titlesec}
\usepackage{mdframed}
\usepackage[colorlinks=true, linkcolor=red!70!black, urlcolor=red!70!black]{hyperref}
\usepackage{eso-pic}
\usepackage{tikz}

\definecolor{pjatkRed}{RGB}{180,0,0}
\definecolor{pjatkGray}{RGB}{80,80,80}
\definecolor{pjatkLightGray}{RGB}{245,245,245}
\definecolor{tableHeader}{RGB}{220,220,220}

\pagestyle{fancy}\fancyhf{}
\renewcommand{\headrulewidth}{0.4pt}
\renewcommand{\footrulewidth}{0.4pt}
\fancyhead[L]{\small\textcolor{pjatkGray}{PJATK -- Filia w Gdańsku \textbar\ Informatyka}}
\fancyhead[R]{\small\textcolor{pjatkGray}{Sylabus: CPP}}
\fancyfoot[C]{\small\thepage}

\titleformat{\section}{\large\bfseries\color{pjatkRed}}{\thesection.}{0.5em}{}
  [\color{pjatkRed}\rule{\linewidth}{0.8pt}]
\setlist{noitemsep, topsep=3pt, parsep=2pt}

\newmdenv[linecolor=pjatkRed, linewidth=1.2pt, backgroundcolor=pjatkLightGray,
  innerleftmargin=10pt, innerrightmargin=10pt, innertopmargin=8pt,
  innerbottommargin=8pt, roundcorner=4pt]{infobox}

\begin{document}

\AddToShipoutPictureBG{%
  \begin{tikzpicture}[remember picture, overlay]
    \node[opacity=0.5] at (current page.center) {%
      \includegraphics[width=14cm]{C:/Users/adamu/WebstormProjects/pj-studies/latex/PJATK_pl_sygnet_transparent-eps-converted-to}%
    };
  \end{tikzpicture}%
}

\begin{center}
  \includegraphics[height=2cm]{C:/Users/adamu/WebstormProjects/pj-studies/latex/PJATK_pl_poziom_1}\\[0.8cm]
  {\LARGE\bfseries\color{pjatkRed} SYLABUS PRZEDMIOTU}\\[0.8cm]
\end{center}

\begin{infobox}
\begin{tabularx}{\textwidth}{@{}lX@{}}
  \textbf{Nazwa przedmiotu:}  & {\bfseries Programowanie w języku C++} \\[3pt]
  \textbf{Kod przedmiotu:}    & CPP \\[3pt]
  \textbf{Kierunek / Profil:} & Informatyka / praktyczny \\[3pt]
  \textbf{Tryb studiów:}      & stacjonarny \\[3pt]
  \textbf{Rok / Semestr:}     & 1 / 2 \\[3pt]
  \textbf{Charakter:}         & obieralny \\[3pt]
  \textbf{Odpowiedzialny:}    & do ustalenia \\[3pt]
  \textbf{Wersja z dnia:}     & 20.02.2026 \\
\end{tabularx}
\end{infobox}

\vspace{1cm}

\section{Godziny zajęć i punkty ECTS}

\begin{center}
\begin{tabular}{|>{\centering\arraybackslash}p{2.0cm}
                |>{\centering\arraybackslash}p{2.0cm}
                |>{\centering\arraybackslash}p{2.0cm}
                |>{\centering\arraybackslash}p{2.4cm}
                |>{\centering\arraybackslash}p{2.4cm}
                |>{\centering\arraybackslash}p{2.0cm}
                |>{\centering\arraybackslash}p{1.4cm}|}
\hline
\rowcolor{tableHeader}
\textbf{Wykłady} & \textbf{Ćwiczenia} & \textbf{Laboratorium} &
\textbf{Z prowadzącym} & \textbf{Praca własna} & \textbf{Łącznie} & \textbf{ECTS} \\
\hline
15 h & --- & 15 h & 30 h & 20 h & 50 h & \textbf{2} \\
\hline
\end{tabular}
\end{center}

\section{Forma zajęć}

\begin{tabular}{ll}
  \hline
  \textbf{Forma zajęć} & \textbf{Sposób zaliczenia} \\
  \hline
  Projekt & Zaliczenie z oceną \\
  \hline
\end{tabular}

\section{Cel dydaktyczny}

Celem przedmiotu jest zapoznanie studentów z podstawami programowania w języku C++ oraz wykształcenie umiejętności tworzenia prostych aplikacji konsolowych z wykorzystaniem klasycznych konstrukcji programistycznych. Zajęcia kładą nacisk na zrozumienie modelu pamięci, poprawne użycie typów danych, funkcji i struktur, a także wprowadzenie do podstaw programowania obiektowego. Przedmiot przygotowuje studentów do dalszej nauki języków kompilowanych oraz pracy nad projektami wymagającymi większej kontroli nad wydajnością i zasobami.

\section{Treści programowe}

\begin{enumerate}
  \item Wprowadzenie do języka C++: zastosowania, kompilacja, struktura programu, pierwszy program.
  \item Typy danych, zmienne, stałe, operatory, instrukcje wejścia i wyjścia (cin, cout).
  \item Instrukcje sterujące: if, switch, pętle for, while, do while.
  \item Funkcje: definicja, parametry, wartości zwracane, przeciążanie funkcji.
  \item Tablice i łańcuchy znaków, podstawy pracy z pamięcią.
  \item Wskaźniki i referencje – wprowadzenie i zastosowania.
  \item Struktury i typy złożone.
  \item Dynamiczna alokacja pamięci (new, delete) i podstawy bezpiecznego zarządzania zasobami.
  \item Wprowadzenie do programowania obiektowego: klasy i obiekty.
  \item  Konstruktory, destruktory, enkapsulacja.
  \item  Dziedziczenie i polimorfizm – podstawy.
  \item  Biblioteka standardowa STL – wprowadzenie (vector, string).
\end{enumerate}

\section{Efekty kształcenia}

\subsection*{Wiedza}
\begin{itemize}
  \item Student zna podstawowe konstrukcje języka C++, typy danych, instrukcje sterujące oraz zasady działania kompilatora i linkera. Rozumie podstawy zarządzania pamięcią i różnice między typami wartościowymi i referencyjnymi.
\end{itemize}

\subsection*{Umiejętności}
\begin{itemize}
  \item Student potrafi tworzyć, kompilować i debugować proste programy w języku C++, korzystać z dokumentacji technicznej, dzielić program na funkcje i pliki źródłowe oraz zaimplementować niewielki projekt programistyczny rozwiązujący praktyczny problem.
\end{itemize}

\subsection*{Kompetencje społeczne}
\begin{itemize}
  \item Student potrafi samodzielnie planować pracę nad projektem programistycznym, dbać o jakość i czytelność kodu oraz terminowo realizować powierzone zadania.
\end{itemize}

\section{Kryteria oceny}

\begin{itemize}
  \item Projekt programistyczny realizowany indywidualnie lub w parach – 100\%
  \item Ocena projektu obejmuje: poprawność działania programu, jakość i czytelność kodu, strukturę rozwiązania, zarządzanie pamięcią oraz dokumentację
  \item Warunkiem zaliczenia jest oddanie kompletnego projektu oraz jego pozytywna ocena
\end{itemize}

\section{Metody dydaktyczne}

Wykład, laboratoria, praca własna studenta.

\section{Literatura}

\textbf{Podstawowa:}
\begin{itemize}
  \item Bjarne Stroustrup, Programming: Principles and Practice Using C++
  \item Stephen Prata, C++ Primer Plus
  \item Dokumentacja C++: https://en.cppreference.com
\end{itemize}

\textbf{Uzupełniająca:}
\begin{itemize}
  \item Scott Meyers, Effective C++
  \item Herb Sutter, Exceptional C++
\end{itemize}

\end{document}
