% ===========================================================
%  Sylabus: Technologie backendowe (TBK)
% ===========================================================
\documentclass[12pt, a4paper]{article}

\usepackage[T1]{fontenc}
\usepackage[utf8]{inputenc}
\usepackage[polish]{babel}
\usepackage{lmodern}
\usepackage{microtype}
\usepackage[a4paper, top=2.5cm, bottom=2.5cm, left=2.5cm, right=2.5cm]{geometry}
\usepackage{xcolor}
\usepackage{graphicx}
\usepackage{booktabs}
\usepackage{tabularx}
\usepackage{longtable}
\usepackage{multirow}
\usepackage{array}
\usepackage{colortbl}
\usepackage{enumitem}
\usepackage{fancyhdr}
\usepackage{titlesec}
\usepackage{mdframed}
\usepackage[colorlinks=true, linkcolor=red!70!black, urlcolor=red!70!black]{hyperref}
\usepackage{eso-pic}
\usepackage{tikz}

\definecolor{pjatkRed}{RGB}{180,0,0}
\definecolor{pjatkGray}{RGB}{80,80,80}
\definecolor{pjatkLightGray}{RGB}{245,245,245}
\definecolor{tableHeader}{RGB}{220,220,220}

\pagestyle{fancy}\fancyhf{}
\renewcommand{\headrulewidth}{0.4pt}
\renewcommand{\footrulewidth}{0.4pt}
\fancyhead[L]{\small\textcolor{pjatkGray}{PJATK -- Filia w Gdańsku \textbar\ Informatyka}}
\fancyhead[R]{\small\textcolor{pjatkGray}{Sylabus: TBK}}
\fancyfoot[C]{\small\thepage}

\titleformat{\section}{\large\bfseries\color{pjatkRed}}{\thesection.}{0.5em}{}
  [\color{pjatkRed}\rule{\linewidth}{0.8pt}]
\setlist{noitemsep, topsep=3pt, parsep=2pt}

\newmdenv[linecolor=pjatkRed, linewidth=1.2pt, backgroundcolor=pjatkLightGray,
  innerleftmargin=10pt, innerrightmargin=10pt, innertopmargin=8pt,
  innerbottommargin=8pt, roundcorner=4pt]{infobox}

\begin{document}

\AddToShipoutPictureBG{%
  \begin{tikzpicture}[remember picture, overlay]
    \node[opacity=0.5] at (current page.center) {%
      \includegraphics[width=14cm]{C:/Users/adamu/WebstormProjects/pj-studies/latex/PJATK_pl_sygnet_transparent-eps-converted-to}%
    };
  \end{tikzpicture}%
}

\begin{center}
  \includegraphics[height=2cm]{C:/Users/adamu/WebstormProjects/pj-studies/latex/PJATK_pl_poziom_1}\\[0.8cm]
  {\LARGE\bfseries\color{pjatkRed} SYLABUS PRZEDMIOTU}\\[0.8cm]
\end{center}

\begin{infobox}
\begin{tabularx}{\textwidth}{@{}lX@{}}
  \textbf{Nazwa przedmiotu:}  & {\bfseries Technologie backendowe} \\[3pt]
  \textbf{Kod przedmiotu:}    & TBK \\[3pt]
  \textbf{Kierunek / Profil:} & Informatyka / praktyczny \\[3pt]
  \textbf{Tryb studiów:}      & stacjonarny \\[3pt]
  \textbf{Rok / Semestr:}     & 3 / 5 \\[3pt]
  \textbf{Charakter:}         & obieralny \\[3pt]
  \textbf{Odpowiedzialny:}    & mgr Mateusz Miotk \\[3pt]
  \textbf{Wersja z dnia:}     & 19.02.2026 \\
\end{tabularx}
\end{infobox}

\vspace{1cm}

\section{Godziny zajęć i punkty ECTS}

\begin{center}
\begin{tabular}{|>{\centering\arraybackslash}p{2.0cm}
                |>{\centering\arraybackslash}p{2.0cm}
                |>{\centering\arraybackslash}p{2.0cm}
                |>{\centering\arraybackslash}p{2.4cm}
                |>{\centering\arraybackslash}p{2.4cm}
                |>{\centering\arraybackslash}p{2.0cm}
                |>{\centering\arraybackslash}p{1.4cm}|}
\hline
\rowcolor{tableHeader}
\textbf{Wykłady} & \textbf{Ćwiczenia} & \textbf{Laboratorium} &
\textbf{Z prowadzącym} & \textbf{Praca własna} & \textbf{Łącznie} & \textbf{ECTS} \\
\hline
30 h & 30 h & --- & 60 h & 40 h & 100 h & \textbf{4} \\
\hline
\end{tabular}
\end{center}

\section{Forma zajęć}

\begin{tabular}{ll}
  \hline
  \textbf{Forma zajęć} & \textbf{Sposób zaliczenia} \\
  \hline
  Laboratorium & Zaliczenie z oceną \\
  Wykład & Nieoceniany \\
  \hline
\end{tabular}

\section{Cel dydaktyczny}

Przedmiot "Technologie Backendowe" został stworzony z myślą o rozwijaniu umiejętności z zakresu zaawansowanego programowania w języku JavaScript, z wykorzystaniem Node.js i przykładowego narzędzia do tworzenia środowiska po stronie serwera za pomocą protokołu HTTP, w celu tworzenia efektywnych i skalowalnych aplikacji internetowych. Program nauczania skupia się na głębokim zrozumieniu protokołu HTTP i wzorca REST API, co pozwala na efektywne projektowanie i implementację interfejsów API, odpowiadających za komunikację między frontendem a backendem.

\section{Przedmioty wprowadzające}

\begin{tabularx}{\textwidth}{lX}
  \hline
  \textbf{Przedmiot} & \textbf{Wymagane zagadnienia} \\
  \hline
  Użytkowanie komputerów & Systemy operacyjne \\
  Technologie internetu & Umiejętność posługiwania się emulatorem terminala w systemie operacyjnym \\
  Znajomość struktur plików i katalogów w systemie operacyjnym & Znajomość pojęcia procesu w systemie operacyjnym \\
  Znajomość protokołu HTTP & --- \\
  \hline
\end{tabularx}

\section{Treści programowe}

\begin{enumerate}
  \item Historia techologii backendowych. Język JavaScript.
  \item Budowa i działanie serwera Node.js. Narzędzie npm
  \item Menadżer pakietów oraz modułów w Node.js
  \item Asynchroniczność na podstawie działania serwera Node.js
  \item Protokół HTTP/HTTPS. Tworzenie aplikacji webowej na bazie tego protokołu.
  \item Pojęcie szablonu w aplikacji backendowej (handlebars). Obsługa formularzy
  \item Ciasteczka oraz sesje. Pojęcie Middleware.
  \item Wprowadzenie do dokumentowej bazy danych
  \item Operacje CRUD w dokumentowej bazie dancyh
  \item Funkcje agregujące w dokumentowej bazie dancyh
  \item Pojęcie indeksowania w dokumentowej bazie dancyh
  \item Replikacja danych w dokumentowej bazie dancyh
  \item Protokół HTTP. Wzorzec REST API w JavaScript
  \item Tworzenie aplikacji typu REST API z wykorzystania JavaScript oraz dokumentowej bazie danych
  \item Pojęcie bezpieczeństwa w aplikacjach typu REST API
  \item Testowanie aplikacji typu REST API
\end{enumerate}

\section{Efekty kształcenia}

\subsection*{Wiedza}
\begin{itemize}
  \item Student potrafi wyjaśnić, jak są zbudowane współczesne aplikacje internetowe, włączając w to rolę i funkcje backendu.
  \item Student demonstruje głębokie zrozumienie składni, struktur i możliwości języka JavaScript w kontekście programowania backendowego.
  \item Student potrafi wyjaśnić, jak działają żądania i odpowiedzi HTTP, a także zidentyfikować i opisać różne metody, kody odpowiedzi i nagłówki.
  \item Student umie zdefiniować i opisać wzorzec REST API, jego zasady i konwencje, a także zastosowania w budowie aplikacji internetowych.
  \item Student rozumie, jak działa nierelacyjna baza danych, potrafi zidentyfikować główne cechy bazy dokumentowej i zastosować je w praktyce.
  \item Student rozumie kluczowe zagadnienia związane ze skalowalnością i wydajnością aplikacji internetowych, oraz umie zastosować praktyki optymalizacji.
  \item Student rozumie zaawansowane pojęcia w zakresie technologii backendowych.
\end{itemize}

\subsection*{Umiejętności}
\begin{itemize}
  \item Student umie zaprojektować i zaimplementować funkcjonalny i optymalizowany backend aplikacji internetowej, wykorzystując zaawansowane techniki programowania.
  \item Student potrafi wykorzystać protokół HTTP do konstrukcji i obsługi żądań i odpowiedzi w kontekście aplikacji internetowej.
  \item Student umie tworzyć złożone REST API, uwzględniając dobre praktyki i zasady projektowania interfejsów API.
  \item Student potrafi wykonywać operacje CRUD na dokumentowej bazie danych, z wykorzystaniem odpowiednich metod i narzędzi.
  \item Student potrafi stworzyć i przetestować backend aplikacji internetowej za pomocą poznanych narzędzi.
\end{itemize}

\subsection*{Kompetencje społeczne}
\begin{itemize}
  \item Student jest gotów do jest gotów do samodzielnego uczenia się przez całe życie
\end{itemize}

\section{Kryteria oceny}

\begin{itemize}
  \item wykład z elementami dyskusji z prezentacją multimedialną
  \item burza mózgów
  \item rozwiązywanie zadań
  \item analiza przypadków
  \item projekt praktyczny
  \item Kryteria oceny
  \item 60\% Kolokwium
  \item 40\% Projekt programistyczny
\end{itemize}

\section{Metody dydaktyczne}

Wykład, laboratoria, praca własna studenta.

\section{Literatura}

\textbf{Podstawowa:}
\begin{itemize}
  \item E. Brown, Web Development with Node and Express, 2nd Edition, O’Reilly Media, 2019.
  \item S. Brandshaw, E. Brazil, K. Chodorow, MongoDB: The Definitive Guilde, O’Reilly Media, 2020.
\end{itemize}

\textbf{Uzupełniająca:}
\begin{itemize}
  \item F, Dogilo, REST API Development with Node.js, Apress, 2018.
  \item D. Herron, Node.js Web Development, Fifth Edition, Apress, 2020.
  \item A. Mend, Learning Node.js Development, Packt Publishing, 2018.
  \item F. Zammetti, Modern Full Stack Development using Typescript, React, Node.js, Webpack and Docker, Apress, 2020.
\end{itemize}

\end{document}
