% ===========================================================
%  Sylabus: Programowanie grafiki 3D w OpenGL (OGL)
% ===========================================================
\documentclass[12pt, a4paper]{article}

\usepackage[T1]{fontenc}
\usepackage[utf8]{inputenc}
\usepackage[polish]{babel}
\usepackage{lmodern}
\usepackage{microtype}
\usepackage[a4paper, top=2.5cm, bottom=2.5cm, left=2.5cm, right=2.5cm]{geometry}
\usepackage{xcolor}
\usepackage{graphicx}
\usepackage{booktabs}
\usepackage{tabularx}
\usepackage{longtable}
\usepackage{multirow}
\usepackage{array}
\usepackage{colortbl}
\usepackage{enumitem}
\usepackage{fancyhdr}
\usepackage{titlesec}
\usepackage{mdframed}
\usepackage[colorlinks=true, linkcolor=red!70!black, urlcolor=red!70!black]{hyperref}
\usepackage{eso-pic}
\usepackage{tikz}

\definecolor{pjatkRed}{RGB}{180,0,0}
\definecolor{pjatkGray}{RGB}{80,80,80}
\definecolor{pjatkLightGray}{RGB}{245,245,245}
\definecolor{tableHeader}{RGB}{220,220,220}

\pagestyle{fancy}\fancyhf{}
\renewcommand{\headrulewidth}{0.4pt}
\renewcommand{\footrulewidth}{0.4pt}
\fancyhead[L]{\small\textcolor{pjatkGray}{PJATK -- Filia w Gdańsku \textbar\ Informatyka}}
\fancyhead[R]{\small\textcolor{pjatkGray}{Sylabus: OGL}}
\fancyfoot[C]{\small\thepage}

\titleformat{\section}{\large\bfseries\color{pjatkRed}}{\thesection.}{0.5em}{}
  [\color{pjatkRed}\rule{\linewidth}{0.8pt}]
\setlist{noitemsep, topsep=3pt, parsep=2pt}

\newmdenv[linecolor=pjatkRed, linewidth=1.2pt, backgroundcolor=pjatkLightGray,
  innerleftmargin=10pt, innerrightmargin=10pt, innertopmargin=8pt,
  innerbottommargin=8pt, roundcorner=4pt]{infobox}

\begin{document}

\AddToShipoutPictureBG{%
  \begin{tikzpicture}[remember picture, overlay]
    \node[opacity=0.5] at (current page.center) {%
      \includegraphics[width=14cm]{C:/Users/adamu/WebstormProjects/pj-studies/latex/PJATK_pl_sygnet_transparent-eps-converted-to}%
    };
  \end{tikzpicture}%
}

\begin{center}
  \includegraphics[height=2cm]{C:/Users/adamu/WebstormProjects/pj-studies/latex/PJATK_pl_poziom_1}\\[0.8cm]
  {\LARGE\bfseries\color{pjatkRed} SYLABUS PRZEDMIOTU}\\[0.8cm]
\end{center}

\begin{infobox}
\begin{tabularx}{\textwidth}{@{}lX@{}}
  \textbf{Nazwa przedmiotu:}  & {\bfseries Programowanie grafiki 3D w OpenGL} \\[3pt]
  \textbf{Kod przedmiotu:}    & OGL \\[3pt]
  \textbf{Kierunek / Profil:} & Informatyka / praktyczny \\[3pt]
  \textbf{Tryb studiów:}      & stacjonarny \\[3pt]
  \textbf{Rok / Semestr:}     & 3 / 5 \\[3pt]
  \textbf{Charakter:}         & obieralny \\[3pt]
  \textbf{Odpowiedzialny:}    & Liczba punktów ECTS : 4 \\[3pt]
  \textbf{Wersja z dnia:}     & 19.02.2026 \\
\end{tabularx}
\end{infobox}

\vspace{1cm}

\section{Godziny zajęć i punkty ECTS}

\begin{center}
\begin{tabular}{|>{\centering\arraybackslash}p{2.0cm}
                |>{\centering\arraybackslash}p{2.0cm}
                |>{\centering\arraybackslash}p{2.0cm}
                |>{\centering\arraybackslash}p{2.4cm}
                |>{\centering\arraybackslash}p{2.4cm}
                |>{\centering\arraybackslash}p{2.0cm}
                |>{\centering\arraybackslash}p{1.4cm}|}
\hline
\rowcolor{tableHeader}
\textbf{Wykłady} & \textbf{Ćwiczenia} & \textbf{Laboratorium} &
\textbf{Z prowadzącym} & \textbf{Praca własna} & \textbf{Łącznie} & \textbf{ECTS} \\
\hline
30 h & 30 h & --- & 60 h & 40 h & 100 h & \textbf{4} \\
\hline
\end{tabular}
\end{center}

\section{Forma zajęć}

\begin{tabular}{ll}
  \hline
  \textbf{Forma zajęć} & \textbf{Sposób zaliczenia} \\
  \hline
  Laboratorium & Zaliczenie z oceną \\
  Wykład & Nieoceniany \\
  \hline
\end{tabular}

\section{Cel dydaktyczny}

Celem kształcenia jest nabycie umiejętności wykorzystanie zaawansowanych możliwości biblioteki OpenGL

\section{Przedmioty wprowadzające}

\begin{tabularx}{\textwidth}{lX}
  \hline
  \textbf{Przedmiot} & \textbf{Wymagane zagadnienia} \\
  \hline
  • Algebra liniowa i geometria  (ALG) & • Analiza matematyczna  (AM) \\
  • Algorytmy i struktury danych (ASD) & • Grafika komputerowa (GRK) \\
  • programowanie w C/C++ & • operacje na macierzach \\
  • przekształcenia liniowe & • podstawy OpenGL \\
  \hline
\end{tabularx}

\section{Treści programowe}

\begin{enumerate}
  \item Wykład
  \item Modelowanie oświetenia
  \item Obiekt buforu ramki
  \item Shadery geometrii
  \item Teselacja
  \item Import modeli 3W w OpenGL
  \item Modelowanie nieba (Skybox)
  \item Modelowanie mgły
  \item Modelowanie przezroczystości. Sprajty punktowe
  \item Sprajty punktowe
  \item Modelowanie nierówności
  \item Shadery obliczeniowe
  \item Modelowanie systemu cząstek
  \item Podstawy API Vulkan
  \item Bufory w API Vulkan
  \item Teksturowanie w API Vulkan
  \item Laboratorium
  \item Modelowanie oświetenia
  \item Mapowanie cienia
  \item Modelowanie krzywych Béziera
  \item Modelowanie płatów Béziera
  \item Import modeli 3W w OpenGL
  \item Modelowanie nieba (Skybox)
  \item Modelowanie mgły
  \item Modelowanie przezroczystości. Sprajty punktowe
  \item Sprajty punktowe
  \item Modelowanie nierówności
  \item Modelowanie głębi ostrości
  \item Modelowanie systemu cząstek
  \item Podstawy API Vulkan
  \item Bufory w API Vulkan
  \item Teksturowanie w API Vulkan
\end{enumerate}

\section{Efekty kształcenia}

\subsection*{Wiedza}
\begin{itemize}
  \item Student zna i rozumie pojęcia z zakresu kluczowych zagadnień i metod w zakresie grafiki, multimediów i komunikacji człowiek-komputer
  \item Student zna i rozumie zaawansowane pojęcia w zakresie programowania grafiki 3W, stosowanych aktualnie narzędzi i technologii
\end{itemize}

\subsection*{Umiejętności}
\begin{itemize}
  \item Student potrafi zaprojektować i zaprogramować zaawansowaną aplikację 3W z wykorzystaniem współczesnych narzędzi i technologii
\end{itemize}

\subsection*{Kompetencje społeczne}
\begin{itemize}
  \item Student jest gotów do samodzielnego uczenia się przez całe życie
\end{itemize}

\section{Kryteria oceny}

\begin{itemize}
  \item rozwiązywanie zadań
  \item Kryteria oceny
  \item Laboratorium/
  \item Zaliczenie ćwiczeń polega na zbieraniu punktów:
  \item 50\% możliwych punktów daje ocenę 3
  \item 60\% punktów daje ocenę 3½
  \item 70\% — ocenę 4
  \item 80\% — 4½
  \item 90\% i więcej — 5
  \item brak
\end{itemize}

\section{Metody dydaktyczne}

Wykład, laboratoria, praca własna studenta.

\section{Literatura}

\textbf{Podstawowa:}
\begin{itemize}
  \item S. R. Buss: 3D Computer Graphics: A Mathematical Introduction with OpenGL, Revision draft Draft A.10.b. May 28, 2019.
  \item The Khronos Group: OpenGL API Documentation Overview 2024
\end{itemize}

\textbf{Uzupełniająca:}
\begin{itemize}
  \item Graham Sellers, Vulkan Programming Guide, Pearson Education, 2016
  \item Graham Sellers, Richard S. Wright Jr., Nicholas HaemelOpenGL. Księga eksperta. Wydanie VII Helion2016
\end{itemize}

\end{document}
