% ===========================================================
%  Sylabus: Technologie internetu (TIN)
% ===========================================================
\documentclass[12pt, a4paper]{article}

\usepackage[T1]{fontenc}
\usepackage[utf8]{inputenc}
\usepackage[polish]{babel}
\usepackage{lmodern}
\usepackage{microtype}
\usepackage[a4paper, top=2.5cm, bottom=2.5cm, left=2.5cm, right=2.5cm]{geometry}
\usepackage{xcolor}
\usepackage{graphicx}
\usepackage{booktabs}
\usepackage{tabularx}
\usepackage{longtable}
\usepackage{multirow}
\usepackage{array}
\usepackage{colortbl}
\usepackage{enumitem}
\usepackage{fancyhdr}
\usepackage{titlesec}
\usepackage{mdframed}
\usepackage[colorlinks=true, linkcolor=red!70!black, urlcolor=red!70!black]{hyperref}
\usepackage{eso-pic}
\usepackage{tikz}

\definecolor{pjatkRed}{RGB}{180,0,0}
\definecolor{pjatkGray}{RGB}{80,80,80}
\definecolor{pjatkLightGray}{RGB}{245,245,245}
\definecolor{tableHeader}{RGB}{220,220,220}

\pagestyle{fancy}\fancyhf{}
\renewcommand{\headrulewidth}{0.4pt}
\renewcommand{\footrulewidth}{0.4pt}
\fancyhead[L]{\small\textcolor{pjatkGray}{PJATK -- Filia w Gdańsku \textbar\ Informatyka}}
\fancyhead[R]{\small\textcolor{pjatkGray}{Sylabus: TIN}}
\fancyfoot[C]{\small\thepage}

\titleformat{\section}{\large\bfseries\color{pjatkRed}}{\thesection.}{0.5em}{}
  [\color{pjatkRed}\rule{\linewidth}{0.8pt}]
\setlist{noitemsep, topsep=3pt, parsep=2pt}

\newmdenv[linecolor=pjatkRed, linewidth=1.2pt, backgroundcolor=pjatkLightGray,
  innerleftmargin=10pt, innerrightmargin=10pt, innertopmargin=8pt,
  innerbottommargin=8pt, roundcorner=4pt]{infobox}

\begin{document}

\AddToShipoutPictureBG{%
  \begin{tikzpicture}[remember picture, overlay]
    \node[opacity=0.5] at (current page.center) {%
      \includegraphics[width=14cm]{C:/Users/adamu/WebstormProjects/pj-studies/latex/PJATK_pl_sygnet_transparent-eps-converted-to}%
    };
  \end{tikzpicture}%
}

\begin{center}
  \includegraphics[height=2cm]{C:/Users/adamu/WebstormProjects/pj-studies/latex/PJATK_pl_poziom_1}\\[0.8cm]
  {\LARGE\bfseries\color{pjatkRed} SYLABUS PRZEDMIOTU}\\[0.8cm]
\end{center}

\begin{infobox}
\begin{tabularx}{\textwidth}{@{}lX@{}}
  \textbf{Nazwa przedmiotu:}  & {\bfseries Technologie internetu} \\[3pt]
  \textbf{Kod przedmiotu:}    & TIN \\[3pt]
  \textbf{Kierunek / Profil:} & Informatyka / praktyczny \\[3pt]
  \textbf{Tryb studiów:}      & niestacjonarny \\[3pt]
  \textbf{Rok / Semestr:}     & 1 / 2 \\[3pt]
  \textbf{Charakter:}         & obowiązkowy \\[3pt]
  \textbf{Odpowiedzialny:}    & dr Tadeusz Puźniakowski \\[3pt]
  \textbf{Wersja z dnia:}     & 19.02.2026 \\
\end{tabularx}
\end{infobox}

\vspace{1cm}

\section{Godziny zajęć i punkty ECTS}

\begin{center}
\begin{tabular}{|>{\centering\arraybackslash}p{2.0cm}
                |>{\centering\arraybackslash}p{2.0cm}
                |>{\centering\arraybackslash}p{2.0cm}
                |>{\centering\arraybackslash}p{2.4cm}
                |>{\centering\arraybackslash}p{2.4cm}
                |>{\centering\arraybackslash}p{2.0cm}
                |>{\centering\arraybackslash}p{1.4cm}|}
\hline
\rowcolor{tableHeader}
\textbf{Wykłady} & \textbf{Ćwiczenia} & \textbf{Laboratorium} &
\textbf{Z prowadzącym} & \textbf{Praca własna} & \textbf{Łącznie} & \textbf{ECTS} \\
\hline
16 h & --- & 16 h & 32 h & 68 h & 100 h & \textbf{4} \\
\hline
\end{tabular}
\end{center}

\section{Forma zajęć}

\begin{tabular}{ll}
  \hline
  \textbf{Forma zajęć} & \textbf{Sposób zaliczenia} \\
  \hline
  Laboratorium & Zaliczenie z oceną \\
  \hline
\end{tabular}

\section{Cel dydaktyczny}

Zapoznanie studentów z nowoczesnymi technologiami internetowymi, które są kluczowe dla tworzenia i zarządzania dynamicznymi stronami oraz aplikacjami internetowymi. Zdobycie wiedzy na temat tworzenia responsywnych interfejsów, wykorzystania zaawansowanych technik CSS oraz programowania w języku JavaScript, w tym programowania funkcyjnego i obiektowego. Przedmiot obejmuje także tematykę asynchronicznego programowania, modyfikacji elementów DOM oraz korzystania z narzędzi sieciowych, takich jak HTTP i WebSocket. Dodatkowo, studenci poznają podstawy pracy z popularnym narzędziem do budowy aplikacji serwerowych.

\section{Przedmioty wprowadzające}

\begin{tabularx}{\textwidth}{lX}
  \hline
  \textbf{Przedmiot} & \textbf{Wymagane zagadnienia} \\
  \hline
  Warsztaty programistyczne & Bazy danych \\
  Znajomość HTML, obsługi bazy danych za pomocą języka zapytań & --- \\
  \hline
\end{tabularx}

\section{Treści programowe}

\begin{enumerate}
  \item Przegląd technologii internetowych i ich rola w tworzeniu nowoczesnych stron internetowych.
  \item Zasady responsywności stron internetowych i ich implementacja.
  \item Techniki zaawansowanego stylowania CSS oraz organizacja i optymalizacja kodu.
  \item Zastosowanie narzędzi do tworzenia układów i stylizacji stron internetowych.
  \item Wprowadzenie do języka JavaScript: zmienne oraz operatory.
  \item Funkcje i paradygmat programowania funkcyjnego w języku JavaScript.
  \item Programowanie obiektowe w JavaScript oraz manipulacja obiektami.
  \item Zarządzanie zdarzeniami w JavaScript oraz modyfikacja struktury DOM.
  \item Asynchroniczne programowanie w JavaScript: podstawy i praktyczne zastosowania.
  \item Zaawansowane techniki programowania w JavaScript: rozprzestrzenianie i destrukturyzacja.
  \item Wprowadzenie do mechanizmów obsługi zdarzeń w aplikacjach internetowych.
  \item Obsługa komunikacji sieciowej w JavaScript, w tym zapytania HTTP i mechanizmy komunikacji w czasie rzeczywistym.
  \item Podstawy pracy z technologiami serwerowymi oraz budowa aplikacji internetowych.
  \item Wprowadzenie do tworzenia aplikacji internetowych z wykorzystaniem popularnych narzędzi serwerowych.
\end{enumerate}

\section{Efekty kształcenia}

\subsection*{Wiedza}
\begin{itemize}
  \item Student rozumie podstawowe technologie internetowe oraz ich rolę w tworzeniu nowoczesnych stron i aplikacji internetowych.
  \item Student potrafi wyjaśnić zasady responsywności stron internetowych oraz zastosować odpowiednie techniki do ich implementacji.
  \item Student rozumie działanie preprocesorów CSS i potrafi wykorzystać je do optymalizacji oraz organizacji kodu CSS.
  \item Student rozumie zasady działania języka JavaScript, w tym zmienne, operatory oraz funkcje, i potrafi zastosować je w praktycznych zadaniach.
  \item Student potrafi wyjaśnić zasady asynchronicznego programowania w JavaScript i zastosować je w praktyce.
  \item Student rozumie podstawy działania narzędzi serwerowych, takich jak Node.js i potrafi je zastosować w kontekście budowy aplikacji internetowych.
\end{itemize}

\subsection*{Umiejętności}
\begin{itemize}
  \item Student umie zastosować frameworki CSS w celu tworzenia złożonych i estetycznych interfejsów użytkownika.
  \item Student potrafi wyjaśnić pojęcia programowania funkcyjnego oraz obiektowego w JavaScript i zastosować je w kodzie.
  \item Student umie pisać i optymalizować kod w języku JavaScript, stosując zmienne, operatory oraz funkcje.
  \item Student umie wykorzystać narzędzia sieciowe w JavaScript, w tym obsługę protokołów HTTP i WebSocket, do budowy dynamicznych aplikacji internetowych.
\end{itemize}

\subsection*{Kompetencje społeczne}
\begin{itemize}
  \item Student jest gotów do efektywnej współpracy w zespole projektowym, wnosząc wkład w tworzenie oraz rozwój aplikacji internetowych.
  \item Student jest gotów do krytycznego oceniania własnych umiejętności i poszerzania wiedzy w zakresie nowoczesnych technologii internetowych, podejmując inicjatywy samokształcenia.
  \item Student jest gotów do rozwiązywania problemów programistycznych w sposób kreatywny i efektywny, korzystając z nowoczesnych narzędzi i technologii.
  \item Student jest gotów do stosowania najlepszych praktyk programistycznych oraz standardów branżowych przy tworzeniu rozwiązań internetowych, dbając o ich jakość i bezpieczeństwo.
\end{itemize}

\section{Kryteria oceny}

\begin{itemize}
  \item wykład z elementami dyskusji z prezentacją multimedialną
  \item burza mózgów
  \item rozwiązywanie zadań
  \item analiza przypadków
  \item projekt praktyczny
  \item Kryteria oceny
  \item 50\% Kolokwium pisemne
  \item 50\% Projekt praktyczny
\end{itemize}

\section{Metody dydaktyczne}

Wykład, laboratoria, praca własna studenta.

\section{Literatura}

\textbf{Podstawowa:}
\begin{itemize}
  \item D. Herron, Node.js Web Development, Fifth Edition, Apress, 2020.
  \item E. Brown, Web Development with Node and Express, 2nd Edition, O’Reilly Media, 2019.
\end{itemize}

\textbf{Uzupełniająca:}
\begin{itemize}
  \item F. Zammetti, Modern Full Stack Development using Typescript, React, Node.js, Webpack and Docker, Apress, 2020.
\end{itemize}

\end{document}
