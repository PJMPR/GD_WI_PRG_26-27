% ===========================================================
%  Sylabus: Modelowanie i analiza systemów informacyjnych (MAS)
% ===========================================================
\documentclass[12pt, a4paper]{article}

\usepackage[T1]{fontenc}
\usepackage[utf8]{inputenc}
\usepackage[polish]{babel}
\usepackage{lmodern}
\usepackage{microtype}
\usepackage[a4paper, top=2.5cm, bottom=2.5cm, left=2.5cm, right=2.5cm]{geometry}
\usepackage{xcolor}
\usepackage{graphicx}
\usepackage{booktabs}
\usepackage{tabularx}
\usepackage{longtable}
\usepackage{multirow}
\usepackage{array}
\usepackage{colortbl}
\usepackage{enumitem}
\usepackage{fancyhdr}
\usepackage{titlesec}
\usepackage{mdframed}
\usepackage[colorlinks=true, linkcolor=red!70!black, urlcolor=red!70!black]{hyperref}
\usepackage{eso-pic}
\usepackage{tikz}

\definecolor{pjatkRed}{RGB}{180,0,0}
\definecolor{pjatkGray}{RGB}{80,80,80}
\definecolor{pjatkLightGray}{RGB}{245,245,245}
\definecolor{tableHeader}{RGB}{220,220,220}

\pagestyle{fancy}\fancyhf{}
\renewcommand{\headrulewidth}{0.4pt}
\renewcommand{\footrulewidth}{0.4pt}
\fancyhead[L]{\small\textcolor{pjatkGray}{PJATK -- Filia w Gdańsku \textbar\ Informatyka}}
\fancyhead[R]{\small\textcolor{pjatkGray}{Sylabus: MAS}}
\fancyfoot[C]{\small\thepage}

\titleformat{\section}{\large\bfseries\color{pjatkRed}}{\thesection.}{0.5em}{}
  [\color{pjatkRed}\rule{\linewidth}{0.8pt}]
\setlist{noitemsep, topsep=3pt, parsep=2pt}

\newmdenv[linecolor=pjatkRed, linewidth=1.2pt, backgroundcolor=pjatkLightGray,
  innerleftmargin=10pt, innerrightmargin=10pt, innertopmargin=8pt,
  innerbottommargin=8pt, roundcorner=4pt]{infobox}

\begin{document}

\AddToShipoutPictureBG{%
  \begin{tikzpicture}[remember picture, overlay]
    \node[opacity=0.5] at (current page.center) {%
      \includegraphics[width=14cm]{C:/Users/adamu/WebstormProjects/pj-studies/latex/PJATK_pl_sygnet_transparent-eps-converted-to}%
    };
  \end{tikzpicture}%
}

\begin{center}
  \includegraphics[height=2cm]{C:/Users/adamu/WebstormProjects/pj-studies/latex/PJATK_pl_poziom_1}\\[0.8cm]
  {\LARGE\bfseries\color{pjatkRed} SYLABUS PRZEDMIOTU}\\[0.8cm]
\end{center}

\begin{infobox}
\begin{tabularx}{\textwidth}{@{}lX@{}}
  \textbf{Nazwa przedmiotu:}  & {\bfseries Modelowanie i analiza systemów informacyjnych} \\[3pt]
  \textbf{Kod przedmiotu:}    & MAS \\[3pt]
  \textbf{Kierunek / Profil:} & Informatyka / praktyczny \\[3pt]
  \textbf{Tryb studiów:}      & niestacjonarny \\[3pt]
  \textbf{Rok / Semestr:}     & 4 / 7 \\[3pt]
  \textbf{Charakter:}         & obieralny \\[3pt]
  \textbf{Odpowiedzialny:}    & Dr. Hab. Marek A Bednarczyk \\[3pt]
  \textbf{Wersja z dnia:}     & 19.02.2026 \\
\end{tabularx}
\end{infobox}

\vspace{1cm}

\section{Godziny zajęć i punkty ECTS}

\begin{center}
\begin{tabular}{|>{\centering\arraybackslash}p{2.0cm}
                |>{\centering\arraybackslash}p{2.0cm}
                |>{\centering\arraybackslash}p{2.0cm}
                |>{\centering\arraybackslash}p{2.4cm}
                |>{\centering\arraybackslash}p{2.4cm}
                |>{\centering\arraybackslash}p{2.0cm}
                |>{\centering\arraybackslash}p{1.4cm}|}
\hline
\rowcolor{tableHeader}
\textbf{Wykłady} & \textbf{Ćwiczenia} & \textbf{Laboratorium} &
\textbf{Z prowadzącym} & \textbf{Praca własna} & \textbf{Łącznie} & \textbf{ECTS} \\
\hline
16 h & --- & 16 h & 32 h & 93 h & 125 h & \textbf{5} \\
\hline
\end{tabular}
\end{center}

\section{Forma zajęć}

\begin{tabular}{ll}
  \hline
  \textbf{Forma zajęć} & \textbf{Sposób zaliczenia} \\
  \hline
  Laboratorium & Zaliczenie z oceną \\
  Wykład & Egzamin \\
  \hline
\end{tabular}

\section{Cel dydaktyczny}

Głównym zadaniem przedmiotu jest ugruntowanie wśród studentów zrozumienia dla rangi modelowania w in-żynierii systemów oraz wyposażenie ich w techniki i narzędzia modelowania systemów przy pomocy formali-zmów takich jak UML, BPMN oraz Sieci Petriego. Jako narzędzia analizy własności systemów studenci zapo-znani zostaną z pojęciem S-niezmienników i T-niezmienników w przypadku Sieci Petriego oraz podstawowymi pojęciami z zakresu logik temporalnych LTL i CTL.

\section{Przedmioty wprowadzające}

\begin{tabularx}{\textwidth}{lX}
  \hline
  \textbf{Przedmiot} & \textbf{Wymagane zagadnienia} \\
  \hline
  PRI – Projektowanie systemów informatycznych & ALG – Algebra liniowa z geometrią \\
  AM – Analiza matematyczna & MAD – Matematyka Dyskretna \\
  Umiejętność logicznego wnioskowania oraz umiejętność programowania & --- \\
  \hline
\end{tabularx}

\section{Treści programowe}

\begin{enumerate}
  \item Metody modelowania w inżynierii oprogra-mowania; pojęcia; model a meta model, klasyfi-kacja modeli. Modelowanie danych, struktury i zachowania; przegląd notacji. Modelowanie przebiegu sterowania i przetwarza-nia danych. Zaawansowane modelowanie interakcji w syste-mie z wykorzystaniem d. komunikacji. Opracowanie diagramu komunikacji UML. Transformacje pomiędzy modelami interakcji wyrażonymi d. sekwencji i komunikacji. Modelowanie stanów systemu z wykorzystaniem języka UML. Opracowanie modelu maszyn stanowych z uwzględnienie podstanów i czynności wewnętrznych. Aspekty fizyczne w dokumentowaniu systemu informatycznego. Opracowanie diagramu komponentów. Alokacja komponentów na węzłach. Metody i techniki modelowania biznesowego: BPMN, SPEM UML, rozszerzenia biznesowe procesu RUP. Wyszczególnienie i klasyfikacja procesów bizne-sowych – studium przypadku. Opracowanie dia-gramu procesów biznesowych BPMN. Zaawansowane diagramy procesów biznesowych BPMN. Modelowanie bramek, zdarzeń, dokumentów i interakcji z wykorzystaniem notacji BPMN. Analiza strukturalna i diagramy przepływu da-nych . Modelowanie za pomocą diagramów przepływu danych. Sieci Petriego – struktura i zachowanie dyna-miczne sieci elementarnych. Sieci Petriego –PIPE 2 i inne narzędzia do edy-towania i analizy sieci Petriego. Sieci Petriego typu P/T, graf osiągalności. Modelowanie systemów przy pomocy prostych sieci Petriego. Niezmienniki typu S sieci Petriego. Wyznaczanie niezmienników typu S sieci Petriego. Niezmienniki sieci typu T Petriego. Wyznaczanie niezmienników typu T sieci Petriego. Rozszerzenia sieci Petriego – sieci kolorowane. Rozszerzenia sieci Petriego – sieci z czasem. Wprowadzenie do metod formalnych; zastosowa-nia. Abstrakcyjne Typy Danych vs. specyfikacje model-based – VDM, Z, logika temporalna.
\end{enumerate}

\section{Efekty kształcenia}

\subsection*{Wiedza}
\begin{itemize}
  \item Student zna i rozumie idee modelowania systemów z po-mocą notacji UML, BPMN oraz Sieci Petriego. Student zna i rozumie idee specyfikacji i weryfikacji syste-mów za pośrednictwem logik temporalnych. Student zna i rozumie składnię rdzeniowych diagramów notacji BPMN
  \item Student rozumie pojęcie procesu biznesowego, czynności biznesowych oraz metryk.
\end{itemize}

\subsection*{Umiejętności}
\begin{itemize}
  \item Potrafi modelować systemy przy pomocy poznanych na zajęciach formalizmów. Potrafi dokonać analizy systemów w oparciu o grafy stanów osiągalnych. Potrafi dokonać analizy systemów w oparciu o niezmienniki typu S i niezmienniki typu T.
\end{itemize}

\section{Kryteria oceny}

\begin{itemize}
  \item Wykład problemowy, poświęcony analizie kolejnych zagadnień omawianych w ramach przedmiotu
  \item Ćwiczenia rozszerzone o projekty programistyczne poświęcone analizie leksykalnej i rozpoznawaniu języków przy pomocy automatów
  \item zaliczenie ćwiczeń  wymaga zdobycia co najmniej  50\% punktów możliwych do zdobycia na kolokwium oraz za zrealizowane projekty programistyczne
  \item Warunkiem dopuszczenia do egzaminu jest zaliczenie ćwiczeń.
  \item Warunkiem zdania egzaminu pisemnego jest zdobycie co najmniej  50\% punktów.
\end{itemize}

\section{Metody dydaktyczne}

Wykład, laboratoria, praca własna studenta.

\section{Literatura}

\textbf{Podstawowa:}
\begin{itemize}
  \item - M. Szpyrka, Sieci Petriego w modelowaniu i analizie systemów współbieżnych, Helion, 2008
  \item W. Dąbrowski: Modelowanie systemów informatycznych w języku UML, Wydawnictwo Naukowe PWN, 2009
\end{itemize}

\textbf{Uzupełniająca:}
\begin{itemize}
  \item J. Cheesman, Komponenty UML, WNT, 2004
  \item Wrycza S., Marcinkowski B., Wyrzykowski K.: Język UML 2.0 w modelowaniu systemów informatycznych, Helion, 2005
  \item Gawin B., Marcinkowski B.: Symulacja procesów biznesowych. Standardy BPMS i BPMN w praktyce, He-lion, 2013
\end{itemize}

\end{document}
