% ===========================================================
%  Sylabus: Społeczne aspekty informatyki (SAI)
% ===========================================================
\documentclass[12pt, a4paper]{article}

\usepackage[T1]{fontenc}
\usepackage[utf8]{inputenc}
\usepackage[polish]{babel}
\usepackage{lmodern}
\usepackage{microtype}
\usepackage[a4paper, top=2.5cm, bottom=2.5cm, left=2.5cm, right=2.5cm]{geometry}
\usepackage{xcolor}
\usepackage{graphicx}
\usepackage{booktabs}
\usepackage{tabularx}
\usepackage{longtable}
\usepackage{multirow}
\usepackage{array}
\usepackage{colortbl}
\usepackage{enumitem}
\usepackage{fancyhdr}
\usepackage{titlesec}
\usepackage{mdframed}
\usepackage[colorlinks=true, linkcolor=red!70!black, urlcolor=red!70!black]{hyperref}
\usepackage{eso-pic}
\usepackage{tikz}

\definecolor{pjatkRed}{RGB}{180,0,0}
\definecolor{pjatkGray}{RGB}{80,80,80}
\definecolor{pjatkLightGray}{RGB}{245,245,245}
\definecolor{tableHeader}{RGB}{220,220,220}

\pagestyle{fancy}\fancyhf{}
\renewcommand{\headrulewidth}{0.4pt}
\renewcommand{\footrulewidth}{0.4pt}
\fancyhead[L]{\small\textcolor{pjatkGray}{PJATK -- Filia w Gdańsku \textbar\ Informatyka}}
\fancyhead[R]{\small\textcolor{pjatkGray}{Sylabus: SAI}}
\fancyfoot[C]{\small\thepage}

\titleformat{\section}{\large\bfseries\color{pjatkRed}}{\thesection.}{0.5em}{}
  [\color{pjatkRed}\rule{\linewidth}{0.8pt}]
\setlist{noitemsep, topsep=3pt, parsep=2pt}

\newmdenv[linecolor=pjatkRed, linewidth=1.2pt, backgroundcolor=pjatkLightGray,
  innerleftmargin=10pt, innerrightmargin=10pt, innertopmargin=8pt,
  innerbottommargin=8pt, roundcorner=4pt]{infobox}

\begin{document}

\AddToShipoutPictureBG{%
  \begin{tikzpicture}[remember picture, overlay]
    \node[opacity=0.5] at (current page.center) {%
      \includegraphics[width=14cm]{C:/Users/adamu/WebstormProjects/pj-studies/latex/PJATK_pl_sygnet_transparent-eps-converted-to}%
    };
  \end{tikzpicture}%
}

\begin{center}
  \includegraphics[height=2cm]{C:/Users/adamu/WebstormProjects/pj-studies/latex/PJATK_pl_poziom_1}\\[0.8cm]
  {\LARGE\bfseries\color{pjatkRed} SYLABUS PRZEDMIOTU}\\[0.8cm]
\end{center}

\begin{infobox}
\begin{tabularx}{\textwidth}{@{}lX@{}}
  \textbf{Nazwa przedmiotu:}  & {\bfseries Społeczne aspekty informatyki} \\[3pt]
  \textbf{Kod przedmiotu:}    & SAI \\[3pt]
  \textbf{Kierunek / Profil:} & Informatyka / praktyczny \\[3pt]
  \textbf{Tryb studiów:}      & niestacjonarny \\[3pt]
  \textbf{Rok / Semestr:}     & 4 / 8 \\[3pt]
  \textbf{Charakter:}         & obowiązkowy \\[3pt]
  \textbf{Odpowiedzialny:}    & dr Marta Czerwonka (mczerwonka@pjwstk.edu.pl) \\[3pt]
  \textbf{Wersja z dnia:}     & 19.02.2026 \\
\end{tabularx}
\end{infobox}

\vspace{1cm}

\section{Godziny zajęć i punkty ECTS}

\begin{center}
\begin{tabular}{|>{\centering\arraybackslash}p{2.0cm}
                |>{\centering\arraybackslash}p{2.0cm}
                |>{\centering\arraybackslash}p{2.0cm}
                |>{\centering\arraybackslash}p{2.4cm}
                |>{\centering\arraybackslash}p{2.4cm}
                |>{\centering\arraybackslash}p{2.0cm}
                |>{\centering\arraybackslash}p{1.4cm}|}
\hline
\rowcolor{tableHeader}
\textbf{Wykłady} & \textbf{Ćwiczenia} & \textbf{Laboratorium} &
\textbf{Z prowadzącym} & \textbf{Praca własna} & \textbf{Łącznie} & \textbf{ECTS} \\
\hline
16 h & --- & 16 h & 32 h & 68 h & 100 h & \textbf{4} \\
\hline
\end{tabular}
\end{center}

\section{Forma zajęć}

\begin{tabular}{ll}
  \hline
  \textbf{Forma zajęć} & \textbf{Sposób zaliczenia} \\
  \hline
  Ćwiczenia & Zaliczenie z oceną \\
  Wykład & Nieoceniany \\
  \hline
\end{tabular}

\section{Cel dydaktyczny}

Zapoznanie ze społecznymi aspektami rozwoju technologii cyfrowych, ich używaniem i nadużywaniem, z uwzględnieniem odpowiedzialności zawodowo-etycznej informatyków, zagadnień prawnych, wpływu technologii na jednostki, grupy zawodowe oraz społeczeństwo. Poruszane są też zagadnienia komunikacji międzyludzkiej i rozwiązywania konfliktów.

\section{Przedmioty wprowadzające}

\begin{tabularx}{\textwidth}{lX}
  \hline
  \textbf{Przedmiot} & \textbf{Wymagane zagadnienia} \\
  \hline
  PRIN – Procesy Innowacyjne & Wiedza ogólna na temat technologii \\
  \hline
\end{tabularx}

\section{Treści programowe}

\begin{enumerate}
  \item 1Społeczny kontekst informatyki; odpowiedzialność zawodowa i etyczna. Profesjonalizm w inżynierii oprogramowania. Wprowadzenie do zajęć. Dyskusja tematów związanych ze społecznymi aspektami wykorzystania Internetu Analiza zawartości i zakresu ACM/IEEE Code of Ethics and Professional Practice
  \item 2Wyzwania pracy w zespołach zdalnych i rozproszonych.Strategie pracy i narzędzia usprawniające pracę w zespołach zdalnych i rozproszonych
  \item 3Usprawnianie pracy i rozwiązywanie problemów w zespole informatycznym. Metodyka Scrum.Identyfikacja problemów (projekty inżynierskie)
  \item 4Rozszerzona analiza wpływu. Potrzeba, usytuowanie, zakres. Metoda SODIS (Software Development Impact Statement).Prowadzenie analizy rozszerzonej metodą SoDIS
  \item 5Ryzyko w projekcie informatycznym. Ryzyko i odpowiedzialność związane z systemami informatycznymi.Wprowadzenie do analizy ryzyka na przykładzie wczesnego etapu projektów inżynierskich
  \item 6Obszary styku i zagrożeń; przedsiębiorczość; wiarygodność; zjawiska społeczne i kulturowe, swoboda wypowiedzi, własność intelektualnaĆwiczenie sztuki prezentacji
  \item 7Znaczenie „czynnika ludzkiego” w projekcie
  \item informatycznym – szanse i zagrożeniaAnaliza studium przypadku produktów, usług i firm informatycznych – czynników decydujących o sukcesie/porażce
  \item 8Wybrane aspekty efektywnej pracy zespołowejAnaliza strategii i narzędzi sprzyjających pracy zespołowej
  \item 9Psychologiczne aspekty skutecznej komunikacji. Od myślenia abstrakcyjnego do konkretyzacji przekazu.Analiza i projektowanie protokołów komunikacyjnych – zastosowanie w projekcie inżynierskim
  \item 10Prezentowanie informacji. StorytellingZasady efektywnej prezentacji danych
  \item 11Aspekty prawne budowy i działania systemów informatycznych; audyt informatycznyPrzygotowanie i prezentacja seminaryjna tematów związanych z zagadnieniami prywatności i swobód obywatelskich
  \item 12Własność intelektualna. Zagadnienia prywatności, swobód obywatelskich (wykład zaproszony)Przygotowanie i prezentacja seminaryjna tematów związanych z aspektami prawnymi budowy i działania systemów informatycznych
  \item 13Slow Tech Analiza studium przypadków
  \item 14Metody i narzędzia analizy aspektów etyki w projekcie informatycznym. Analiza przypadku, podejście proceduralne; wzorce etyczneAnaliza wybranych przypadków metodą Bynuma i Rogersona
  \item 15Problemy ochrony i poufności danych
\end{enumerate}

\section{Efekty kształcenia}

\subsection*{Wiedza}
\begin{itemize}
  \item Student zna i rozumie jak przeprowadzić analizę wykonalności projektu ze szczególnym uwzględnieniem aspektów ekonomicznych, etycznych i społecznych
  \item Student zna i rozumie podstawowe problemy społeczno-etyczne związane z technologią informacyjną, rozumie zasady odpowiedzialności zawodowej informatyka
\end{itemize}

\subsection*{Umiejętności}
\begin{itemize}
  \item Student potrafi opracować stosowną dokumentację na potrzeby projektu informatycznego
  \item Student potrafi  opracować stosowną dokumentację na potrzeby projektu informatycznego
  \item Student potrafi  opracować stosowną dokumentację na potrzeby projektu informatycznego z uwzględnieniem analizy społeczno-prawno-etyczne
  \item Student potrafi  opracować stosowną dokumentację (np. raport wykonalności) na potrzeby projektu informatycznego
\end{itemize}

\subsection*{Kompetencje społeczne}
\begin{itemize}
  \item Student jest gotów do  podejmowania dyskusji na temat społeczno-etycznego wpływu informatyki
  \item Student jest gotów do  podejmowania dyskusji (także poza uczelnią) na temat konsekwencji działalności inżynierskiej oraz związaną z tym odpowiedzialnością
  \item Student jest gotów do  zaangażowania się w rozwiązywanie problemów etyczno-prawnych w zakresie realizowanego projektu
  \item Student jest gotów do  wykorzystywania umiejętności miękkich z zakresu komunikacji i adaptacji do sytuacji
\end{itemize}

\section{Kryteria oceny}

\begin{itemize}
  \item wykład z elementami dyskusji z prezentacją multimedialną, wykład zaproszony
  \item burza mózgów
  \item rozwiązywanie zadań
  \item analiza przypadków
  \item Kryteria oceny
  \item ustalenie oceny zaliczeniowej na podstawie ocen cząstkowych otrzymanych w trakcie trwania semestru: analizy wpływu, prezentacji i zagadnień z komunikacji
  \item brak
\end{itemize}

\section{Metody dydaktyczne}

Wykład, laboratoria, praca własna studenta.

\section{Literatura}

\textbf{Podstawowa:}
\begin{itemize}
  \item Bartyzel, M (2015) Oprogramowanie szyte na miarę. Jak rozmawiać z klientem, który wiecz czego chce. Helion
  \item Literatura pomocnicza
  \item ACM/IEEE-CS Joint Task Force on Software Engineering Ethics and Professional Practices, Software Engineering Code of Ethics and Professional Practice.
  \item Bartyzel, M. (2016). Getting Things Programmed. Droga do efektywności. Wydawnictwo Helion.
  \item BCS Code of Conduct (2015) – BCS – The Chartered Institute for IT
  \item BCS Code of Practices (2004) – BCS – The Chartered Institute for IT
  \item Bond, M. L. (2014). Siła innych. O presji otoczenia, syndromie grupowego myślenia i o tym, jak ludzie wokół na nas wpływają i na wszystkie nasze poczynania. Wydawnictwo Naukowe PWN.
  \item Bynum T.W. and Rogerson S. (2003), Computer Ethics and Professional Responsibility. Blackwell Publ.
  \item Gotterbarn D. (2001), Reducing Software failures: Addressing the Ethical Risks of the Software Development Lifecycle. Proc. of the 5th International Conference on The Social and Ethical Impacts of ICT ETHICOMP 2001, Gdańsk, June 2001, vol. 2, pp. 10 -19.
  \item Gotterbarn, D.W., Bruckman, A., Flick, C., Miller, K. and Wolf, M.J., 2018. ACM code of ethics: a guide for positive action.
  \item Gotterbarn, D., Wolf, M.J., Flick, C. and Miller, K., (2018). THINKING PROFESSIONALLY The continual evolution of interest in computing ethics. ACM Inroads, 9(2), pp.10-12.
  \item Nowocień, R. (2020). Zespoły wirtualne i rozproszone. Zdalne zarządzanie projektem informatycznym.  Wydawnictwo Helion.
  \item Peeling, N. (2010). Negocjacje. Co dobry negocjator wie, robi i mówi. Polskie Wydawnictwo Ekonomiczne.
  \item Schein, E. H. (2019). Potęga dobrej komunikacji w zespole. O trudnej sztuce zadawania pytań.   Wydawnictwo Naukowe PWN.
  \item Stefaniuk, T. (2014). Komunikacja w zespole wirtualnym. Difin.
  \item Szejko S. (2002), Incorporating Ethics into the Software Process. Proc. of the VIth International Conference on The Transformation of Organisations in the Information Age: Social and Ethical Implications, ETHICOMP 2002, Lisbon, 2002, pp. 271 - 279.
  \item Szejko S.: Patterns of Ethical Behaviour and Decision Making. Mat. VII konferencji ETHICOMP 2004 "Challenges for the Citizen of the Information Society”, Syros, Grecja., kwiecień 2004, vol.2, str. 826 – 838.
  \item Wróblewski, P. (2020).  Zwinnie do przodu. Poradnik kierownika projektów informatycznych.  Wydawnictwo Helion.
  \item Zalewski A., Cegieła R., Sacha K. (2009): Modele i praktyka audytu informatycznego
  \item Ustawy:
  \item PRAWO WŁASNOŚCI PRZEMYSŁOWEJ (Dz. U. nr 119/2003 poz. 1117 z późn. zm.),
  \item PRAWIE AUTORSKIM I PRAWACH POKREWNYCH (Dz.U. z 2016 r. poz. 666),
  \item O OCHRONIE BAZ DANYCH (Dz. U. nr 128/2001 poz. 1402 z późn. zm.),
  \item OGÓLNE ROZPORZĄDZENIE O OCHRONIE DANYCH (Rozporządzenie 679/2016/UE)
\end{itemize}

\textbf{Uzupełniająca:}
\begin{itemize}
  \item Brak danych.
\end{itemize}

\end{document}
