% ===========================================================
%  Sylabus: Algorytmy i struktury danych (ASD)
% ===========================================================
\documentclass[12pt, a4paper]{article}

\usepackage[T1]{fontenc}
\usepackage[utf8]{inputenc}
\usepackage[polish]{babel}
\usepackage{lmodern}
\usepackage{microtype}
\usepackage[a4paper, top=2.5cm, bottom=2.5cm, left=2.5cm, right=2.5cm]{geometry}
\usepackage{xcolor}
\usepackage{graphicx}
\usepackage{booktabs}
\usepackage{tabularx}
\usepackage{longtable}
\usepackage{multirow}
\usepackage{array}
\usepackage{colortbl}
\usepackage{enumitem}
\usepackage{fancyhdr}
\usepackage{titlesec}
\usepackage{mdframed}
\usepackage[colorlinks=true, linkcolor=red!70!black, urlcolor=red!70!black]{hyperref}
\usepackage{eso-pic}
\usepackage{tikz}

\definecolor{pjatkRed}{RGB}{180,0,0}
\definecolor{pjatkGray}{RGB}{80,80,80}
\definecolor{pjatkLightGray}{RGB}{245,245,245}
\definecolor{tableHeader}{RGB}{220,220,220}

\pagestyle{fancy}\fancyhf{}
\renewcommand{\headrulewidth}{0.4pt}
\renewcommand{\footrulewidth}{0.4pt}
\fancyhead[L]{\small\textcolor{pjatkGray}{PJATK -- Filia w Gdańsku \textbar\ Informatyka}}
\fancyhead[R]{\small\textcolor{pjatkGray}{Sylabus: ASD}}
\fancyfoot[C]{\small\thepage}

\titleformat{\section}{\large\bfseries\color{pjatkRed}}{\thesection.}{0.5em}{}
  [\color{pjatkRed}\rule{\linewidth}{0.8pt}]
\setlist{noitemsep, topsep=3pt, parsep=2pt}

\newmdenv[linecolor=pjatkRed, linewidth=1.2pt, backgroundcolor=pjatkLightGray,
  innerleftmargin=10pt, innerrightmargin=10pt, innertopmargin=8pt,
  innerbottommargin=8pt, roundcorner=4pt]{infobox}

\begin{document}

\AddToShipoutPictureBG{%
  \begin{tikzpicture}[remember picture, overlay]
    \node[opacity=0.5] at (current page.center) {%
      \includegraphics[width=14cm]{C:/Users/adamu/WebstormProjects/pj-studies/latex/PJATK_pl_sygnet_transparent-eps-converted-to}%
    };
  \end{tikzpicture}%
}

\begin{center}
  \includegraphics[height=2cm]{C:/Users/adamu/WebstormProjects/pj-studies/latex/PJATK_pl_poziom_1}\\[0.8cm]
  {\LARGE\bfseries\color{pjatkRed} SYLABUS PRZEDMIOTU}\\[0.8cm]
\end{center}

\begin{infobox}
\begin{tabularx}{\textwidth}{@{}lX@{}}
  \textbf{Nazwa przedmiotu:}  & {\bfseries Algorytmy i struktury danych} \\[3pt]
  \textbf{Kod przedmiotu:}    & ASD \\[3pt]
  \textbf{Kierunek / Profil:} & Informatyka / praktyczny \\[3pt]
  \textbf{Tryb studiów:}      & niestacjonarny \\[3pt]
  \textbf{Rok / Semestr:}     & 2 / 3 \\[3pt]
  \textbf{Charakter:}         & obowiązkowy \\[3pt]
  \textbf{Odpowiedzialny:}    & dr hab. Marek A. Bednarczyk \\[3pt]
  \textbf{Wersja z dnia:}     & 19.02.2026 \\
\end{tabularx}
\end{infobox}

\vspace{1cm}

\section{Godziny zajęć i punkty ECTS}

\begin{center}
\begin{tabular}{|>{\centering\arraybackslash}p{2.0cm}
                |>{\centering\arraybackslash}p{2.0cm}
                |>{\centering\arraybackslash}p{2.0cm}
                |>{\centering\arraybackslash}p{2.4cm}
                |>{\centering\arraybackslash}p{2.4cm}
                |>{\centering\arraybackslash}p{2.0cm}
                |>{\centering\arraybackslash}p{1.4cm}|}
\hline
\rowcolor{tableHeader}
\textbf{Wykłady} & \textbf{Ćwiczenia} & \textbf{Laboratorium} &
\textbf{Z prowadzącym} & \textbf{Praca własna} & \textbf{Łącznie} & \textbf{ECTS} \\
\hline
16 h & --- & 24 h & 40 h & 85 h & 125 h & \textbf{5} \\
\hline
\end{tabular}
\end{center}

\section{Forma zajęć}

\begin{tabular}{ll}
  \hline
  \textbf{Forma zajęć} & \textbf{Sposób zaliczenia} \\
  \hline
  Ćwiczenia & Zaliczenie z oceną \\
  Wykład & Egzamin \\
  \hline
\end{tabular}

\section{Cel dydaktyczny}

Opanowanie umiejętności analizy programów iteracyjnych i rekursywnych oraz wykorzystania aparatu matematycznego do analizy złożoności algorytmówi wiedzy na temat złożoności problemów algorytmicznych, klas złożoności i ich praktycznegoznaczenia; analiza złożoności algorytmów sortujących i wyszukujących; drzewa, stosy, kolejki itablice z haszowaniem; programowanie dynamiczne i algorytmy zachłanne.

\section{Przedmioty wprowadzające}

\begin{tabularx}{\textwidth}{lX}
  \hline
  \textbf{Przedmiot} & \textbf{Wymagane zagadnienia} \\
  \hline
  PRG1 – Programowanie & POJ – Programowanie obiektowe \\
  ALG – Algebra liniowa z geometrią & AM – Analiza matematyczna \\
  MAD – Matematyka Dyskretna & Umiejętność programowania programów iteracyjnych i rekursywnych \\
  Umiejętność liczenia granic i sum szeregów oraz całek & funkcji elementarnych \\
  Umiejętność wnioskowania & --- \\
  \hline
\end{tabularx}

\section{Treści programowe}

\begin{enumerate}
  \item Algorytmy iproblemy algorytmiczne; Złożoność czasowa i pamięciowa; przykłady.
  \item Analiza złożoności problemu wyszukiwania informacji w kolekcji uporządkowanej i nieuporządkowanej, notacje asymptotyczne
  \item Specyfikowanie problemów, własność stopu, problemy nierozstrzygalne
  \item Logika Hoare’ai metody dowodzenia własności stopu
  \item Proste algorytmy sortujące
  \item Quicksort; algorytmy sortujące w czasie liniowym
  \item Programowanie dynamiczne i zachłanne
  \item Kolejki, stosy, listy i drzewa
  \item Tablice z haszowaniem
  \item Twierdzenie o rekursji uniwersalnej
  \item Drzewa BST, drzewa czerwono-czarne
  \item Algorytmy grafowe
  \item Podstawy teorii złożoności, klasy złożoności, metody redukcji i problemy zupełne w klasie
  \item Języki formalne – języki regularne, kontekstowe i bezkontekstowe,automaty i gramatyki.
\end{enumerate}

\section{Efekty kształcenia}

\subsection*{Wiedza}
\begin{itemize}
  \item Student zna i rozumie pojęcia takie jak specyfikacja, weryfikacja i własność stopu algorytmu.
  \item Student zna i rozumieróżnicę między algorytmami iteracyjnymi i rekursywnymi.
  \item Student zna i rozumie istotę podstawowych struktur danych oraz potrafi dobrać je do zadania.
  \item Student zna i rozumie notację asymptotyczną opisującą złożoność obliczeniową.
  \item Zna i rozumie pojęcie złożoności problemu algorytmicznego oraz klasy złożoności.
\end{itemize}

\subsection*{Umiejętności}
\begin{itemize}
  \item Student potrafi dokonać analizyzłożoności programówiteracyjnych.
  \item Potrafi przeanalizowaćzłożoność programówopartych o zasadę ,,dziel izwyciężaj’’.
  \item Student potrafi dobrać struktury danych odpowiednie do zadania oraz potrafi oszacować złożoność algorytmu.
  \item Student potrafi wykorzystać wiedzę i umiejętności analizy algorytmów w realizacjiprojektu dyplomowego.
\end{itemize}

\section{Kryteria oceny}

\begin{itemize}
  \item Wykład problemowy, poświęcony analizie oceny lub wymagania egzaminacyjnekolejnych zagadnień omawianych w ramach przedmiotu
  \item rozwiązywanie zadań
  \item Kryteria oceny
  \item Na ćwiczeniach odbędą się dwa kolokwia oraz oceniane aktywności dodatkowe (kartkówki, referaty).
  \item Student musi uzyskać minimum 50\% punktów możliwych do zdobycia.
  \item Skala ocen:
  \item Poniżej 50\% - ndst
  \item Od 50\% - dst
  \item Od 60\% - dst+
  \item Od 70\% - db
  \item Od 80\% - db+
  \item Od 90\% - bdb
  \item Zdanie egzaminu pisemnyz zadaniami problemowymi oraz testowymi jednokrotnego wyboru wymaga zdobyciaminimum 50\% punktów.
  \item Skala ocen:
  \item Poniżej 50\% - ndst
  \item Od 50\% - dst
  \item Od 60\% - dst+
  \item Od 70\% - db
  \item Od 80\% - db+
  \item Od 90\% - bdb
\end{itemize}

\section{Metody dydaktyczne}

Wykład, laboratoria, praca własna studenta.

\section{Literatura}

\textbf{Podstawowa:}
\begin{itemize}
  \item T. Cormen, Ch. Leiserson, R. Rivest, C. Stein.Wprowadzenie do algorytmów, Wydawnictwo Naukowe PWN,2022
\end{itemize}

\textbf{Uzupełniająca:}
\begin{itemize}
  \item Szepietowski.Matematyka dyskretna, Wydawnictwo UG, 2004
  \item G. Mirkowska.Algorytmy i struktury danych, Wydawnictwo PJWSTK, 2006
  \item G. Mirkowska.Algorytmy i struktury danych – Zadania, Wydawnictwo PJWSTK, 2006
  \item Z. Michalewicz.Algorytmy genetyczne + struktury danych = programy ewolucyjne, Wydawnictwo PJWSTK 2003
\end{itemize}

\end{document}
