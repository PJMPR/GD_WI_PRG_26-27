% ===========================================================
%  Sylabus: Projekt (PRO)
% ===========================================================
\documentclass[12pt, a4paper]{article}

\usepackage[T1]{fontenc}
\usepackage[utf8]{inputenc}
\usepackage[polish]{babel}
\usepackage{lmodern}
\usepackage{microtype}
\usepackage[a4paper, top=2.5cm, bottom=2.5cm, left=2.5cm, right=2.5cm]{geometry}
\usepackage{xcolor}
\usepackage{graphicx}
\usepackage{booktabs}
\usepackage{tabularx}
\usepackage{longtable}
\usepackage{multirow}
\usepackage{array}
\usepackage{colortbl}
\usepackage{enumitem}
\usepackage{fancyhdr}
\usepackage{titlesec}
\usepackage{mdframed}
\usepackage[colorlinks=true, linkcolor=red!70!black, urlcolor=red!70!black]{hyperref}
\usepackage{eso-pic}
\usepackage{tikz}

\definecolor{pjatkRed}{RGB}{180,0,0}
\definecolor{pjatkGray}{RGB}{80,80,80}
\definecolor{pjatkLightGray}{RGB}{245,245,245}
\definecolor{tableHeader}{RGB}{220,220,220}

\pagestyle{fancy}\fancyhf{}
\renewcommand{\headrulewidth}{0.4pt}
\renewcommand{\footrulewidth}{0.4pt}
\fancyhead[L]{\small\textcolor{pjatkGray}{PJATK -- Filia w Gdańsku \textbar\ Informatyka}}
\fancyhead[R]{\small\textcolor{pjatkGray}{Sylabus: PRO}}
\fancyfoot[C]{\small\thepage}

\titleformat{\section}{\large\bfseries\color{pjatkRed}}{\thesection.}{0.5em}{}
  [\color{pjatkRed}\rule{\linewidth}{0.8pt}]
\setlist{noitemsep, topsep=3pt, parsep=2pt}

\newmdenv[linecolor=pjatkRed, linewidth=1.2pt, backgroundcolor=pjatkLightGray,
  innerleftmargin=10pt, innerrightmargin=10pt, innertopmargin=8pt,
  innerbottommargin=8pt, roundcorner=4pt]{infobox}

\begin{document}

\AddToShipoutPictureBG{%
  \begin{tikzpicture}[remember picture, overlay]
    \node[opacity=0.5] at (current page.center) {%
      \includegraphics[width=14cm]{C:/Users/adamu/WebstormProjects/pj-studies/latex/PJATK_pl_sygnet_transparent-eps-converted-to}%
    };
  \end{tikzpicture}%
}

\begin{center}
  \includegraphics[height=2cm]{C:/Users/adamu/WebstormProjects/pj-studies/latex/PJATK_pl_poziom_1}\\[0.8cm]
  {\LARGE\bfseries\color{pjatkRed} SYLABUS PRZEDMIOTU}\\[0.8cm]
\end{center}

\begin{infobox}
\begin{tabularx}{\textwidth}{@{}lX@{}}
  \textbf{Nazwa przedmiotu:}  & {\bfseries Projekt} \\[3pt]
  \textbf{Kod przedmiotu:}    & PRO \\[3pt]
  \textbf{Kierunek / Profil:} & Informatyka / praktyczny \\[3pt]
  \textbf{Tryb studiów:}      & stacjonarny \\[3pt]
  \textbf{Rok / Semestr:}     & 3 / 5 \\[3pt]
  \textbf{Charakter:}         & obowiązkowy \\[3pt]
  \textbf{Odpowiedzialny:}    & dr hab. inż. Marta Łabuda \\[3pt]
  \textbf{Wersja z dnia:}     & 19.02.2026 \\
\end{tabularx}
\end{infobox}

\vspace{1cm}

\section{Godziny zajęć i punkty ECTS}

\begin{center}
\begin{tabular}{|>{\centering\arraybackslash}p{2.0cm}
                |>{\centering\arraybackslash}p{2.0cm}
                |>{\centering\arraybackslash}p{2.0cm}
                |>{\centering\arraybackslash}p{2.4cm}
                |>{\centering\arraybackslash}p{2.4cm}
                |>{\centering\arraybackslash}p{2.0cm}
                |>{\centering\arraybackslash}p{1.4cm}|}
\hline
\rowcolor{tableHeader}
\textbf{Wykłady} & \textbf{Ćwiczenia} & \textbf{Laboratorium} &
\textbf{Z prowadzącym} & \textbf{Praca własna} & \textbf{Łącznie} & \textbf{ECTS} \\
\hline
--- & --- & 45 h & 45 h & 30 h & 75 h & \textbf{3} \\
\hline
\end{tabular}
\end{center}

\section{Forma zajęć}

\begin{tabular}{ll}
  \hline
  \textbf{Forma zajęć} & \textbf{Sposób zaliczenia} \\
  \hline
  Projekt & Zaliczenie z oceną \\
  \hline
\end{tabular}

\section{Cel dydaktyczny}

Celem zajęć jest zwiększenie umiejętności związanych z podstawowymi zagadnieniami i praktyką planowania i prowadzenia projektu informatycznego, w szczególności – projektu mającego na celu wytworzenie produktu programowego w technologii obiektowej oraz zdobycie przez studentów doświadczeń pracy w zespole. Zajęcia projektowe są skorelowane z wykładem przedmiotu Budowa i integracja systemów (BYT), zapewniając kompleksowość przedstawianych zagadnień inżynierii oprogramowania i dając studentom możliwość samodzielnego, metodycznego przeprowadzenia niedużych projektów informatycznych – od zdefiniowania problemu, poprzez dobór strategii i zaplanowanie jego rozwiązania do wykonania praktycznych elementów produktu, z użyciem wzorców i komponentów projektowych oraz przeprowadzeniem testów systemowych i walidacyjnych, a także ich udokumentowaniem. Wzorcowym projektem jest wykorzystanie technologii obiektowej w cyklu kaskadowym lub komponentowym, oraz zwinne strategie realizacji projektów. Istotnymi czynnikami doboru tematów są praktyczna przydatność, a także, w miarę możliwości, zapewnienie współpracy studentów z rzeczywistym klientem projektu. Tematy są prowadzone są w 2-4 osobowych grupach.

\section{Przedmioty wprowadzające}

\begin{tabularx}{\textwidth}{lX}
  \hline
  \textbf{Przedmiot} & \textbf{Wymagane zagadnienia} \\
  \hline
  • Programowanie 2 & • Wstęp do informatyki i architektury komputerów \\
  • Relacyjne bazy danych & • Projektowanie systemów informacyjnych \\
  • Umiejętność programowania obiektowego na poziomie podstawowym & • Wiedza o organizacji systemu komputerowego oraz architekturze \\
  • Znajomość projektowania baz danych & • Znajomość zagadnień analizy i projektowania SI i notacji UML \\
  • Wcześniejsze lub równoczesne zrozumienie zagadnień planowania i prowadzenia projektu informatycznego. & --- \\
  \hline
\end{tabularx}

\section{Treści programowe}

\begin{enumerate}
  \item Projekt
  \item Konstrukcja aplikacji w oparciu o przyjętą metodykę, wzorce (analityczne, projektowe) frameworki i komponenty + dokumentacja projektu
  \item 1. Wprowadzenie, zasady zaliczenia przedmiotu. Założenia prac projektowych, ich prowadzenia i dokumentowania. Zapoznanie się z dostępnym środowiskiem. Dyskusja nad sformułowaniami projektów i grup projektowych. Opis projektu, założenia projektowe Analiza dokumentacji przykładowych rozwiązań projektowych.
  \item 2. Praca nad projektem: Identyfikacja problemu. Kontekst i udziałowcy problemu; klient projektu. Wzbogacony wizerunek, definicja i wizja systemu, zakres systemu, rodzaj produktu. Przygotowanie Dokumentu Założeń Wstępnych
  \item 3. Specyfikacja wymagań na system, podział i analiza wymagań. User stories. Studium wykonalności projektu; analiza organizacyjna, mikro i makrootoczenie projektu; dyskusja wariantów rozwiązania. Dobór strategii realizacji projektu w zależności od rodzaju produktu. Debata
  \item 4. Modelowanie systemu (przypadki użycia, diagramy sekwencji i/lub czynności). Logiczny diagram klas
  \item 5. Modelowanie systemu. Architektura systemu: podział na podsystemy i komponenty. Projekt bazy danych.
  \item 6. Harmonogram prac implementacyjnych, integracji, testowania oraz dodatkowej dokumentacji z podziałem na członków zespołu i ich role.
  \item 7. Konfiguracja środowiska programistycznego i technologicznego.
  \item 8. Prace implementacyjne. Realizacja zadań projektowych wg. Harmonogramu.
  \item 9. Prace implementacyjne. Realizacja zadań projektowych wg. Harmonogramu.
  \item 10. Prace implementacyjne. Wstępna ocena postępów, analiza zagrożeń.
  \item 11. Implementacja i integracja oprogramowania.
  \item 12. Implementacja i integracja. Inspekcja kodu. Analiza bezpieczeństwa.
  \item 13. Ocena i poprawa jakości oprogramowania. Scenariusze testów. Testowanie i walidacja oprogramowania; przygotowanie raportu z przeprowadzonych testów.
  \item 14. Finalizacja prac implementacyjnych i dokumentacyjnych; odbiór techniczny, przygotowanie raportu końcowego.
  \item 15. Prezentacja, omówienie i podsumowanie projektów. Zaliczenie przedmiotu.
\end{enumerate}

\section{Efekty kształcenia}

\subsection*{Wiedza}
\begin{itemize}
  \item Student zna i rozumie zaawansowane pojęcia z zakresu zagadnień inżynierii oprogramowania, standardów i kształtu cykli wytwórczych oraz ewolucji oprogramowania; zna podstawy zarządzania przedsięwzięciem programistycznym i rozumie problem jakości oprogramowania; rozumie rolę modelowania i ma szczegółową wiedzę o obiektowym wytwarzaniu oprogramowania i notacji UML, zna i rozumie zasady korzystania z wzorców programowych i standardowych API; ma wiedzę o typowych narzędziach i środowiskach wspomagających;
  \item Student zna i rozumie podstawowe pojęcia z zakresu kluczowych zagadnień inżynierii wymagań, rozumie potrzebę systematycznego budowania i pielęgnacji specyfikacji wymagań; ma szczegółową wiedzę dotyczącą ich specyfikacji, analizy i modelowania z użyciem dostępnych narzędzi.
  \item Student zna i rozumie kluczowe pojęcia z zakresu walidacji i testowania oprogramowania.
  \item Student zna i rozumie pojęcia z zakresu planowania przedsięwzięcia informatycznego, wstępnej oceny ekonomicznej, aspektów społecznych oraz analizy wykonalności.
\end{itemize}

\subsection*{Umiejętności}
\begin{itemize}
  \item Student potrafi pozyskiwać specjalistyczne informacje z literatury, baz danych, systemów patentowych, Internetu oraz innych źródeł, w języku polskim i angielskim w zakresie informatyki; potrafi dokonywać oceny, krytycznej analizy i syntezy tych informacji,  a także wyciągać wnioski oraz formułować i uzasadniać opinie.
  \item Student potrafi przygotować w języku polskim i języku obcym dobrze udokumentowane opracowanie problemów z zakresu informatyki lub dokumentację realizacji zadania inżynierskiego.
  \item Student potrafi pracować w zespole; potrafi oszacować czas i koszty potrzebne na realizację zleconego zadania; potrafi planować, opracować i realizować harmonogram prac zapewniający dotrzymanie terminów.
  \item Student potrafi analizować i wyjaśniać obserwowane zjawiska; tworzyć i weryfikować modeli świata rzeczywistego oraz posługiwać się nimi w celu predykcji zdarzeń i stanów; potrafi posłużyć się właściwe dobranymi środowiskami programistycznymi, symulatorami oraz narzędziami wspomagania komputerowego do symulacji, projektowania i analizy prostych systemów.
  \item Student potrafi wyspecyfikować,  zaprojektować, zaimplementować, przetestować oraz debuggować program; potrafi korzystać z bibliotek, środowisk programistycznych, integrujących i uruchomieniowych.
  \item Student potrafi zaplanować i zrealizować prosty system oprogramowania zgodnie z metodyką obiektową, posługując się wzorcami programowymi, standardami i dobrymi praktykami programistycznymi; potrafi dobrać model procesu wytwarzania oprogramowania do specyfiki przedsięwzięcia, a także dobrać narzędzia wspomagające budowę oprogramowania.
  \item Student potrafi zaplanować i przeprowadzić procesy pozyskiwania, analizy, specyfikacji i modelowania wymagań wobec oprogramowania oraz ich pielęgnacji.
  \item Student potrafi dokonać przeglądu projektu oprogramowania i poprawić jego jakość.
  \item Student potrafi zaplanować i przeprowadzić proces integracji, oceny i realizacji planu testowania oraz dokonać diagnozy defektów.
  \item Student potrafi zaplanować i wytworzyć podstawowe dokumenty związane z realizacją prostego przedsięwzięcia informatycznego, wstępnie ocenić efekty ekonomiczne i społeczne przedsięwzięcia oraz ich wpływ na udziałowców;
  \item Student potrafi zaplanować i przeprowadzić  proces instalacji i uruchomienia całości prostego systemu (system operacyjny, baza danych, aplikacja, oprogramowanie współdziałające).
\end{itemize}

\subsection*{Kompetencje społeczne}
\begin{itemize}
  \item Student jest gotów do współdziałań i współpracy w zespole, przyjmując różne role, m.in. zamawiającego, klienta, analityka, projektanta, wykonawcy.
  \item Student jest gotów do ustalania priorytetów zadań.
\end{itemize}

\section{Kryteria oceny}

\begin{itemize}
  \item Wykład (powiązany treściami z przedmiotem BYT):
  \item dokumentowanie produktów projektowych
  \item burza mózgów
  \item praca grupowa nad projektem z wykorzystaniem nowoczesnych technik i narzędzi informatycznych
  \item Kryteria oceny
  \item prezentacja projektu i dokumentacji
  \item obrona projektu
  \item raport z wykonanego zadania
  \item ocena sporządzonego dokumentu: szablony dokumentów, raporty
  \item ocena sporządzonego oprogramowania: weryfikacja powstałych komponentów oraz produktu końcowego
  \item ocena doboru i wykonania rozwiązań projektowych i powiązanej dokumentacji
  \item Uwaga: Ocena na podstawie oceny produktu realizowanego projektu oraz ocen częściowych, otrzymywanych za wykonanie udokumentowanych komponentów projektowych
\end{itemize}

\section{Metody dydaktyczne}

Wykład, laboratoria, praca własna studenta.

\section{Literatura}

\textbf{Podstawowa:}
\begin{itemize}
  \item 1. Richard Hundhausen  Profesjonalne wytwarzanie oprogramowania z zastosowaniem Scruma i usług Azure DevOps, Prima 2021
  \item 2. Krzysztof Sacha,  Inżynieria Oprogramowania, PWN 2022
  \item 3. Ian Sommerville,  Inżynieria Oprogramowania, wyd.10, PWN 2020
  \item 4. Max Kanat-Alexander, Zrozumieć oprogramowanie, Helion 2019
  \item 5. Robert C. Martin, Czysta architektura. Struktura i design oprogramowania. Przewodnik dla profesjonalistów, Helion 2022
  \item 6. Mike Cohn, Agile. Metodyki zwinne w planowaniu projektów, Helion 2019
  \item 7. Robert C. Martin, Zwinnewytwarzanie oprogramowania. Najlepsze zasady, wzorce i praktyki, Helion 2017
\end{itemize}

\textbf{Uzupełniająca:}
\begin{itemize}
  \item 1. Bernd Bruegge,  Allen H. Dutoit, Inżynieria oprogramowania w ujęciu obiektowym UML, wzorce projektowe i Java Helion 2011
  \item 2. Keeling Michael,Zostań architektem oprogramowania, PWN 2019
  \item 3. Piotr Gaczkowski, Adrian Ostrowski, Architektura oprogramowania bez tajemnic, Helion 20224.
  \item 4. Arnon Axelrod, Automatyzacja testów. Kompletny przewodnik dla testerów oprogramowaniaPWN 2022
  \item 5. Strony domowe narzędzi i technologii. Źródła internetowe.
\end{itemize}

\end{document}
