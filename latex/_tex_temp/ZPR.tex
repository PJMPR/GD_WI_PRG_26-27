% ===========================================================
%  Sylabus: Zarządzanie projektem informatycznym (ZPR)
% ===========================================================
\documentclass[12pt, a4paper]{article}

\usepackage[T1]{fontenc}
\usepackage[utf8]{inputenc}
\usepackage[polish]{babel}
\usepackage{lmodern}
\usepackage{microtype}
\usepackage[a4paper, top=2.5cm, bottom=2.5cm, left=2.5cm, right=2.5cm]{geometry}
\usepackage{xcolor}
\usepackage{graphicx}
\usepackage{booktabs}
\usepackage{tabularx}
\usepackage{longtable}
\usepackage{multirow}
\usepackage{array}
\usepackage{colortbl}
\usepackage{enumitem}
\usepackage{fancyhdr}
\usepackage{titlesec}
\usepackage{mdframed}
\usepackage[colorlinks=true, linkcolor=red!70!black, urlcolor=red!70!black]{hyperref}
\usepackage{eso-pic}
\usepackage{tikz}

\definecolor{pjatkRed}{RGB}{180,0,0}
\definecolor{pjatkGray}{RGB}{80,80,80}
\definecolor{pjatkLightGray}{RGB}{245,245,245}
\definecolor{tableHeader}{RGB}{220,220,220}

\pagestyle{fancy}\fancyhf{}
\renewcommand{\headrulewidth}{0.4pt}
\renewcommand{\footrulewidth}{0.4pt}
\fancyhead[L]{\small\textcolor{pjatkGray}{PJATK -- Filia w Gdańsku \textbar\ Informatyka}}
\fancyhead[R]{\small\textcolor{pjatkGray}{Sylabus: ZPR}}
\fancyfoot[C]{\small\thepage}

\titleformat{\section}{\large\bfseries\color{pjatkRed}}{\thesection.}{0.5em}{}
  [\color{pjatkRed}\rule{\linewidth}{0.8pt}]
\setlist{noitemsep, topsep=3pt, parsep=2pt}

\newmdenv[linecolor=pjatkRed, linewidth=1.2pt, backgroundcolor=pjatkLightGray,
  innerleftmargin=10pt, innerrightmargin=10pt, innertopmargin=8pt,
  innerbottommargin=8pt, roundcorner=4pt]{infobox}

\begin{document}

\AddToShipoutPictureBG{%
  \begin{tikzpicture}[remember picture, overlay]
    \node[opacity=0.5] at (current page.center) {%
      \includegraphics[width=14cm]{C:/Users/adamu/WebstormProjects/pj-studies/latex/PJATK_pl_sygnet_transparent-eps-converted-to}%
    };
  \end{tikzpicture}%
}

\begin{center}
  \includegraphics[height=2cm]{C:/Users/adamu/WebstormProjects/pj-studies/latex/PJATK_pl_poziom_1}\\[0.8cm]
  {\LARGE\bfseries\color{pjatkRed} SYLABUS PRZEDMIOTU}\\[0.8cm]
\end{center}

\begin{infobox}
\begin{tabularx}{\textwidth}{@{}lX@{}}
  \textbf{Nazwa przedmiotu:}  & {\bfseries Zarządzanie projektem informatycznym} \\[3pt]
  \textbf{Kod przedmiotu:}    & ZPR \\[3pt]
  \textbf{Kierunek / Profil:} & Informatyka / praktyczny \\[3pt]
  \textbf{Tryb studiów:}      & niestacjonarny \\[3pt]
  \textbf{Rok / Semestr:}     & 4 / 8 \\[3pt]
  \textbf{Charakter:}         & obieralny \\[3pt]
  \textbf{Odpowiedzialny:}    & mgr. Aleksandr Polin (alex.polin@pjwstk.edu.pl) \\[3pt]
  \textbf{Wersja z dnia:}     & 19.02.2026 \\
\end{tabularx}
\end{infobox}

\vspace{1cm}

\section{Godziny zajęć i punkty ECTS}

\begin{center}
\begin{tabular}{|>{\centering\arraybackslash}p{2.0cm}
                |>{\centering\arraybackslash}p{2.0cm}
                |>{\centering\arraybackslash}p{2.0cm}
                |>{\centering\arraybackslash}p{2.4cm}
                |>{\centering\arraybackslash}p{2.4cm}
                |>{\centering\arraybackslash}p{2.0cm}
                |>{\centering\arraybackslash}p{1.4cm}|}
\hline
\rowcolor{tableHeader}
\textbf{Wykłady} & \textbf{Ćwiczenia} & \textbf{Laboratorium} &
\textbf{Z prowadzącym} & \textbf{Praca własna} & \textbf{Łącznie} & \textbf{ECTS} \\
\hline
8 h & --- & 8 h & 16 h & 34 h & 50 h & \textbf{2} \\
\hline
\end{tabular}
\end{center}

\section{Forma zajęć}

\begin{tabular}{ll}
  \hline
  \textbf{Forma zajęć} & \textbf{Sposób zaliczenia} \\
  \hline
  Ćwiczenia & Zaliczenie \\
  Wykład & Nieoceniany \\
  \hline
\end{tabular}

\section{Cel dydaktyczny}

Zapoznanie studentów z kluczowymi koncepcjami i praktykami zarządzania projektami informatycznymi. Wyposażenie studentów w umiejętności planowania, organizowania, kierowania i kontroli projektów IT. Zrozumienie specyfiki zarządzania zespołami IT, motywacji i komunikacji.

\section{Przedmioty wprowadzające}

\begin{tabularx}{\textwidth}{lX}
  \hline
  \textbf{Przedmiot} & \textbf{Wymagane zagadnienia} \\
  \hline
  Znajomość podstawowych pojęć z zakresu inżynierii oprogramowania, cyklu życia oprogramowania. & Umiejętność pracy w zespole. \\
  \hline
\end{tabularx}

\section{Treści programowe}

\begin{enumerate}
  \item Wprowadzenie do zarządzania projektami IT
  \item Definicje, cele i zakres zarządzania projektami.
  \item Cykl życia projektu IT.
  \item Case study: Denver International Airport Baggage System - analiza błędów i wnioski.
  \item Metodologie zarządzania projektami
  \item Waterfall, Agile, Scrum, Kanban - omówienie, porównanie, zastosowania.
  \item Scrum vs. Scream - rozpoznawanie i unikanie patologii w zarządzaniu.
  \item Motywacja i zarządzanie zespołami IT
  \item Teorie motywacji, rola motywacji wewnętrznej i zewnętrznej.
  \item Algorytmiczne vs. Heurystyczne zadania - wpływ na motywację.
  \item Budowanie efektywnych zespołów, role i odpowiedzialności.
  \item Planowanie projektu
  \item Etapy planowania projektu, analiza sytuacji, SMART goals, WBS.
  \item Identyfikacja i analiza ryzyka.
  \item Case study: Manhattan Project vs. Maginot Line - dobre i złe praktyki planowania.
  \item Ćwiczenia: planowanie projektu z użyciem "Ninja Planning Card".
  \item Realizacja i kontrola projektu
  \item Zarządzanie czasem, budżetem i zasobami.
  \item Monitoring postępów projektu, raportowanie, analiza odchyleń.
  \item Zarządzanie zmianą, kontrola jakości.
  \item Zamykanie projektu
  \item Ocena projektu, dokumentacja, lessons learned.
  \item Prezentacja wyników projektu.
  \item Dodatkowe zagadnienia
  \item PRINCE2, Six Sigma - wprowadzenie.
  \item Teorie X, Y, Z - style zarządzania.
  \item Cargo Cults - unikanie bezmyślnego naśladowania.
  \item Etyka w zarządzaniu projektami IT.
\end{enumerate}

\section{Efekty kształcenia}

\subsection*{Wiedza}
\begin{itemize}
  \item Zna i rozumie zaawansowane pojęcia z zakresu zagadnień inżynierii oprogramowania, standardów i kształtu cykli wytwórczych oraz ewolucji oprogramowania.
  \item Zna i rozumie pojęcia z zakresu planowania przedsięwzięcia informatycznego, wstępnej oceny ekonomicznej, aspektów społecznych oraz analizy wykonalności.
  \item Zna i rozumie podstawowe problemy etyczne, społeczne i zawodowe informatyki, rozumie odpowiedzialność związaną z działalnością w obszarze informatyki.
\end{itemize}

\subsection*{Umiejętności}
\begin{itemize}
  \item Potrafi pracować w zespole, oszacować czas i koszty realizacji zadania, planować i realizować harmonogram prac.
  \item Potrafi zaplanować i wytworzyć podstawowe dokumenty związane z realizacją projektu informatycznego.
\end{itemize}

\subsection*{Kompetencje społeczne}
\begin{itemize}
  \item Jest gotów do współdziałań i współpracy w zespole, przyjmując różne role.
  \item Jest gotów do działania w sposób przedsiębiorczy.
\end{itemize}

\section{Kryteria oceny}

\begin{itemize}
  \item Wykład z elementami dyskusji,
  \item prezentacja multimedialna,
  \item case study,
  \item wykład zaproszony.
  \item dyskusja,
  \item burza mózgów,
  \item rozwiązywanie zadań,
  \item analiza przypadków,
  \item prezentacje.
  \item Zaliczenie
  \item Kryteria oceny
  \item Studenci tworzą i prezentują projekt planu zarządzania projektem informatycznym, wybranego przez grupę.
  \item Oceniane są: kompletność planu, zastosowanie teorii z wykładów, realizm, innowacyjność, jakość prezentacji.
  \item Ocena grupowa i indywidualna (uwzględniająca wkład).
  \item Nie dotyczy.
\end{itemize}

\section{Metody dydaktyczne}

Wykład, laboratoria, praca własna studenta.

\section{Literatura}

\textbf{Podstawowa:}
\begin{itemize}
  \item Dąbrowski, M. Wieczne opóźnienie. Zarządzanie projektami IT. OnePress, 2021.
  \item Schwalbe, Kathy. Information Technology Project Management. 9th ed. Cengage Learning, 2019
  \item Kisielnicki, J., Zarządzanie projektami, Nieoczywiste
\end{itemize}

\textbf{Uzupełniająca:}
\begin{itemize}
  \item PMI. A Guide to the Project Management Body of Knowledge (PMBOK® Guide). 7th ed. Project Management Institute, 2021.
  \item Sutherland, Jeff. Scrum: The Art of Doing Twice the Work in Half the Time. Crown Business, 2014.
\end{itemize}

\end{document}
