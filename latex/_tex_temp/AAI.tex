% ===========================================================
%  Sylabus: Zastosowanie sztucznej inteligencji ()
% ===========================================================
\documentclass[12pt, a4paper]{article}

\usepackage[T1]{fontenc}
\usepackage[utf8]{inputenc}
\usepackage[polish]{babel}
\usepackage{lmodern}
\usepackage{microtype}
\usepackage[a4paper, top=2.5cm, bottom=2.5cm, left=2.5cm, right=2.5cm]{geometry}
\usepackage{xcolor}
\usepackage{graphicx}
\usepackage{booktabs}
\usepackage{tabularx}
\usepackage{longtable}
\usepackage{multirow}
\usepackage{array}
\usepackage{colortbl}
\usepackage{enumitem}
\usepackage{fancyhdr}
\usepackage{titlesec}
\usepackage{mdframed}
\usepackage[colorlinks=true, linkcolor=red!70!black, urlcolor=red!70!black]{hyperref}
\usepackage{eso-pic}
\usepackage{tikz}

\definecolor{pjatkRed}{RGB}{180,0,0}
\definecolor{pjatkGray}{RGB}{80,80,80}
\definecolor{pjatkLightGray}{RGB}{245,245,245}
\definecolor{tableHeader}{RGB}{220,220,220}

\pagestyle{fancy}\fancyhf{}
\renewcommand{\headrulewidth}{0.4pt}
\renewcommand{\footrulewidth}{0.4pt}
\fancyhead[L]{\small\textcolor{pjatkGray}{PJATK -- Filia w Gdańsku \textbar\ Informatyka}}
\fancyhead[R]{\small\textcolor{pjatkGray}{Sylabus: }}
\fancyfoot[C]{\small\thepage}

\titleformat{\section}{\large\bfseries\color{pjatkRed}}{\thesection.}{0.5em}{}
  [\color{pjatkRed}\rule{\linewidth}{0.8pt}]
\setlist{noitemsep, topsep=3pt, parsep=2pt}

\newmdenv[linecolor=pjatkRed, linewidth=1.2pt, backgroundcolor=pjatkLightGray,
  innerleftmargin=10pt, innerrightmargin=10pt, innertopmargin=8pt,
  innerbottommargin=8pt, roundcorner=4pt]{infobox}

\begin{document}

\AddToShipoutPictureBG{%
  \begin{tikzpicture}[remember picture, overlay]
    \node[opacity=0.5] at (current page.center) {%
      \includegraphics[width=14cm]{C:/Users/adamu/WebstormProjects/pj-studies/latex/PJATK_pl_sygnet_transparent-eps-converted-to}%
    };
  \end{tikzpicture}%
}

\begin{center}
  \includegraphics[height=2cm]{C:/Users/adamu/WebstormProjects/pj-studies/latex/PJATK_pl_poziom_1}\\[0.8cm]
  {\LARGE\bfseries\color{pjatkRed} SYLABUS PRZEDMIOTU}\\[0.8cm]
\end{center}

\begin{infobox}
\begin{tabularx}{\textwidth}{@{}lX@{}}
  \textbf{Nazwa przedmiotu:}  & {\bfseries Zastosowanie sztucznej inteligencji} \\[3pt]
  \textbf{Kod przedmiotu:}    &  \\[3pt]
  \textbf{Kierunek / Profil:} & Informatyka / praktyczny \\[3pt]
  \textbf{Tryb studiów:}      & stacjonarny \\[3pt]
  \textbf{Rok / Semestr:}     & 4 / 7 \\[3pt]
  \textbf{Charakter:}         & obowiązkowy \\[3pt]
  \textbf{Odpowiedzialny:}    & Dr Tadeusz Puźniakowski \\[3pt]
  \textbf{Wersja z dnia:}     & 19.02.2026 \\
\end{tabularx}
\end{infobox}

\vspace{1cm}

\section{Godziny zajęć i punkty ECTS}

\begin{center}
\begin{tabular}{|>{\centering\arraybackslash}p{2.0cm}
                |>{\centering\arraybackslash}p{2.0cm}
                |>{\centering\arraybackslash}p{2.0cm}
                |>{\centering\arraybackslash}p{2.4cm}
                |>{\centering\arraybackslash}p{2.4cm}
                |>{\centering\arraybackslash}p{2.0cm}
                |>{\centering\arraybackslash}p{1.4cm}|}
\hline
\rowcolor{tableHeader}
\textbf{Wykłady} & \textbf{Ćwiczenia} & \textbf{Laboratorium} &
\textbf{Z prowadzącym} & \textbf{Praca własna} & \textbf{Łącznie} & \textbf{ECTS} \\
\hline
30 h & 30 h & --- & 60 h & 65 h & 125 h & \textbf{5} \\
\hline
\end{tabular}
\end{center}

\section{Forma zajęć}

\begin{tabular}{ll}
  \hline
  \textbf{Forma zajęć} & \textbf{Sposób zaliczenia} \\
  \hline
  Laboratorium & Zaliczenie z oceną \\
  Wykład & Egzamin \\
  \hline
\end{tabular}

\section{Cel dydaktyczny}

Celem przedmiotu jest zapoznanie studentów z algorytmami sztucznej inteligencji, wykorzystywanymi aktualnie w badaniach naukowych i przemyśle, w szczególności zrozumienie i implementacja architektur sztucznych sieci neuronywch wykorzystywanych w analizie obrazu, analizie danych danych tabelarycznych i przetwarzaniu języka naturalnego. Do celów pośrednich przedmiotu należy również nauka wyszukiwania informacji na temat najnowszych osiągnięć z dziedziny sztucznej inteligencji oraz otwarty zbiorów danych przydatnych w analizie danego zagadnienia.

\section{Przedmioty wprowadzające}

\begin{tabularx}{\textwidth}{lX}
  \hline
  \textbf{Przedmiot} & \textbf{Wymagane zagadnienia} \\
  \hline
  Machine Learning & Znajomość budowy podstawowych architektur sztucznych sieci neuronowych \\
  Znajomość podstawowych funkcji aktywacji wykorzystywanych w sztucznych sieciach neuronowych & Znajomość podstawowych architektur splotowych sieci neuronowych \\
  Znajomość podstawowych architektur rekurencyjnych sieci neuronowych & Znajomość wybranych algorytmów uczenia maszynowego (k-means, maszyna wektorów nośnych (SVM), drzewo decyzyjne, las losowy, klasyfikator Bayesowski) \\
  \hline
\end{tabularx}

\section{Treści programowe}

\begin{enumerate}
  \item Wprowadzenie do sztucznej inteligencji, Wprowadzenie do języka Python, implementacja klasycznych algorytmów uczenia maszynowego z wykorzystaniem biblioteki sklearn
  \item Wprowadzenie do sieci neuronowych. Wprowadzenie do biblioteki Pytorch Lightning i Pytorch, implementacja prostych architektur sieci neuronowych
  \item Regularyzacja sieci neuronowych. Regularyzacja sieci neuronowych z wykorzystaniem biblioteki Pytorch Lightning/Pytorch, implementacja algorytmów batch normalization i dropout
  \item Wprowadzenie do splotowych sieci neuronowych, architektury splotowych sieci neuronowych, algorytmy segmentacji i klasyfikacji obrazu
  \item Implementacja prostych splotowych sieci neuronowych z wykorzystaniem biblioteki Pytorch Lightning/Pytorch, implementacja generatorów danych obrazowych, implementacja zaawansowanych splotowych sieci neuronowych z wykorzystaniem biblioteki Pytorch Lightning/Pytorch, wykorzystanie transferu wiedzy z modeli dostępnych w TIMM, implementacja algorytmów segmentacji i klasyfikacji obrazów z wykorzystaniem biblioteki Pytorch Lightning/Pytorch
  \item Wprowadzenie do przetwarzania języka naturalnego, rekurencyjne sieci neuronowe, architektury typy transformer i mechanizm atencji w sieciach neuronowych, modelowanie języka naturalnego, transfer wiedzy w przetwarzaniu języka naturalnego
  \item Wykorzystanie wektorów słów używając bibliotek Gensim i Flair, wizualizacja wektorów słów z wykorzystaniem algorytmu t-SNE, implementacja rekurencyjnych sieci neuronowych z wykorzystaniem biblioteki Pytorch Lightning/Pytorch w szczególności sieci typu BiLSTM w zagadnieniach klasyfikacji i oznaczania tekstu, implementacja sieci typu transformer z wykorzystaniem biblioteki Pytorch Lightning/Pytorch. Implementacja pre-trenowanych modeli języka z wykorzystaniem biblioteki Huggingface Transformers
  \item Optymalizacja hiper parametrów sieci neuronowych, generacyjne sieci neuronowe i wyjaśnialna sztuczna inteligencja
  \item Implementacja algorytmów automatycznego doboru hiperparametów sieci neuronowych. Implementacja generacyjnych sieci neuronowych z wykorzystaniem biblioteki Pytorch Lightning/Pytorch. Implementacja algorytmów wyjaśniania predykcji sieci neuronowych z wykorzystaniem bibliotek Dalex i Lime
  \item Kapsułowe sieci neuronowe i analiza obrazu w wykorzystaniem architektury typu transformer
  \item Implementacja kapsułowych sieci neuronowych i modelu Vision Transformer z wykorzystaniem biblioteki Pytorch Lightning/Pytorch
  \item Metody nauki nienadzorowanej jako mechanizm pretreningu sieci neuronowych
\end{enumerate}

\section{Efekty kształcenia}

\subsection*{Wiedza}
\begin{itemize}
  \item Student zna i rozumie opis matematyczny sztucznych sieci neuronowych z wykorzystaniem elementów algebry. Zna i rozumie algorytmy optymalizacji funkcji błędu z wykorzystaniem metod gradientowych.
  \item Student zna i rozumie wykorzystywane w przemyśle i badaniach naukowych algorytmy uczenia maszynowego i modele sieci neuronowych
  \item Zna sposoby  planowania i przeprowadzania eksperymentów w wykorzystaniem sieci neuronowych oraz ocenić ich wyniki.
  \item Student zna i rozumie pojęcia z zastosowania z zakresu AI od strony oprogramowania.
\end{itemize}

\subsection*{Umiejętności}
\begin{itemize}
  \item Student potrafi zastosować aparat matematyczny w kontekście AI.
  \item Student potrafi wykorzystać odpowiednie narzędzia sztucznych sieci neuronowych celem implemantacji algorytmów AI.
\end{itemize}

\section{Kryteria oceny}

\begin{itemize}
  \item Wykład z elementami dyskusji z prezentacją multimedialną
  \item Przedstawienie implementacji zagadnień poruszanych na wykładzie z wykorzystaniem interaktywnego środowiska wykonawczego języka python - Google Colab Notebooks
  \item Zadania wykonywane samodzielnie przez studentów z wykorzystaniem interaktywnego środowiska wykonawczego języka python - Google Colab Notebooks
  \item Kryteria oceny
  \item Ocena z projektu i aktywności studenta na zajęciach.
\end{itemize}

\section{Metody dydaktyczne}

Wykład, laboratoria, praca własna studenta.

\section{Literatura}

\textbf{Podstawowa:}
\begin{itemize}
  \item Goodfellow, Ian, Yoshua Bengio, and Aaron Courville. Deep learning. MIT press, 2016
  \item Yang, Qiang, et al. Transfer learning. Cambridge University Press, 2020
\end{itemize}

\textbf{Uzupełniająca:}
\begin{itemize}
  \item Goldberg, Yoav, and Graeme Hirst. "Neural Network Methods in Natural Language Processing. Morgan \& Claypool Publishers(2017)."
\end{itemize}

\end{document}
