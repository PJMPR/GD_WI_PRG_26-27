% ===========================================================
%  Sylabus: Symulacje 3D (S3D)
% ===========================================================
\documentclass[12pt, a4paper]{article}

\usepackage[T1]{fontenc}
\usepackage[utf8]{inputenc}
\usepackage[polish]{babel}
\usepackage{lmodern}
\usepackage{microtype}
\usepackage[a4paper, top=2.5cm, bottom=2.5cm, left=2.5cm, right=2.5cm]{geometry}
\usepackage{xcolor}
\usepackage{graphicx}
\usepackage{booktabs}
\usepackage{tabularx}
\usepackage{longtable}
\usepackage{multirow}
\usepackage{array}
\usepackage{colortbl}
\usepackage{enumitem}
\usepackage{fancyhdr}
\usepackage{titlesec}
\usepackage{mdframed}
\usepackage[colorlinks=true, linkcolor=red!70!black, urlcolor=red!70!black]{hyperref}
\usepackage{eso-pic}
\usepackage{tikz}

\definecolor{pjatkRed}{RGB}{180,0,0}
\definecolor{pjatkGray}{RGB}{80,80,80}
\definecolor{pjatkLightGray}{RGB}{245,245,245}
\definecolor{tableHeader}{RGB}{220,220,220}

\pagestyle{fancy}\fancyhf{}
\renewcommand{\headrulewidth}{0.4pt}
\renewcommand{\footrulewidth}{0.4pt}
\fancyhead[L]{\small\textcolor{pjatkGray}{PJATK -- Filia w Gdańsku \textbar\ Informatyka}}
\fancyhead[R]{\small\textcolor{pjatkGray}{Sylabus: S3D}}
\fancyfoot[C]{\small\thepage}

\titleformat{\section}{\large\bfseries\color{pjatkRed}}{\thesection.}{0.5em}{}
  [\color{pjatkRed}\rule{\linewidth}{0.8pt}]
\setlist{noitemsep, topsep=3pt, parsep=2pt}

\newmdenv[linecolor=pjatkRed, linewidth=1.2pt, backgroundcolor=pjatkLightGray,
  innerleftmargin=10pt, innerrightmargin=10pt, innertopmargin=8pt,
  innerbottommargin=8pt, roundcorner=4pt]{infobox}

\begin{document}

\AddToShipoutPictureBG{%
  \begin{tikzpicture}[remember picture, overlay]
    \node[opacity=0.5] at (current page.center) {%
      \includegraphics[width=14cm]{C:/Users/adamu/WebstormProjects/pj-studies/latex/PJATK_pl_sygnet_transparent-eps-converted-to}%
    };
  \end{tikzpicture}%
}

\begin{center}
  \includegraphics[height=2cm]{C:/Users/adamu/WebstormProjects/pj-studies/latex/PJATK_pl_poziom_1}\\[0.8cm]
  {\LARGE\bfseries\color{pjatkRed} SYLABUS PRZEDMIOTU}\\[0.8cm]
\end{center}

\begin{infobox}
\begin{tabularx}{\textwidth}{@{}lX@{}}
  \textbf{Nazwa przedmiotu:}  & {\bfseries Symulacje 3D} \\[3pt]
  \textbf{Kod przedmiotu:}    & S3D \\[3pt]
  \textbf{Kierunek / Profil:} & Informatyka / praktyczny \\[3pt]
  \textbf{Tryb studiów:}      & stacjonarny \\[3pt]
  \textbf{Rok / Semestr:}     & 3 / 5 \\[3pt]
  \textbf{Charakter:}         & obowiązkowy \\[3pt]
  \textbf{Odpowiedzialny:}    & Dr Piotr Arłukowicz \\[3pt]
  \textbf{Wersja z dnia:}     & 19.02.2026 \\
\end{tabularx}
\end{infobox}

\vspace{1cm}

\section{Godziny zajęć i punkty ECTS}

\begin{center}
\begin{tabular}{|>{\centering\arraybackslash}p{2.0cm}
                |>{\centering\arraybackslash}p{2.0cm}
                |>{\centering\arraybackslash}p{2.0cm}
                |>{\centering\arraybackslash}p{2.4cm}
                |>{\centering\arraybackslash}p{2.4cm}
                |>{\centering\arraybackslash}p{2.0cm}
                |>{\centering\arraybackslash}p{1.4cm}|}
\hline
\rowcolor{tableHeader}
\textbf{Wykłady} & \textbf{Ćwiczenia} & \textbf{Laboratorium} &
\textbf{Z prowadzącym} & \textbf{Praca własna} & \textbf{Łącznie} & \textbf{ECTS} \\
\hline
30 h & 30 h & --- & 60 h & 65 h & 125 h & \textbf{5} \\
\hline
\end{tabular}
\end{center}

\section{Forma zajęć}

\begin{tabular}{ll}
  \hline
  \textbf{Forma zajęć} & \textbf{Sposób zaliczenia} \\
  \hline
  Laboratorium & Zaliczenie z oceną \\
  Wykład & Egzamin \\
  \hline
\end{tabular}

\section{Cel dydaktyczny}

Tematyka przedmiotu realizowana jest w oparciu o silniki symulacji dostępne w oprogramowaniu Open-Source (Blender). Poruszane zagadnienia to symulacje układów z bezwładnością i tarciem w modelu brył sztywnych, symulacje brył elastycznych z deformowalną powierzchnią, symulacje odzieży i tkanin, symulacje cząsteczkowe w modelach Newtonian, Fluid, Boids i Keyed, oraz symulacje ognia, dymu, wybuchów, iskier, kurzu, mgły i innych zjawisk wolumetrycznych. Dodatkowo omówione będą zagadnienia z dziedziny Dynamic Paint, gdzie układ symulacyjny wpływa na system symulowany, przykładowo: symulowanie fal na wodzie, zniszczeń, deformacji, odbić, zmiany gęstości ośrodka, itp. Przedmiot rozszerza i ubogaca wiedzę wyniesioną z przedmiotu Grafika Komputerowa.

\section{Przedmioty wprowadzające}

\begin{tabularx}{\textwidth}{lX}
  \hline
  \textbf{Przedmiot} & \textbf{Wymagane zagadnienia} \\
  \hline
  Grafika komputerowa & Znajomość Blendera \\
  \hline
\end{tabularx}

\section{Treści programowe}

\begin{enumerate}
  \item 1Symulacje Rigid-BodyTworzenie obiektów Rigid-Body i symulowanie ich zderzeń i ewolucji w systemie. Sprawdzanie, kto zbuduje najbardziej wytrzymałą konstrukcję.
  \item 2Wykorzystanie driverów i motorów do sterowania więzami sztywnymi i łamalnymiĆwiczenia z burzenia budynków.
  \item 3Symulacje Soft-BodyBryły elastyczne o różnej sztywności, bryły odkształcające się po kolizji.
  \item 4Zaawansowane ustawienia symulacji soft-bodyTworzenie kolizji samochodów.
  \item 5Symulacje ClothSymulowanie odzieży - skórzanej, gumowej, jedwabnej, dżinsowej i innej. Kolizje wewnętrzne
  \item 6Symulowanie odzieży na modelu ludzkim.
  \item Od prostych ćwiczeń takich jak flaga do ubrania ludzi.
  \item 7Symulacje cząsteczkoweSymulowanie w modelach Newtonian, Boids, Fluid i Keyed. Animacje cząsteczkowe.
  \item 8Tworzenie animacji napisów i elementów graficznych z cząstek.Animacja wybuchających sztucznych ogni.
  \item 9Symulacje trawy, włosów, futra i układów podobnych do nich.Włosy czesane, dynamicznie reagujące na obiekty na scenie, kolidujące ze sobą, edycja i układanie włosów.
  \item 10Tworzenie trawnika lub pola zboża.Animacja trawy lub kłosów zboża w wielkiej skali.
  \item 11Symulacje cieczy. Ciecze gęste, miód, czekolada, syrop. Dodawanie i usuwanie cieczy.Ćwiczenia z budowania scen w których występują symulacje płynów.
  \item 12Symulacje cieczy ze zmienną gęstością.Efekt kropli wpadającej do wody.
  \item 13Symulacje dymu, ognia, wybuchy.Wysadzanie modelu uczelni lub inne ćwiczenia.
  \item 14Zaawansowane przypadki symulacji ognia i dymy.Ćwiczenie spalania kartki papieru.
\end{enumerate}

\section{Efekty kształcenia}

\subsection*{Wiedza}
\begin{itemize}
  \item Student zna i rozumie zasady projektowania aplikacji graficznych i prezentacji multimedialnych, jak też współczesne techniki i narzędzia graficzne
\end{itemize}

\subsection*{Umiejętności}
\begin{itemize}
  \item Student potrafi  porozumiewać się w języku polskim i angielskim  w środowisku zawodowy.
  \item Student potrafi  wykonywać aplikacje graficzne za pomocą Blendera.
\end{itemize}

\section{Kryteria oceny}

\begin{itemize}
  \item wykład z elementami dyskusji z prezentacją multimedialną
  \item praca indywidualna przy komputerze
  \item Kryteria oceny
  \item Projekty (oceny cząstkowe otrzymywane w trakcie semestru)
  \item Wykonanie i obrona pracy zaliczeniowej.
\end{itemize}

\section{Metody dydaktyczne}

Wykład, laboratoria, praca własna studenta.

\section{Literatura}

\textbf{Podstawowa:}
\begin{itemize}
  \item Brak danych.
\end{itemize}

\textbf{Uzupełniająca:}
\begin{itemize}
  \item każda książka o Blenderze w wersji 4.2
\end{itemize}

\end{document}
