% ===========================================================
%  Sylabus: Bezpieczeństwo systemów informacyjnych (BSI) (BSI)
% ===========================================================
\documentclass[12pt, a4paper]{article}

\usepackage[T1]{fontenc}
\usepackage[utf8]{inputenc}
\usepackage[polish]{babel}
\usepackage{lmodern}
\usepackage{microtype}
\usepackage[a4paper, top=2.5cm, bottom=2.5cm, left=2.5cm, right=2.5cm]{geometry}
\usepackage{xcolor}
\usepackage{graphicx}
\usepackage{booktabs}
\usepackage{tabularx}
\usepackage{longtable}
\usepackage{multirow}
\usepackage{array}
\usepackage{colortbl}
\usepackage{enumitem}
\usepackage{fancyhdr}
\usepackage{titlesec}
\usepackage{mdframed}
\usepackage[colorlinks=true, linkcolor=red!70!black, urlcolor=red!70!black]{hyperref}
\usepackage{eso-pic}
\usepackage{tikz}

\definecolor{pjatkRed}{RGB}{180,0,0}
\definecolor{pjatkGray}{RGB}{80,80,80}
\definecolor{pjatkLightGray}{RGB}{245,245,245}
\definecolor{tableHeader}{RGB}{220,220,220}

\pagestyle{fancy}\fancyhf{}
\renewcommand{\headrulewidth}{0.4pt}
\renewcommand{\footrulewidth}{0.4pt}
\fancyhead[L]{\small\textcolor{pjatkGray}{PJATK -- Filia w Gdańsku \textbar\ Informatyka}}
\fancyhead[R]{\small\textcolor{pjatkGray}{Sylabus: BSI}}
\fancyfoot[C]{\small\thepage}

\titleformat{\section}{\large\bfseries\color{pjatkRed}}{\thesection.}{0.5em}{}
  [\color{pjatkRed}\rule{\linewidth}{0.8pt}]
\setlist{noitemsep, topsep=3pt, parsep=2pt}

\newmdenv[linecolor=pjatkRed, linewidth=1.2pt, backgroundcolor=pjatkLightGray,
  innerleftmargin=10pt, innerrightmargin=10pt, innertopmargin=8pt,
  innerbottommargin=8pt, roundcorner=4pt]{infobox}

\begin{document}

\AddToShipoutPictureBG{%
  \begin{tikzpicture}[remember picture, overlay]
    \node[opacity=0.5] at (current page.center) {%
      \includegraphics[width=14cm]{C:/Users/adamu/WebstormProjects/pj-studies/latex/PJATK_pl_sygnet_transparent-eps-converted-to}%
    };
  \end{tikzpicture}%
}

\begin{center}
  \includegraphics[height=2cm]{C:/Users/adamu/WebstormProjects/pj-studies/latex/PJATK_pl_poziom_1}\\[0.8cm]
  {\LARGE\bfseries\color{pjatkRed} SYLABUS PRZEDMIOTU}\\[0.8cm]
\end{center}

\begin{infobox}
\begin{tabularx}{\textwidth}{@{}lX@{}}
  \textbf{Nazwa przedmiotu:}  & {\bfseries Bezpieczeństwo systemów informacyjnych (BSI)} \\[3pt]
  \textbf{Kod przedmiotu:}    & BSI \\[3pt]
  \textbf{Kierunek / Profil:} & Informatyka / praktyczny \\[3pt]
  \textbf{Tryb studiów:}      & niestacjonarny \\[3pt]
  \textbf{Rok / Semestr:}     & 3 / 5 \\[3pt]
  \textbf{Charakter:}         & obowiązkowy \\[3pt]
  \textbf{Odpowiedzialny:}    & dr Andrzej Bobyk \\[3pt]
  \textbf{Wersja z dnia:}     & 19.02.2026 \\
\end{tabularx}
\end{infobox}

\vspace{1cm}

\section{Godziny zajęć i punkty ECTS}

\begin{center}
\begin{tabular}{|>{\centering\arraybackslash}p{2.0cm}
                |>{\centering\arraybackslash}p{2.0cm}
                |>{\centering\arraybackslash}p{2.0cm}
                |>{\centering\arraybackslash}p{2.4cm}
                |>{\centering\arraybackslash}p{2.4cm}
                |>{\centering\arraybackslash}p{2.0cm}
                |>{\centering\arraybackslash}p{1.4cm}|}
\hline
\rowcolor{tableHeader}
\textbf{Wykłady} & \textbf{Ćwiczenia} & \textbf{Laboratorium} &
\textbf{Z prowadzącym} & \textbf{Praca własna} & \textbf{Łącznie} & \textbf{ECTS} \\
\hline
8 h & --- & 16 h & 24 h & 76 h & 100 h & \textbf{3} \\
\hline
\end{tabular}
\end{center}

\section{Forma zajęć}

\begin{tabular}{ll}
  \hline
  \textbf{Forma zajęć} & \textbf{Sposób zaliczenia} \\
  \hline
  Wykład & Nieoceniany \\
  \hline
\end{tabular}

\section{Cel dydaktyczny}

Celem przedmiotu jest zapoznanie słuchaczy z zasadami i metodami ochrony informacji w systemach i sieciach komputerowych.

\section{Przedmioty wprowadzające}

\begin{tabularx}{\textwidth}{lX}
  \hline
  \textbf{Przedmiot} & \textbf{Wymagane zagadnienia} \\
  \hline
  ALG, SKOA & Podstawy algebry, wiedza z zakresu systemów operacyjnych i sieci komputerowych. \\
  \hline
\end{tabularx}

\section{Treści programowe}

\begin{enumerate}
  \item Polityka bezpieczeństwa systemów informatycznych: etapy tworzenia struktur bezpieczeństwa, zasady efektywnej strategii i polityki bezpieczeństwa. Usługi związane z ochroną informacji, kategorie zagrożeń systemów informatycznych.
  \item Elementy kryptografii: podstawowe techniki szyfrowania, zastosowania technik szyfrowania, algorytmy symetryczne (DES, 3DES, IDEA).Algorytmy asymetryczne: (RSA, El-Gamal), uwierzytelnianie i sygnatury cyfrowe. Podpis elektroniczny.
  \item Bezpieczeństwo w sieciach: usługi i protokoły kryptograficzne (SSL, SSH, Kerberos), bezpieczeństwo poczty elektronicznej; protokoły PGP i PEM, zarządzanie kluczami, certyfikaty.
  \item Praktyka ochrony danych w systemach komputerowych: programy złośliwe (wirusy, robaki, konie trojańskie), archiwizacja danych – procedury, metody, programy.
\end{enumerate}

\section{Efekty kształcenia}

\subsection*{Wiedza}
\begin{itemize}
  \item Student zna i rozumiepodstawowe zagadnienia i pojęcia kryptografii i kryptoanalizy oraz zasady działania algorytmów kryptograficznych.
  \item Student zna problematykę bezpieczeństwa w nowoczesnych systemach informatycznych; rozumie pojęcia związane z poufnością, integralnością, dostępnością, uwierzytelnianiem, autoryzacją i ewidencjonowaniem.
  \item Student zna i rozumie pojęcia związane z implementowaniem zapór ogniowych, ochrony przed wtargnięciami do sieci, systemów kryptograficznych, implementacji wirtualnych sieci prywatnych i zarządzania bezpiecznymi sieciami komputerowymi; zna metody zabezpieczania urządzeń sieciowych i systemów komputerowych.
\end{itemize}

\subsection*{Umiejętności}
\begin{itemize}
  \item Brak danych.
\end{itemize}

\section{Kryteria oceny}

\begin{itemize}
  \item Ćwiczenia / Laboratorium:
  \item rozwiązywanie zadań
  \item Ćwiczenia/Laboratorium
  \item Kryteria oceny
  \item Ćwiczenia/Laboratorium
  \item Student jest zobowiązany uzyskać ponad 50\% punktów możliwych do zdobycia w trakcie trwania semestru.
\end{itemize}

\section{Metody dydaktyczne}

Wykład, laboratoria, praca własna studenta.

\section{Literatura}

\textbf{Podstawowa:}
\begin{itemize}
  \item D. E. Robling Denning: Kryptografia i ochrona danych. WNT, Warszawa 1993.
  \item S. Garfinkel, G. Spafford: Bezpieczeństwo w Unixie i Internecie. RM, Warszawa 1997 (+A. Schwartz: Practical UNIX and Internet Security. 3rd Edition. O'Reilly Media 2003).
  \item W. Stallings: Ochrona danych w sieci i intersieci. W teorii i praktyce. WNT, Warszawa 1997.
  \item M. Kutyłowski, W.-B. Strothmann: Kryptografia. Teoria i praktyka zabezpieczania systemów komputerowych (wyd. 2). Read Me, Warszawa 1999.
  \item J.-P. Aumasson: Nowoczesna kryptografia. Praktyczne wprowadzenie do szyfrowania. PWN, Warszawa 2018.
  \item J. Andress: Podstawy bezpieczeństwa informacji. Praktyczne wprowadzenie. Helion, Gliwice 2021.
\end{itemize}

\textbf{Uzupełniająca:}
\begin{itemize}
  \item C. Adams, S. Lloyd: Podpis elektroniczny. Klucz publiczny. Robomatic, Warszawa 2002.
  \item R. Wobst: Kryptologia. Budowa i łamanie zabezpieczeń. RM, Warszawa, 2002.
  \item F. L. Bauer: Sekrety kryptografii. Helion, Gliwice 2002.
  \item D. R. Stinson: Kryptografia. WNT, Warszawa, 2005.
\end{itemize}

\end{document}
