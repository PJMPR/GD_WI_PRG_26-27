% ===========================================================
%  Sylabus: Interakcja Człowiek-Komputer (ICK)
% ===========================================================
\documentclass[12pt, a4paper]{article}

\usepackage[T1]{fontenc}
\usepackage[utf8]{inputenc}
\usepackage[polish]{babel}
\usepackage{lmodern}
\usepackage{microtype}
\usepackage[a4paper, top=2.5cm, bottom=2.5cm, left=2.5cm, right=2.5cm]{geometry}
\usepackage{xcolor}
\usepackage{graphicx}
\usepackage{booktabs}
\usepackage{tabularx}
\usepackage{longtable}
\usepackage{multirow}
\usepackage{array}
\usepackage{colortbl}
\usepackage{enumitem}
\usepackage{fancyhdr}
\usepackage{titlesec}
\usepackage{mdframed}
\usepackage[colorlinks=true, linkcolor=red!70!black, urlcolor=red!70!black]{hyperref}
\usepackage{eso-pic}
\usepackage{tikz}

\definecolor{pjatkRed}{RGB}{180,0,0}
\definecolor{pjatkGray}{RGB}{80,80,80}
\definecolor{pjatkLightGray}{RGB}{245,245,245}
\definecolor{tableHeader}{RGB}{220,220,220}

\pagestyle{fancy}\fancyhf{}
\renewcommand{\headrulewidth}{0.4pt}
\renewcommand{\footrulewidth}{0.4pt}
\fancyhead[L]{\small\textcolor{pjatkGray}{PJATK -- Filia w Gdańsku \textbar\ Informatyka}}
\fancyhead[R]{\small\textcolor{pjatkGray}{Sylabus: ICK}}
\fancyfoot[C]{\small\thepage}

\titleformat{\section}{\large\bfseries\color{pjatkRed}}{\thesection.}{0.5em}{}
  [\color{pjatkRed}\rule{\linewidth}{0.8pt}]
\setlist{noitemsep, topsep=3pt, parsep=2pt}

\newmdenv[linecolor=pjatkRed, linewidth=1.2pt, backgroundcolor=pjatkLightGray,
  innerleftmargin=10pt, innerrightmargin=10pt, innertopmargin=8pt,
  innerbottommargin=8pt, roundcorner=4pt]{infobox}

\begin{document}

\AddToShipoutPictureBG{%
  \begin{tikzpicture}[remember picture, overlay]
    \node[opacity=0.5] at (current page.center) {%
      \includegraphics[width=14cm]{C:/Users/adamu/WebstormProjects/pj-studies/latex/PJATK_pl_sygnet_transparent-eps-converted-to}%
    };
  \end{tikzpicture}%
}

\begin{center}
  \includegraphics[height=2cm]{C:/Users/adamu/WebstormProjects/pj-studies/latex/PJATK_pl_poziom_1}\\[0.8cm]
  {\LARGE\bfseries\color{pjatkRed} SYLABUS PRZEDMIOTU}\\[0.8cm]
\end{center}

\begin{infobox}
\begin{tabularx}{\textwidth}{@{}lX@{}}
  \textbf{Nazwa przedmiotu:}  & {\bfseries Interakcja Człowiek-Komputer} \\[3pt]
  \textbf{Kod przedmiotu:}    & ICK \\[3pt]
  \textbf{Kierunek / Profil:} & Informatyka / praktyczny \\[3pt]
  \textbf{Tryb studiów:}      & niestacjonarny \\[3pt]
  \textbf{Rok / Semestr:}     & 3 / 5 \\[3pt]
  \textbf{Charakter:}         & obowiązkowy \\[3pt]
  \textbf{Odpowiedzialny:}    &  \\[3pt]
  \textbf{Wersja z dnia:}     & 19.02.2026 \\
\end{tabularx}
\end{infobox}

\vspace{1cm}

\section{Godziny zajęć i punkty ECTS}

\begin{center}
\begin{tabular}{|>{\centering\arraybackslash}p{2.0cm}
                |>{\centering\arraybackslash}p{2.0cm}
                |>{\centering\arraybackslash}p{2.0cm}
                |>{\centering\arraybackslash}p{2.4cm}
                |>{\centering\arraybackslash}p{2.4cm}
                |>{\centering\arraybackslash}p{2.0cm}
                |>{\centering\arraybackslash}p{1.4cm}|}
\hline
\rowcolor{tableHeader}
\textbf{Wykłady} & \textbf{Ćwiczenia} & \textbf{Laboratorium} &
\textbf{Z prowadzącym} & \textbf{Praca własna} & \textbf{Łącznie} & \textbf{ECTS} \\
\hline
16 h & --- & 16 h & 32 h & 68 h & 100 h & \textbf{5} \\
\hline
\end{tabular}
\end{center}

\section{Forma zajęć}

\begin{tabular}{ll}
  \hline
  \textbf{Forma zajęć} & \textbf{Sposób zaliczenia} \\
  \hline
  Laboratorium & Zaliczenie z oceną \\
  Wykład & Nieoceniany \\
  \hline
\end{tabular}

\section{Cel dydaktyczny}

Celem przedmiotu jest zaznajomienie studentów z podstawowymi zagadnieniami projektowania interakcji człowieka z komputerem, tworzenia użytecznych interfejsów użytkownika oraz wykorzystania podejścia UCD (User-Centred Design) w projektowaniu, testowaniu i doskonaleniu użyteczności systemów informatycznych.

\section{Treści programowe}

\begin{enumerate}
  \item \#
  \item Wykład
  \item Laboratorium
  \item 1
  \item HCI – wprowadzenie, Charakterystyki użytkownika
  \item Bad Design – Identyfikacja błędów w interfejsach użytkownika
  \item 2
  \item Interfejs GUI – wytyczne i zasady projektowania
  \item Podział na grupy i wybór tematu dla projektu,  , analiza konkurencji, rynek docelowy i uzasadnienie biznesowe
  \item 3
  \item Współpraca z klientem/użytkownikiem  w projektach IT – określanie wymagań. Design Thinking (DT) w projektach IT.
  \item Generowanie pomysłów, wartościowanie pomysłów
  \item 4
  \item Poznawanie potrzeb użytkowników – potrzeby i oczekiwania
  \item Empatyzacja – wywiady na bazie kanw DT
  \item 5
  \item Interfejs WWW i mobilny – wymagania i zasady projektowania
  \item Persony, analiza potrzeb użytkowników
  \item 6
  \item Wizja produktu i projekt koncepcyjny
  \item Opracowanie historyjek użytkownika
  \item 7
  \item Historyjki użytkownika i ich mapowanie
  \item Mapowanie historyjek użytkownika
  \item 8
  \item Projekt interakcji – user flow, priorytety dla funkcjonalności
  \item User flow, BPMN
  \item 9
  \item Prototypy wstępne
  \item Prototypowanie Lo-Fi – prototypowanie papierowe cz. 2
  \item 10
  \item Opracowanie prototypów cyfrowych
  \item Prototypowanie Lo-Fi – prototypowanie papierowe cz. 2
  \item 11
  \item Testy użyteczności
  \item Prototypowanie cyfrowe (Hi-Fi)
  \item 12
  \item Pozyskiwanie, analizowanie i prezentowanie  danych z badań od użytkowników
  \item Prototypowanie cyfrowe (Hi-Fi)
  \item 13
  \item Interfejsy multimodalne - wirtualni agenci, interfejsy głosowe
  \item Scenariusze testowe
  \item 14
  \item Projektowanie interakcji z użytkownikiem dla usług cyfrowych
  \item Testy użyteczności
  \item 15
  \item Projektowanie interakcji dla rozwiązań typu „smart”
  \item Prezentacja wykonanych prototypów interfejsu użytkownika
\end{enumerate}

\section{Efekty kształcenia}

\subsection*{Wiedza}
\begin{itemize}
  \item Student zna i rozumie zasady projektowania interakcji użytkownik-system w klasycznych i zwinnych projektach informatycznych
\end{itemize}

\subsection*{Umiejętności}
\begin{itemize}
  \item Student potrafi stosować właściwe formy organizacji współpracy z użytkownikami podczas realizacji przedsięwzięć informatycznych
\end{itemize}

\subsection*{Kompetencje społeczne}
\begin{itemize}
  \item Student jest gotów współdziałania i współpracy w zespołach projektowych
\end{itemize}

\section{Kryteria oceny}

\begin{itemize}
  \item rozwiązywanie zadań
  \item Kryteria oceny
  \item Laboratorium/
  \item - ćwiczenia zawiązane z tematyką wykładu, opracowania obliczeniowe i koncepcyjne, ocena sporządzonego sprawozdania
  \item -  kolokwium pisemne
\end{itemize}

\section{Metody dydaktyczne}

Wykład, laboratoria, praca własna studenta.

\section{Literatura}

\textbf{Podstawowa:}
\begin{itemize}
  \item Sikorski M. (2017). Interakcja człowiek-komputer. Wyd. PJWSTK Warszawa.
\end{itemize}

\textbf{Uzupełniająca:}
\begin{itemize}
  \item Badura C. (2019). UXUI. Design Zoptymalizowany. Wyd. Helion.
\end{itemize}

\end{document}
