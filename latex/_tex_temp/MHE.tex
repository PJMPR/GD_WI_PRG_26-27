% ===========================================================
%  Sylabus: Metaheurystyki ()
% ===========================================================
\documentclass[12pt, a4paper]{article}

\usepackage[T1]{fontenc}
\usepackage[utf8]{inputenc}
\usepackage[polish]{babel}
\usepackage{lmodern}
\usepackage{microtype}
\usepackage[a4paper, top=2.5cm, bottom=2.5cm, left=2.5cm, right=2.5cm]{geometry}
\usepackage{xcolor}
\usepackage{graphicx}
\usepackage{booktabs}
\usepackage{tabularx}
\usepackage{longtable}
\usepackage{multirow}
\usepackage{array}
\usepackage{colortbl}
\usepackage{enumitem}
\usepackage{fancyhdr}
\usepackage{titlesec}
\usepackage{mdframed}
\usepackage[colorlinks=true, linkcolor=red!70!black, urlcolor=red!70!black]{hyperref}
\usepackage{eso-pic}
\usepackage{tikz}

\definecolor{pjatkRed}{RGB}{180,0,0}
\definecolor{pjatkGray}{RGB}{80,80,80}
\definecolor{pjatkLightGray}{RGB}{245,245,245}
\definecolor{tableHeader}{RGB}{220,220,220}

\pagestyle{fancy}\fancyhf{}
\renewcommand{\headrulewidth}{0.4pt}
\renewcommand{\footrulewidth}{0.4pt}
\fancyhead[L]{\small\textcolor{pjatkGray}{PJATK -- Filia w Gdańsku \textbar\ Informatyka}}
\fancyhead[R]{\small\textcolor{pjatkGray}{Sylabus: }}
\fancyfoot[C]{\small\thepage}

\titleformat{\section}{\large\bfseries\color{pjatkRed}}{\thesection.}{0.5em}{}
  [\color{pjatkRed}\rule{\linewidth}{0.8pt}]
\setlist{noitemsep, topsep=3pt, parsep=2pt}

\newmdenv[linecolor=pjatkRed, linewidth=1.2pt, backgroundcolor=pjatkLightGray,
  innerleftmargin=10pt, innerrightmargin=10pt, innertopmargin=8pt,
  innerbottommargin=8pt, roundcorner=4pt]{infobox}

\begin{document}

\AddToShipoutPictureBG{%
  \begin{tikzpicture}[remember picture, overlay]
    \node[opacity=0.5] at (current page.center) {%
      \includegraphics[width=14cm]{C:/Users/adamu/WebstormProjects/pj-studies/latex/PJATK_pl_sygnet_transparent-eps-converted-to}%
    };
  \end{tikzpicture}%
}

\begin{center}
  \includegraphics[height=2cm]{C:/Users/adamu/WebstormProjects/pj-studies/latex/PJATK_pl_poziom_1}\\[0.8cm]
  {\LARGE\bfseries\color{pjatkRed} SYLABUS PRZEDMIOTU}\\[0.8cm]
\end{center}

\begin{infobox}
\begin{tabularx}{\textwidth}{@{}lX@{}}
  \textbf{Nazwa przedmiotu:}  & {\bfseries Metaheurystyki} \\[3pt]
  \textbf{Kod przedmiotu:}    &  \\[3pt]
  \textbf{Kierunek / Profil:} & Informatyka / praktyczny \\[3pt]
  \textbf{Tryb studiów:}      & stacjonarny \\[3pt]
  \textbf{Rok / Semestr:}     & 3 / 5 \\[3pt]
  \textbf{Charakter:}         & obowiązkowy \\[3pt]
  \textbf{Odpowiedzialny:}    & dr Tadeusz Puźniakowski \\[3pt]
  \textbf{Wersja z dnia:}     & 19.02.2026 \\
\end{tabularx}
\end{infobox}

\vspace{1cm}

\section{Godziny zajęć i punkty ECTS}

\begin{center}
\begin{tabular}{|>{\centering\arraybackslash}p{2.0cm}
                |>{\centering\arraybackslash}p{2.0cm}
                |>{\centering\arraybackslash}p{2.0cm}
                |>{\centering\arraybackslash}p{2.4cm}
                |>{\centering\arraybackslash}p{2.4cm}
                |>{\centering\arraybackslash}p{2.0cm}
                |>{\centering\arraybackslash}p{1.4cm}|}
\hline
\rowcolor{tableHeader}
\textbf{Wykłady} & \textbf{Ćwiczenia} & \textbf{Laboratorium} &
\textbf{Z prowadzącym} & \textbf{Praca własna} & \textbf{Łącznie} & \textbf{ECTS} \\
\hline
30 h & 30 h & --- & 60 h & 65 h & 125 h & \textbf{5} \\
\hline
\end{tabular}
\end{center}

\section{Forma zajęć}

\begin{tabular}{ll}
  \hline
  \textbf{Forma zajęć} & \textbf{Sposób zaliczenia} \\
  \hline
  Laboratorium & Zaliczenie z oceną \\
  Wykład & Egzamin \\
  \hline
\end{tabular}

\section{Cel dydaktyczny}

Kurs traktuje o metodach metaheurystycznych. Jest to przegląd technik z tej dziedziny wraz z metodami testowania i porównywania różnych podejść do problemu optymalizacyjnego. Student zapozna się z właściwościami oraz zastosowaniem algorytmów heurystycznych. Wiedza wykładowa zostanie przetrenowana na laboratoriach, które pozwolą w sposób bardziej praktyczny zapoznać się z tematem i przygotują studenta do podejmowania decyzji odnośnie zastosowania metod metaheurystycznych lub wybrania innych technik odpowiednich do danego zadania..

\section{Przedmioty wprowadzające}

\begin{tabularx}{\textwidth}{lX}
  \hline
  \textbf{Przedmiot} & \textbf{Wymagane zagadnienia} \\
  \hline
  Algorytmy i struktury danych, podstawy języka C++, Analiza matematyczna, Matematyka dyskretna & --- \\
  \hline
\end{tabularx}

\section{Treści programowe}

\begin{enumerate}
  \item Złożoność obliczeniowa
  \item Zagadnienia optymalizacyjne
  \item Metoda dokładna
  \item Algorytm losowego próbkowania
  \item Algorytm wspinaczkowy
  \item Algorytm Tabu
  \item Symulowane Wyżarzanie
  \item Algorytm genetyczny
  \item Metody zrównoleglania obliczeń dla Algorytmu Genetycznego
  \item Strategia Ewolucyjna
  \item Programowanie Genetyczne
  \item Metody testowania i porównywania metod heurystycznych
\end{enumerate}

\section{Efekty kształcenia}

\subsection*{Wiedza}
\begin{itemize}
  \item Student zna i rozumie pojęcia związane z algorytmami heurystycznymi takie jak funkcja celu, funkcja selekcji, automatyczne dostosowanie zasięgu mutacji, programowanie genetyczne, metoda tabu, krzywa zbieżności. Student potrafi zidentyfikować problemy które można rozwiązać za pomocą algorytmów z dziedziny metaheurystyk. Student potrafi także zastosować metody statystyczne od analizy i porównani różnych metod rozwiązania tego samego problemu.
  \item Student zna i rozumie metody pracy na zbiorach danych.
\end{itemize}

\subsection*{Umiejętności}
\begin{itemize}
  \item Student potrafi zastosować aparat matematyczny do interpretowania pojęć z zakresu informatyki oraz rozwiązywania problemów o charakterze informatycznym
  \item Student potrafi  zastosować metody testowania i porównywania metod heurystycznych.
\end{itemize}

\subsection*{Kompetencje społeczne}
\begin{itemize}
  \item Student jest gotów do zastosowań informatyki na rzecz rozwoju nauki i społeczeństwa informacyjnego
  \item Student jest gotów do przekazywania społeczeństwu informacji i opinii dotyczących osiągnięć techniki i innych aspektów działalności inżynierskiej
\end{itemize}

\section{Kryteria oceny}

\begin{itemize}
  \item warsztaty
  \item Kryteria oceny
  \item Laboratorium/Projekt
  \item Niewielkie praktyczne zadania programistyczne składające się na duży projekt powstający w sposób przyrostowy. Każde zadanie jest oceniane pozytywnie lub nie i  punktowane w zależności od terminu dostarczenia. Projekt prowadzony z wykorzystaniem narzędzia git. Podstawowy język programowania to C++. Każde zadanie musi być obronione.
  \item Skala ocen:
  \item Poniżej 50\% - ndst
  \item Od 50\% - dst
  \item Od 60\% - dst+
  \item Od 70\% - db
  \item Od 80\% - db+
  \item Od 90\% - bdb
\end{itemize}

\section{Metody dydaktyczne}

Wykład, laboratoria, praca własna studenta.

\section{Literatura}

\textbf{Podstawowa:}
\begin{itemize}
  \item Iwona Karcz-Dulęba, "Nowoczesne metody optymalizacji globalnej", Wydawnictow EXIT, 2022
  \item Jarosław Arabas, "Wykłady z algorytmów genetycznych", 2004, Wydawnictwo WNT
\end{itemize}

\textbf{Uzupełniająca:}
\begin{itemize}
  \item „Algorytmy Genetyczne i ich zastosowania”, D. E. Goldberg, WNT, 1998
  \item „Algorytmy Genetyczne + Struktury Danych = programy ewolucyjne”, Z. Michalewicz, WNT, 1996
\end{itemize}

\end{document}
