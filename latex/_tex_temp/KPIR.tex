% ===========================================================
%  Sylabus: Komunikacja i protokoły dla Internetu rzeczy (KPIR)
% ===========================================================
\documentclass[12pt, a4paper]{article}

\usepackage[T1]{fontenc}
\usepackage[utf8]{inputenc}
\usepackage[polish]{babel}
\usepackage{lmodern}
\usepackage{microtype}
\usepackage[a4paper, top=2.5cm, bottom=2.5cm, left=2.5cm, right=2.5cm]{geometry}
\usepackage{xcolor}
\usepackage{graphicx}
\usepackage{booktabs}
\usepackage{tabularx}
\usepackage{longtable}
\usepackage{multirow}
\usepackage{array}
\usepackage{colortbl}
\usepackage{enumitem}
\usepackage{fancyhdr}
\usepackage{titlesec}
\usepackage{mdframed}
\usepackage[colorlinks=true, linkcolor=red!70!black, urlcolor=red!70!black]{hyperref}
\usepackage{eso-pic}
\usepackage{tikz}

\definecolor{pjatkRed}{RGB}{180,0,0}
\definecolor{pjatkGray}{RGB}{80,80,80}
\definecolor{pjatkLightGray}{RGB}{245,245,245}
\definecolor{tableHeader}{RGB}{220,220,220}

\pagestyle{fancy}\fancyhf{}
\renewcommand{\headrulewidth}{0.4pt}
\renewcommand{\footrulewidth}{0.4pt}
\fancyhead[L]{\small\textcolor{pjatkGray}{PJATK -- Filia w Gdańsku \textbar\ Informatyka}}
\fancyhead[R]{\small\textcolor{pjatkGray}{Sylabus: KPIR}}
\fancyfoot[C]{\small\thepage}

\titleformat{\section}{\large\bfseries\color{pjatkRed}}{\thesection.}{0.5em}{}
  [\color{pjatkRed}\rule{\linewidth}{0.8pt}]
\setlist{noitemsep, topsep=3pt, parsep=2pt}

\newmdenv[linecolor=pjatkRed, linewidth=1.2pt, backgroundcolor=pjatkLightGray,
  innerleftmargin=10pt, innerrightmargin=10pt, innertopmargin=8pt,
  innerbottommargin=8pt, roundcorner=4pt]{infobox}

\begin{document}

\AddToShipoutPictureBG{%
  \begin{tikzpicture}[remember picture, overlay]
    \node[opacity=0.5] at (current page.center) {%
      \includegraphics[width=14cm]{C:/Users/adamu/WebstormProjects/pj-studies/latex/PJATK_pl_sygnet_transparent-eps-converted-to}%
    };
  \end{tikzpicture}%
}

\begin{center}
  \includegraphics[height=2cm]{C:/Users/adamu/WebstormProjects/pj-studies/latex/PJATK_pl_poziom_1}\\[0.8cm]
  {\LARGE\bfseries\color{pjatkRed} SYLABUS PRZEDMIOTU}\\[0.8cm]
\end{center}

\begin{infobox}
\begin{tabularx}{\textwidth}{@{}lX@{}}
  \textbf{Nazwa przedmiotu:}  & {\bfseries Komunikacja i protokoły dla Internetu rzeczy} \\[3pt]
  \textbf{Kod przedmiotu:}    & KPIR \\[3pt]
  \textbf{Kierunek / Profil:} & Informatyka / praktyczny \\[3pt]
  \textbf{Tryb studiów:}      & stacjonarny \\[3pt]
  \textbf{Rok / Semestr:}     & 3 / 6 \\[3pt]
  \textbf{Charakter:}         & obowiązkowy \\[3pt]
  \textbf{Odpowiedzialny:}    & Dr Tadeusz Puźniakowski \\[3pt]
  \textbf{Wersja z dnia:}     & 19.02.2026 \\
\end{tabularx}
\end{infobox}

\vspace{1cm}

\section{Godziny zajęć i punkty ECTS}

\begin{center}
\begin{tabular}{|>{\centering\arraybackslash}p{2.0cm}
                |>{\centering\arraybackslash}p{2.0cm}
                |>{\centering\arraybackslash}p{2.0cm}
                |>{\centering\arraybackslash}p{2.4cm}
                |>{\centering\arraybackslash}p{2.4cm}
                |>{\centering\arraybackslash}p{2.0cm}
                |>{\centering\arraybackslash}p{1.4cm}|}
\hline
\rowcolor{tableHeader}
\textbf{Wykłady} & \textbf{Ćwiczenia} & \textbf{Laboratorium} &
\textbf{Z prowadzącym} & \textbf{Praca własna} & \textbf{Łącznie} & \textbf{ECTS} \\
\hline
30 h & 30 h & --- & 60 h & 65 h & 125 h & \textbf{5} \\
\hline
\end{tabular}
\end{center}

\section{Forma zajęć}

\begin{tabular}{ll}
  \hline
  \textbf{Forma zajęć} & \textbf{Sposób zaliczenia} \\
  \hline
  Laboratorium & Zaliczenie z oceną \\
  Wykład & Egzamin \\
  \hline
\end{tabular}

\section{Cel dydaktyczny}

Celem głównym przedmiotu jest przedstawienie budowy i działania systemów Internetu rzeczy a w szczególności metod i sposobów komunikowania w tym protokołów, które umożliwią wymianę informacji między czujnikami, efektorami i sterownikami.Cechą systemów Internetu rzeczy jest wykorzystywanie urządzeń o obniżonym poborze mocy oraz działanie w warunkach wymagających zapewnienia bezpieczeństwa transmisji i przetwarzania danych.

\section{Przedmioty wprowadzające}

\begin{tabularx}{\textwidth}{lX}
  \hline
  \textbf{Przedmiot} & \textbf{Wymagane zagadnienia} \\
  \hline
  Analiza Matematyczna, Algebra Liniowa i Geometria, Fizyka, Użytkowanie komputerów, Wstęp do informatyki i architektury komputerów, Programowanie 1,  Elektronika. & Umiejętność programowania w języku typu C, C++ lub Java. \\
  Znajomość regulaminu i zasad BHP obowiązujących w laboratorium. & --- \\
  \hline
\end{tabularx}

\section{Treści programowe}

\begin{enumerate}
  \item Wprowadzenie i sprawy organizacyjne. Wprowadzenie do zajęć projektowych i laboratoryjnych.
  \item System wbudowany a system IoT.Analiza dedykowane systemu IoT w  CISCO PacketTracer.
  \item Urządzenia i układy IoT.
  \item Projekt prostego systemu IoT w CISCO Packet Tracer.
  \item Architektura systemów IoT.
  \item Projekt systemu IoT z wykorzystaniem wybranego mikrokontrolera lub komputera jednoukładowego.
  \item Typy protokołów IoT.
  \item Konfiguracja i oprogramowanie systemu IoT.
  \item Zaawansowany protokół kolejkowania wiadomości.
  \item Walidacja i testowanie systemy IoT.
  \item Standardy komunikacyjne dla IoT.
  \item Wdrożenie i prezentacja systemu IoT.
\end{enumerate}

\section{Efekty kształcenia}

\subsection*{Wiedza}
\begin{itemize}
  \item Student zna i rozumie zaawansowane pojęcia w zakresie techniki cyfrowej i systemów cyfrowych, architektury i organizacji systemów komputerowych, architektura wieloprocesorowych oraz programowania na poziomie assemblera
  \item Student zna i rozumie zaawansowane pojęcia z zakresu mikrokontrolerów i systemów wbudowanych oraz metodyk ich projektowania; rozumie powiązanie informatyki z problemami automatyki i robotyki oraz potrzebę przenoszenia ich dobrych praktyk na grunt informatyki
\end{itemize}

\subsection*{Umiejętności}
\begin{itemize}
  \item Student potrafi zaprojektować złożone układy sekwencyjne i kombinacyjne, obliczyć reprezentacje liczb całkowitych i rzeczywistych oraz wykonać podstawowe operacji arytmetyczne na tych reprezentacjach, a także pisać proste programy na poziomie asemblera
  \item Student potrafi przeanalizować, zsyntezować i oprogramować system wbudowany, z uwzględnieniem zasad bezpieczeństwa i niezawodności oraz sporządzić jego dokumentację
\end{itemize}

\subsection*{Kompetencje społeczne}
\begin{itemize}
  \item Student jest gotów do samodzielnego zgłębiania wiedzy z zakresu systemów wbudowanych oraz dzielenia się nią
\end{itemize}

\section{Kryteria oceny}

\begin{itemize}
  \item wykład z elementami dyskusji z prezentacją multimedialną
  \item rozwiązywanie zadań
  \item analiza przypadków
  \item badania symulacyjne
  \item Kryteria oceny
  \item Laboratorium/
  \item - pozytywna ocena za mikroprojekty,
  \item - pozytywna ocena za ćwiczenia laboratoryjne,
  \item - egzamin pisemny składający się z co najwyżej dziesięciu zadań
\end{itemize}

\section{Metody dydaktyczne}

Wykład, laboratoria, praca własna studenta.

\section{Literatura}

\textbf{Podstawowa:}
\begin{itemize}
  \item Gary Smart, “Practical Python Programming for IoT. Build advanced IoT projects using a Raspberry Pi 4, MQTT, RESTful APIs, WebSockets, and Python 3”, SBN Ebooka: 978-18-389-8283-6, 9781838982836, 2020
  \item Joey Bernal, Bharath Sridhar, “Industrial IoT for Architects and Engineers. Architecting secure, robust, and scalable industrial IoT solutions with AWS”, ISBN Ebooka: 978-18-032-4619-2, 9781803246192, 2023
  \item Michael Roshak, “Artificial Intelligence for IoT Cookbook. Over 70 recipes for building AI solutions for smart homes, industrial IoT, and smart cities”, ISBN Ebooka: 978-18-389-8649-0, 9781838986490, 2021
  \item Vedat Ozan Oner, Developing IoT Projects with ESP32. Automate your home or business with inexpensive Wi-Fi devices, ISBN Ebooka: 978-18-386-4280-8, 9781838642808, 2021
  \item Gaston C. Hillar, “Hands-On MQTT Programming with Python. Work with the lightweight IoT protocol in Python”, ISBN Ebooka: 978-17-891-3781-1, 9781789137811, 2018
\end{itemize}

\textbf{Uzupełniająca:}
\begin{itemize}
  \item Wybrane serwisy internetowe
\end{itemize}

\end{document}
