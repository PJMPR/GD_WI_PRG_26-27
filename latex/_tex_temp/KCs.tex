% ===========================================================
%  Sylabus: Kryminalistyka Cyfrowa (KC)
% ===========================================================
\documentclass[12pt, a4paper]{article}

\usepackage[T1]{fontenc}
\usepackage[utf8]{inputenc}
\usepackage[polish]{babel}
\usepackage{lmodern}
\usepackage{microtype}
\usepackage[a4paper, top=2.5cm, bottom=2.5cm, left=2.5cm, right=2.5cm]{geometry}
\usepackage{xcolor}
\usepackage{graphicx}
\usepackage{booktabs}
\usepackage{tabularx}
\usepackage{longtable}
\usepackage{multirow}
\usepackage{array}
\usepackage{colortbl}
\usepackage{enumitem}
\usepackage{fancyhdr}
\usepackage{titlesec}
\usepackage{mdframed}
\usepackage[colorlinks=true, linkcolor=red!70!black, urlcolor=red!70!black]{hyperref}
\usepackage{eso-pic}
\usepackage{tikz}

\definecolor{pjatkRed}{RGB}{180,0,0}
\definecolor{pjatkGray}{RGB}{80,80,80}
\definecolor{pjatkLightGray}{RGB}{245,245,245}
\definecolor{tableHeader}{RGB}{220,220,220}

\pagestyle{fancy}\fancyhf{}
\renewcommand{\headrulewidth}{0.4pt}
\renewcommand{\footrulewidth}{0.4pt}
\fancyhead[L]{\small\textcolor{pjatkGray}{PJATK -- Filia w Gdańsku \textbar\ Informatyka}}
\fancyhead[R]{\small\textcolor{pjatkGray}{Sylabus: KC}}
\fancyfoot[C]{\small\thepage}

\titleformat{\section}{\large\bfseries\color{pjatkRed}}{\thesection.}{0.5em}{}
  [\color{pjatkRed}\rule{\linewidth}{0.8pt}]
\setlist{noitemsep, topsep=3pt, parsep=2pt}

\newmdenv[linecolor=pjatkRed, linewidth=1.2pt, backgroundcolor=pjatkLightGray,
  innerleftmargin=10pt, innerrightmargin=10pt, innertopmargin=8pt,
  innerbottommargin=8pt, roundcorner=4pt]{infobox}

\begin{document}

\AddToShipoutPictureBG{%
  \begin{tikzpicture}[remember picture, overlay]
    \node[opacity=0.5] at (current page.center) {%
      \includegraphics[width=14cm]{C:/Users/adamu/WebstormProjects/pj-studies/latex/PJATK_pl_sygnet_transparent-eps-converted-to}%
    };
  \end{tikzpicture}%
}

\begin{center}
  \includegraphics[height=2cm]{C:/Users/adamu/WebstormProjects/pj-studies/latex/PJATK_pl_poziom_1}\\[0.8cm]
  {\LARGE\bfseries\color{pjatkRed} SYLABUS PRZEDMIOTU}\\[0.8cm]
\end{center}

\begin{infobox}
\begin{tabularx}{\textwidth}{@{}lX@{}}
  \textbf{Nazwa przedmiotu:}  & {\bfseries Kryminalistyka Cyfrowa} \\[3pt]
  \textbf{Kod przedmiotu:}    & KC \\[3pt]
  \textbf{Kierunek / Profil:} & Informatyka / praktyczny \\[3pt]
  \textbf{Tryb studiów:}      & niestacjonarny \\[3pt]
  \textbf{Rok / Semestr:}     & 3 / 6 \\[3pt]
  \textbf{Charakter:}         & obowiązkowy \\[3pt]
  \textbf{Odpowiedzialny:}    &  \\[3pt]
  \textbf{Wersja z dnia:}     & 19.02.2026 \\
\end{tabularx}
\end{infobox}

\vspace{1cm}

\section{Godziny zajęć i punkty ECTS}

\begin{center}
\begin{tabular}{|>{\centering\arraybackslash}p{2.0cm}
                |>{\centering\arraybackslash}p{2.0cm}
                |>{\centering\arraybackslash}p{2.0cm}
                |>{\centering\arraybackslash}p{2.4cm}
                |>{\centering\arraybackslash}p{2.4cm}
                |>{\centering\arraybackslash}p{2.0cm}
                |>{\centering\arraybackslash}p{1.4cm}|}
\hline
\rowcolor{tableHeader}
\textbf{Wykłady} & \textbf{Ćwiczenia} & \textbf{Laboratorium} &
\textbf{Z prowadzącym} & \textbf{Praca własna} & \textbf{Łącznie} & \textbf{ECTS} \\
\hline
30 h & --- & 30 h & 60 h & 65 h & 125 h & \textbf{5} \\
\hline
\end{tabular}
\end{center}

\section{Forma zajęć}

\begin{tabular}{ll}
  \hline
  \textbf{Forma zajęć} & \textbf{Sposób zaliczenia} \\
  \hline
  Wykład & Egzamin \\
  \hline
\end{tabular}

\section{Cel dydaktyczny}

Kryminalistyka cyfrowa to dziedzina zajmująca się śledztwem i badaniem dowodów związanych z przestępstwami cyfrowymi. Jest to rodzaj kryminalistyki, która koncentruje się na zdolności do wykorzystywania technologii i nauki do zgromadzenia dowodów, które mogą być przedstawione w sądzie. Kryminalistyka Cyfrowa odgrywa kluczową rolę w walce z cyberprzestępczością, pomagając organom ścigania zrozumieć, jak doszło do przestępstwa, kto może za nie odpowiadać i jakie kroki mogą być podjęte, aby zapobiec przyszłym incydentom. Główne cele dydaktyczne to badanie sieci, czyli umiejętność analizy ruchu sieciowego w celu wykrycia nieprawidłowości lub dowodów na działania przestępcze.Inny cel to analiza logów systemowych, czyli zebranie informacji o tym, co się działo na danym systemie w określonym czasie. Analiza tych logów może pomóc w ustaleniu, co się stało podczas incydentu.

\section{Treści programowe}

\begin{enumerate}
  \item Wykład:
  \item Wprowadzenie do Kryminalistyki Cyfrowej
  \item Bezpieczeństwo protokołów i urządzeń warstwy 1-2 modelu OSI
  \item Bezpieczeństwo protokołów i urządzeń warstwy 3-4 modelu OSI
  \item Bezpieczeństwo protokołów i urządzeń warstwy 5-7 modelu OSI
  \item Metody analizy ruchu sieciowego (Network analysis)
  \item Metody analizy oprogramowania (Software analysis)
  \item Metody analizy mediów (Media analysis)
  \item Metody analizy sprzętu (Hardware analysis)
  \item Ćwiczenia:
  \item [Linux 1] Operacje na plikach
  \item [Linux 1] Operacje identyfikacji oraz kodowania danych
  \item [Linux 1] Analiza logów
  \item [Linux 2] Mechanizmy systemowe i skrypty Bash
  \item [Linux 2] Analiza danych z pliku
  \item [Linux 2] Administracja systemem GNU/Linux
  \item [Linux 2] Zarządzanie procesami
  \item [Sieci 2] Protokół http
  \item [Sieci 2] Protokół DNS
  \item [Sieci 2] Protokoły poczty
  \item [Sieci 2] Inne protokoły
  \item [Sieci 2] Bezpieczna komunikacja
  \item [Sieci 2] Bezpieczeństwo sieci WiFi
\end{enumerate}

\section{Efekty kształcenia}

\subsection*{Wiedza}
\begin{itemize}
  \item Student zna i rozumie anatomię kluczowych zagrożeń Cyber
  \item Student zna i rozumie podstawowe środki ochrony i reakcji na kluczowe zagrożenia Cyber
\end{itemize}

\subsection*{Umiejętności}
\begin{itemize}
  \item Student potrafi przygotować zestaw środków ochrony informacji odpowiednich dla różnych typów zagrożeń
  \item Student potrafi wskazać mocne i słabe strony systemów ochrony informacji
\end{itemize}

\subsection*{Kompetencje społeczne}
\begin{itemize}
  \item Student jest gotów do pracy zespołowej w dziale SOC
\end{itemize}

\section{Kryteria oceny}

\begin{itemize}
  \item Zadania CTF
  \item Laboratoria
  \item Kryteria oceny
  \item Laboratorium/Projekt
  \item raport z wykonanych zadań laboratoryjnych, kolokwia
  \item Kryteria oceny:niedostateczny: < 50\% możliwych do zdobycia punktów dostateczny  :     50\%-60\% możliwych do zdobycia punktów dostateczny+ :    61\%- 70\% możliwych do zdobycia punktówdobry :                71\%-80\% możliwych do zdobycia punktówdobry +:              81\%-90\% możliwych do zdobycia punktów bardzo dobry  :   >90\%  możliwych do zdobycia punktów
  \item Kryteria oceny:niedostateczny: < 50\% poprawnie rozwiązanych zadań egz.dostateczny  :     50\%-60\% poprawnie rozwiązanych zadań egz.dostateczny+ :    61\%- 70\% poprawnie rozwiązanych zadań egz.dobry :                71\%-80\% poprawnie rozwiązanych zadań egz.dobry +:              81\%-90\% poprawnie rozwiązanych zadań egz.bardzo dobry  :   >90\%  poprawnie rozwiązanych zadań zaliczeniowych egz.
\end{itemize}

\section{Metody dydaktyczne}

Wykład, laboratoria, praca własna studenta.

\section{Literatura}

\textbf{Podstawowa:}
\begin{itemize}
  \item William Stallings, Lawrie Brown - Bezpieczeństwo systemów informatycznych. Zasady i praktyka. (ang. Computer Security: Principles and Practice), Wydanie IV. Tom 1 i 2, Helion 2019
\end{itemize}

\textbf{Uzupełniająca:}
\begin{itemize}
  \item Brak danych.
\end{itemize}

\end{document}
